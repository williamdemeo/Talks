%\documentclass[xcolor=dvipsnames,9pt]{beamer} 
\documentclass[xcolor=dvipsnames,9pt,hide notes,mathserif]{beamer}

\usepackage{pgfpages}
\usepackage{listings}
%\usepackage{enumitem}

%% For creating a handout:
%\pgfpagesuselayout{4 on 1}[border shrink=5mm]
%\mode<handout>{\setbeamercolor{background canvas}{bg=black!5}}

%% For creating notes for the speaker:
%\setbeameroption{notes on second screen}
%\setbeameroption{show notes}

\setbeamerfont{structure}{family=\rmfamily,shape=\scshape} 
\usepackage{graphicx}
\usepackage{tikz}
\usepackage{scalefnt}
\usepackage{amsmath}%
\usepackage{amsfonts}%
\usepackage{amssymb}%
%(wjd) added stmaryrd and enumerate packages
\usepackage{stmaryrd,enumerate}
\usepackage{graphicx}
\usepackage{comment}
\usetikzlibrary{matrix,arrows}

\usepackage{mathrsfs,textcomp}
\setbeamertemplate{navigation symbols}{}
\usepackage{verbatim}
\usepackage[mathcal]{euscript}

% This changes the color of alerted text to blue:
\definecolor{MyDarkBlue}{rgb}{0.2,0.2,0.7}
\definecolor{olivegreen}{cmyk}{0.64,0,0.95,0.40} % PANTONE 582
\setbeamercolor{alerted text}{fg=blue}
\newcommand{\emphcyan}[1]{\textcolor{MyDarkBlue}{\textbf{#1}}}
%\renewcommand{\alert}[1]{\textcolor{olivegreen}{\emph{#1}}}
\renewcommand{\alert}[1]{\textcolor{olivegreen}{#1}}
%\renewcommand{\alert}[1]{\textbf{{\emph{#1}}}}
% (default is red, but my slides are green and I don't like red and green together)

%\usecolortheme[named=OliveGreen]{structure} 
\usecolortheme[named=olivegreen]{structure} 
\setbeamertemplate{items}[square] 
\setbeamertemplate{blocks}[rounded][shadow=false]


\newcommand{\Hawaii}{Hawai\kern.05em`\kern.05em\relax i}
\newcommand{\Manoa}{M\=anoa}
\newcommand{\dotsize}{.8pt}
\newcommand{\FLRP}{{\small FLRP}}
\newcommand{\ssubnormal}{\ensuremath{\vartriangleleft}}
\newcommand{\core}{\ensuremath{\mathrm{core}}}
\newcommand{\bs}{\ensuremath{\backslash}}
\newcommand{\GAP}{{\small GAP}}
\newcommand{\cd}{\ensuremath{\otimes}}
\newcommand{\con}[1]{\ensuremath{\langle #1 \rangle}}
\newcommand{\ii}[1]{{\it #1}}
\newcommand{\power}[1]{\ensuremath{\mathscr{P}(#1)}}
\newcommand{\scrA}{\ensuremath{\mathscr{A}}}
\newcommand{\bM}{\ensuremath{\mathbf{M}}}
\newcommand{\Mn}{\ensuremath{\mathbf{M}_n}}
\newcommand{\bF}{\ensuremath{\mathbf{F}}}
\newcommand{\bE}{\ensuremath{\mathbf{E}}}
\newcommand{\bR}{\ensuremath{\mathbf{R}}}
\newcommand{\bA}{\ensuremath{\mathbf{A}}}
\newcommand{\bG}{\ensuremath{\mathbf{G}}}
\newcommand{\bH}{\ensuremath{\mathbf{H}}}
\newcommand{\bK}{\ensuremath{\mathbf{K}}}
\newcommand{\bL}{\ensuremath{\mathbf{L}}}
\newcommand{\bB}{\ensuremath{\mathbf{B}}}
\newcommand{\svert}{\ensuremath{\; \vert \; }}
\newcommand{\Z}{\ensuremath{\mathbb{Z}}}
\newcommand{\sE}{\ensuremath{\mathcal{E}}}
\newcommand{\sO}{\ensuremath{\mathcal{O}}}
\newcommand{\sH}{\ensuremath{\mathcal{H}}}
\newcommand{\sS}{\ensuremath{\mathcal{S}}}
\newcommand{\sL}{\ensuremath{\mathcal{L}}}
\newcommand{\bN}{\ensuremath{\mathbf{N}}}
\newcommand{\bX}{\ensuremath{\mathbf{X}}}
\newcommand{\resB}{\ensuremath{|_{_B}}}
\newcommand{\eps}{\ensuremath{\varepsilon}}
\newcommand{\sA}{\ensuremath{\mathcal{A}}}
\newcommand{\sB}{\ensuremath{\mathcal{B}}}
\newcommand{\sC}{\ensuremath{\mathcal{C}}}
\newcommand{\SSS}{\text{\emphslb{S}}}
\newcommand{\id}{\mbox{id}}
\newcommand{\Hom}{\mbox{Hom}}
\newcommand{\End}{\ensuremath{\mathrm{End}}}
\newcommand{\bEnd}{\ensuremath{\mathbf{End}}}
\newcommand{\Aut}{\ensuremath{\mathrm{Aut}}}
\newcommand{\bAut}{\ensuremath{\mathbf{Aut}}}
\newcommand{\Cg}{\ensuremath{\mathrm{Cg}}}
\newcommand{\Con}{\ensuremath{\mathrm{Con}\,}}
\newcommand{\bCon}{\ensuremath{\mathbf{Con}\,}}
\newcommand{\Sub}{\mbox{Sub}}
\newcommand{\bSub}{\ensuremath{\mathbf{Sub}}}
\newcommand{\CSub}[1]{\ensuremath{\mathbf{CSub}[#1]}}
\newcommand{\csub}{\ensuremath{\mbox{CSub}}}
\newcommand{\Stab}{\mbox{Stab}}
\newcommand{\bStab}{\ensuremath{\mathbf{Stab}}}
\newcommand{\X}{\ensuremath{\mathbf{X}}}
\newcommand{\image}{\mbox{im}}
\newcommand{\Eq}{\mbox{Eq}}
\newcommand{\bEq}{\ensuremath{\mathbf{Eq}}}
\newcommand{\bEqX}{\ensuremath{\mathbf{Eq}(X)}}
\newcommand{\idemdec}{\ensuremath{\mbox{Idemdec}(X)}}
\newcommand{\EqX}{\ensuremath{\mbox{Eq}(X)}}
\newcommand{\upalpha}{\ensuremath{\alpha^{\uparrow}}}
\newcommand{\downalpha}{\ensuremath{\alpha^{\downarrow}}}
\newcommand{\upbeta}{\ensuremath{\beta^{\uparrow}}}
\newcommand{\downbeta}{\ensuremath{\beta^{\downarrow}}}
\newcommand{\meet}{\ensuremath{\wedge}}
\newcommand{\join}{\ensuremath{\vee}}
\newcommand{\Meet}{\ensuremath{\bigwedge}}
\renewcommand{\Join}{\ensuremath{\bigvee}}
\renewcommand{\leq}{\ensuremath{\leqslant}}
\renewcommand{\nleq}{\ensuremath{\nleqslant}}
\renewcommand{\geq}{\ensuremath{\geqslant}}
\newcommand{\pb}{\ensuremath{\protect{|}}}

\newcommand{\code}[1]{{\small {\tt #1}}}
\newcommand{\<}{\langle}	     %% left angle for order sequences <a,b>
\renewcommand{\>}{\rangle}	     %% right angle
\newcommand{\lb}{\ensuremath{\llbracket}}
\newcommand{\rb}{\ensuremath{\rrbracket}}

\newcommand{\Palfy}{P\'alfy}
\newcommand{\Pudlak}{Pudl\'ak}
\newcommand{\PAP}{P\'alfy-Pudl\'ak}
\newcommand{\Tuma}{T\r{u}ma}
\newcommand{\res}{\ensuremath{\upharpoonright}}  % restriction


\mode<presentation>{\usetheme{boxes}}  %boxes,Pittsburgh JuanLesPins, PaloAlto, Singapore, Szeged, Warsaw, Boadilla
%\usetheme{Madrid}}
%\usetheme{boxes}  %boxes,Pittsburgh JuanLesPins, PaloAlto, Singapore, Szeged, Warsaw, Boadilla

\usepackage[english]{babel}
\usepackage[latin1]{inputenc}
\usepackage{times}
\usepackage[T1]{fontenc}
% Or whatever. Note that the encoding and the font should match. If T1
% does not look nice, try deleting the line with the fontenc.

\title{CSPs of Finite Commutative Idempotent Binars}

\author[William DeMeo]{William DeMeo\\
  {\small \url{williamdemeo@gmail.com}}\\[4pt]
%  {\small Iowa State University}\\[4pt]
  {\footnotesize joint work with}\\[4pt] 
  Cliff Bergman\\
  Jiali Li
}
\institute{Iowa State University}

\date[BLAST 2015]{ % (optional, should be abbreviation of conference name)
  BLAST 2015\\{\small University of North Texas}\\
  {\small June 7--12}}

\subject{Universal Algebra; Lattice Theory; CSP.}% (optional) inserted into PDF info catalog.

\begin{document}
\thicklines

\includeonlyframes{titlepage,intro,intro2,intro3,intro4,intro5,intro6,problem1,cib,general1,cube,cp,general2,examples1,examples2,examples3,examples4,others}


\frame[label=titlepage]{
  \titlepage
}


%%%%%%%%%%%%%%%%%%%%%%%%%%%%%%%%%%%%%%%%%%%%%%%%%%%%%%%%%%%
%% 1: CSP Intro 1
\frame[label=intro]{
  \frametitle{Constraint Satisfaction Problems}
%  \underline{Input}
  {\bf Input}
  \begin{itemize}
  \item \emph{variables:} $V = \{v_1, v_2, \dots\}$
  \item \emph{domain:}  $D$
  \item \emph{constraints:} $C_1, C_2, \dots$
  \end{itemize}
  \vskip1cm
%  \underline{Output}
 {\bf Output}
  \begin{itemize}
  \item ``yes'' if there is a \emph{solution}  
    \[
    \sigma : V \rightarrow D \quad 
    \text{    {\small (an assignment of values to variables that satisfies all $C_i$)}}
    \] 

  \item ``no'' otherwise
  \end{itemize}

}

\note{
  In computer science, the Boolean Satisfiability Problem 
  (abbreviated as SAT) is the problem of determining if there exists an
  \emph{interpretation} that satisfies a given Boolean formula. 

  In other words, it asks whether the \emph{variables} of a given Boolean
  formula can be consistently replaced by the values TRUE or FALSE in such a
  way that the formula evaluates to TRUE. 

  If this is the case, then the formula is called \emph{satisfiable}. 
  On the other hand, if no such assignment exists, the function expressed by
  the formula is identically FALSE for all possible variable assignments and
  the formula is unsatisfiable. For example, the formula ``$x \meet \neg y$'' is
  satisfiable because one can find the values, namely $x = 1$ and $y = 0$, 
  which make $(x \meet \neg y) = 1$. 

  In contrast, ``$x \meet \neg x$'' is unsatisfiable. 
}


%%%%%%%%%%%%%%%%%%%%%%%%%%%%%%%%%%%%%%%%%%%%%%%%%%%%%%%%%%%
%% 2: CSP intro 2
\frame[label=intro2]{
  \frametitle{Constraint Satisfaction Problems}
  \framesubtitle{Example: 3-SAT}
%  \underline{Input}
  {\bf Input}
  \begin{itemize}
  \item \emph{variables:} $V = \{v_1, \dots, v_n\}$
  \item \emph{domain:} $D = \{0, 1\}$
  \item \emph{constraints:} a formula, say,
    \[
    f(v_1, \dots, v_n) = (v_1 \join v_2 \join \neg v_3) \meet 
    (\neg v_1 \join v_3 \join v_4) \meet  \cdots
    \] 
  \end{itemize}
%  \underline{Output}
 {\bf Output}
  \begin{itemize}
  \item ``yes'' if there is a solution:  $\sigma : V \rightarrow D$ such that
    \[
    f(\sigma v_1,\dots, \sigma v_n) = 1
    \]
  \item ``no'' otherwise
  \end{itemize}
}

%%%%%%%%%%%%%%%%%%%%%%%%%%%%%%%%%%%%%%%%%%%%%%%%%%%%%%%%%%%
%% 3: CSP intro 3
\frame[label=intro3]{
  \frametitle{Constraint Satisfaction Problems}
  \framesubtitle{Example: NAE-SAT}
%  \underline{Input}
  {\bf Input}
  \begin{itemize}
  \item  \emph{variables:} $V = \{v_1, \dots, v_n\}$
  \item \emph{domain:} $D = \{0, 1\}$
  \item \emph{constraints:} $(s_1,C_1), (s_2,C_2),\dots $ {\small of the form}
    \begin{align*}
    s &= (i, j, k) \in \{1, \dots, n\}^3  \qquad \text{ {\small (scopes)}}\\
    C &= \neg (v_i = v_j = v_k)%  \qquad \text{ {\small (constraints)}}
    \end{align*}
    %% $(s_1,C_1), (s_2,C_2), \dots$ where 
    %% \begin{itemize}
    %% \item $s_i = (i_1, i_2, i_3) \in [n]^3$ {\small ``scopes''}
    %% \item $C_i = \neg (v_{i_1} = v_{i_2} = v_{i_3})$
    %% \end{itemize}
  \end{itemize}
  \onslide<2->
  In terms of relational structures...
  \begin{align*}
  \text{Let } \quad S &:= \{(v_i, v_j, v_k) : (i, j, k) 
  \text{ {\small is a scope} }\}
  \subseteq V^3\\
  R &:= \{(0,0,1), (0,1,0), (0,1,1), (1,0,0), (1,0,1), (1,1,0)\}
  \subseteq D^3
  \end{align*}
  %% \end{overprint}
\onslide<3->
  Then a solution $\sigma$ must satisfy
  ``$\sigma \sC \subseteq \sR$''
\[
\text{that is, } \quad (x,y,z) \in S \; \Longrightarrow \;
(\sigma x, \sigma y, \sigma z) \in R
\]
\onslide<4->
\emph{Solutions are homomorphisms!}
\[\sigma : \<V, S\> \rightarrow \<D, R\>\]
}

%%%%%%%%%%%%%%%%%%%%%%%%%%%%%%%%%%%%%%%%%%%%%%%%%%%%%%%%%%%
%% 4: CSP intro 4
\frame[label=intro4]{
  \frametitle{CSP: relational formulation}

  Let $\bbD = \<D, \sR\>$ be a relational structure.\\[4pt]
  $\CSP (\bbD)$ is the decision problem with\\[6pt]
%  \underline{Input}
  {\bf Input}
  \begin{itemize}
  \item  A structure $\bbV = \<V, \sC\>$ \emph{similar}  to $\bbD$.
  \end{itemize}
%  \underline{Output}
 {\bf Output}
  \begin{itemize}
  \item ``yes'' if there is a homomorphism $\sigma : \bbV \rightarrow \bbD$
  \item ``no'' otherwise
  \end{itemize}
\vskip5mm

  \begin{overprint}
  \onslide<2->
  Alternatively, let $\Rightarrow$ be the binary relation on
  similar structures
  \[
  \bbV \Rightarrow \bbD \quad 
  \text{ iff there is a homomorphism $\sigma: \bbV \rightarrow \bbD$}
  \]
  Then the CSP is the membership problem for the set
  \[
  \CSP(\bbD) := \{\bbV : \bbV \Rightarrow \bbD\}
  \]
  \onslide<3->
  We call $\bbD$ ``tractable'' if there is a
  polynomial-time algorithm for
  $\CSP(\bbD)$.
  %%  in time polynomial in the size of $\bbV$.
  %% any instance $\bbV$\\ 
  %%   of $\CSP(\bbD)$ in time polynomial in the size of $\bbV$.
  \end{overprint}
}

%%%%%%%%%%%%%%%%%%%%%%%%%%%%%%%%%%%%%%%%%%%%%%%%%%%%%%%%%%%
%% 5: CSP intro 5
\frame[label=intro5]{
%  \frametitle{Constraint Satisfaction Problems}
  \frametitle{CSP: algebraic formulation}

  Let $\bbD = \<D, \sR\>$ be a relational structure. \\[4pt]
  For $R \subseteq \sR$ %of relations on $D$, 
  define the \emph{polymorphisms} of $R$,
  \[
  \sansPoly(R) := \{f:D^k \rightarrow D \mid
  f(\rho)\subseteq \rho \text{ for every } \rho \in R \}
  \]
  \begin{overprint}
  \onslide<2->
  that is, 
    $f\in \sansPoly(R)$ iff for every $\rho\in R$ {\small (say, $n$-ary)}
  \vskip2mm 
  \begin{tabular}{rcl}
    $(a_1, b_1, \dots, z_1)$ &$\in$ & $\rho$\\
    &$\vdots$ & \\
    $(a_k, b_k, \dots, z_k)$ & $\in$ & $ \rho$\\[1mm]
    \hline\\[-2mm]
    $(f(a_1, \dots, a_k), \dots, f(z_1, \dots, z_k))$ & $\in $ & $\rho$
  \end{tabular}
  \onslide<3->
  \vskip4mm
  Define the algebra $\bD := \<D, \sansPoly(R)\>$.\\[6pt]  
  We call $\bD$ ``tractable'' if the corresponding 
  structure $\<D, R\>$ is tractable.
  \end{overprint}
  \onslide<4-> For $F$ a set of operations on $D$,
  define the \emph{relational clone} of $F$,
  \[
  \sansRel(F) := \{\rho \subseteq D^n \mid
  f(\rho)\subseteq \rho \text{ for every } f \in F \}
  \]
  Let $\bar{R} := \sansRel (\sansPoly (R))$ be the ``closure'' of $R$.
  \onslide<5->
  \[\text{Then,} \quad \CSP\<D, R\> \leq_P \CSP\<D, \bar{R}\> \]
  %% \onslide<6->
  %% \[\text{{\bf Theorem:}} \quad \CSP\<D, \bar{R}\> \leq_P \CSP\<D, R\> \]
  %% \begin{theorem}
  %%   \[\CSP\<D, \bar{R}\> \leq_P \CSP\<D, R\> \]
  %% \end{theorem}

}

%%%%%%%%%%%%%%%%%%%%%%%%%%%%%%%%%%%%%%%%%%%%%%%%%%%%%%%%%%%
%% 6: CSP intro 6
\frame[label=intro6]{
%  \frametitle{Constraint Satisfaction Problems}
  \frametitle{CSP: algebraic formulation}

  \begin{theorem}[Bodnar\v{c}uk et al.; Geiger, 1968]
    Let $R$ be a set of relations on a finite set.\\[4pt]
    Then $\bar{R} := \sansRel (\sansPoly (R))$ is the set of relations
    that are \emph{pp-definable} from $R$.
  \end{theorem}
  \vskip5mm
  \onslide<2->
  \begin{theorem}
  Let $S\subseteq R$ be sets of relations.
  \begin{itemize}
  \item<2-> $\CSP(S) \leq_P \CSP(R)$ 
    \onslide<3->{{\small ($\CSP(S)$
    reducible to $\CSP(R)$)}}
  \item<4-> $\CSP(R) \equiv_P \CSP(\bar{R})$
    \onslide<5->{{\small ($\CSP(\bar{R})$
    reducible to $\CSP(R)$!)}}
  \end{itemize}
  \end{theorem}
  \vskip5mm
  \onslide<6->{
  \alert{Corollary} % If $R$ and $S$ are sets of relations on $D$, then
  $\quad \<D, \sansPoly(R)\> = \<D, \sansPoly(S)\> \quad \Longrightarrow \quad
  \CSP(R) \equiv_P \CSP(S)$
  \vskip5mm
  \emph{The algebras determine the complexity of
  the corresponding constraint satisfaction problem!} }
}





%%%%%%%%%%%%%%%%%%%%%%%%%%%%%%%%%%%%%%%%%%%%%%%%%%%%%%%%%%%
%% 7: CSP dichotomy conjecture
\frame[label=problem1]{
  \frametitle{~}
  %  \framesubtitle{CSP dichotomy conjecture}

  \begin{block}{General Problem}
  %% \alert{General Problem:}
  Find properties (of algebras) that characterize
  the complexity of CSPs.
  \end{block}

  \begin{overprint}
    \onslide<1|handout:0>
    \begin{block}{csp dichotomy conjecture}
      For a (finite, idempotent) algebra $\mathbf A$...
      \[
      \CSP (\mathbf A) \text{ is tractable } \; \Longleftrightarrow \;  \bA
      \text{ has a weak-nu term operation}  \phantom{\quad \text{ ?}}
      \]
    \end{block}
    \onslide<2|handout:0>
    \begin{block}{csp dichotomy conjecture}
      For a (finite, idempotent) algebra $\mathbf A$...
      \[
      \CSP (\mathbf A) \text{ is tractable } \; \Longrightarrow \;  \bA
      \text{ has a weak-nu term operation}  \quad \checkmark
      \]
    \end{block}
    \onslide<3-|handout:1>
    \begin{block}{csp dichotomy conjecture}
      For a (finite, idempotent) algebra $\mathbf A$...
      \[
      \CSP (\mathbf A) \text{ is tractable } \; \Longleftarrow \;  \bA
      \text{ has a weak-nu term operation}  \quad \text{ (?)}
      \]
    \end{block}
  \end{overprint}

  \vskip3mm

  \begin{overprint}
    \onslide<2|handout:0>The left-to-right direction is known. \\[5pt]
    \onslide<3|handout:0>The right-to-left direction is open. \\[5pt]
    \onslide<4-|handout:1>
    A \alert{weak near unanimity} (weak-nu) term operation is one that satisfies 
    \begin{align*}
      t(x, x, \dots, x)&\approx x \quad \text{ {\small (idempotent)}}\\[4pt]
      t(y, x, \dots, x) &\approx t(x, y, \dots, x) \approx \dots \approx
      t(x, x, \dots, y)
    \end{align*}
  \end{overprint}
  
  \vskip3mm

  \onslide<5->{
    A \emph{binary} operation $t(x,y)$ is weak-nu if 
    \begin{align*}
      t(x, x)&\approx x \qquad \text{  {\small (idempotent)}}\\[4pt]
      t(y, x) &\approx t(x, y)  \quad \text{  {\small (commutative)}}
    \end{align*}
    So let's try to prove (?) for 
    \alert{commutative idempotent binars}.
  }
}




%%%%%%%%%%%%%%%%%%%%%%%%%%%%%%%%%%%%%%%%%%%%%%%%%%%%%%%%%%%
%% 8: Commutative Idempotent Binars
\frame[label=cib]{
  \frametitle{Commutative Idempotent Binars}
  A \alert{CIB} is an algebra $\bA = \<A, \cdot\>$ satisfying
  $x\cdot y \approx y\cdot x$ and $x\cdot x \approx x$.

  \onslide<2->{
    \begin{block}{Question}
      Is every finite commutative idempotent binar tractable?
    \end{block}
    %  If the dichotomy conjecture is to hold, then the answer must be ``yes.''
    % We let **\cib** denote the variety of **commutative idempotent binars.**
    \vskip2mm
    \onslide<3->{
      First Example: a semilattice is an associative CIB.\\[4pt]
      Semilattices are tractable. % (in fact, they have \emph{finite width}).  
    }
    \vskip1cm
    \onslide<4->{
      Pause to consider more general case for a minute...
    }
  }
}



%%%%%%%%%%%%%%%%%%%%%%%%%%%%%%%%%%%%%%%%%%%%%%%%%%%%%%%%%%%
%% 9: More well known facts

\frame[label=general1]{
  \frametitle{General case}
  \begin{block}{Some well known facts}
  Let $\bA$ be a finite idempotent algebra. Let $\mathbf S_2$ be the 2-elt semilattice.
  %% \begin{block}{}
  \begin{align*}
    \V(\bA) \text{ is CP } &\Longleftrightarrow \; \bA \text{ has Malcev term}\\
    \onslide<3->{        &\Longrightarrow \; \bA \text{ has cube term}\\
      \onslide<4->{        &\Longrightarrow \; \V(\bA) \text{ is CM}\\
        \onslide<5->{        &\Longrightarrow \; \bS_2 \text{ is not in }
          \V(\bA)
        }
      }
    }
  \end{align*}
  \end{block}

  \begin{overprint}
    \onslide<2|handout:1>
    \begin{center}\includegraphics[height=2in]{figures/CP-cropped.png}\end{center}
    \onslide<3|handout:0>
    \begin{center}\includegraphics[height=2in]{figures/Cube-cropped.png}\end{center}
    \onslide<4|handout:0>
    \begin{center}\includegraphics[height=2in]{figures/CM-cropped.png}\end{center}
    \onslide<5|handout:0>
    \begin{center}\includegraphics[height=2in]{figures/NoSL-cropped.png}\end{center}
  \end{overprint}
}




%%%%%%%%%%%%%%%%%%%%%%%%%%%%%%%%%%%%%%%%%%%%%%%%%%%%%%%%%%%
%% 10
\frame[label=cube]{
  \frametitle{First Reduction}
  \framesubtitle{by Cube-Term Blockers}

  Markovi{\'c}, M.~Mar{\'o}ti, McKenzie ($M^4$)\\
  ``Finitely related clones and algebras with cube terms'' (2012)

  \begin{block}{}
    %% \begin{definition}
    A \alert{cube-term blocker} (CTB) is a pair $(C, B)$ of subuniverses
    %% of $\bA$ 
    satisfying $\emptyset < C < B \leq A$ and for every
    $t(x_1, \dots, x_n)$ %% of $\bA$ 
    there is an index $i \in [n]$ with
    \[
    (\forall (b_1, \dots, b_n) \in B^n) (b_i \in C \longrightarrow t(b_1, \dots, b_n)\in C).
    \]
  \end{block}
  %% \end{definition}
  %% We call a set $D$ a (proper) \defn{ideal} of a \cib $\bA = \< A, \cdot\>$
  %%   if $D$ is a (proper) subset of $A$ satisfying $d\cdot a \in D$ for all 
  %% $d\in D$ and $a \in A$.
\onslide<2->$M^4$ prove a finite idempotent algebra has
  a cube term iff it has no CTB.  

  \vfill
  \onslide<3->{
    \begin{lemma}
      A finite CIB $\bA$ has a CTB if and only if 
      $\bS_2 \in \sansH \sansS (\bA)$.
    \end{lemma}
    \onslide<4->{
      \begin{proof}
        $(C, B)$ a CTB implies
        $\theta = C^2 \cup (B- C)^2$ a congruence with $\bB/\theta \cong \bS_2$. 
        \\[5pt]
        Conversely, suppose $\bS_2 \in \sansH \sansS (\bA)$, and $\bB$ is 
        a subalgebra of $\bA$ with $\bB/\theta$ a meet-SL for some $\theta$. 
        Let $C/\theta$ be the bottom of $\bB/\theta$, then $(C, B)$ is a CTB.
      \end{proof}
    }
  }
}



%%%%%%%%%%%%%%%%%%%%%%%%%%%%%%%%%%%%%%%%%%%%%%%%%%%%%%%%%%%
%% 11
\frame[label=cp]{
  \frametitle{Second Reduction}

  Kearnes and Tschantz\\
  ``Automorphism groups of squares and of free algebras'' (2007)

  \begin{lemma}
    If $V$ is an idempotent variety that is not congruence permutable, then there
    are subuniverses $U$ and $W$ of $\bF := \bF_V\{x, y\}$ %% (the 2-generated free
    %% algebra)
    satisfying 
    \begin{enumerate}[1.]
    \item $x\in U \cap W$
    \item $y \in U^c \cap W^c$
    \item $(U \times F) \cup (F \times W) \leq \bF^2$
    \end{enumerate}
  \end{lemma}
  \onslide<2->{
    For CIB's, either $U$ or $W$ will be an ideal.\\[4pt]
    This implies a CTB and a semilattice.}
}


%%%%%%%%%%%%%%%%%%%%%%%%%%%%%%%%%%%%%%%%%%%%%%%%%%%%%%%%%%%
%% 12: Some well known facts
\frame[label=general2]{
  \frametitle{}

  $\bA=$ a finite CIB \hskip1cm
  $\mathbf S_2=$ the 2-elt semilattice.

  \begin{align*}
    \V(\bA) \text{ is CP } &\Longleftrightarrow \quad \bA \text{ has a Malcev term}\\
    &\Longrightarrow \quad \bA \text{ has a cube term}\\
    &\Longrightarrow \quad \V(\bA) \text{ is CM}\\
    &\Longrightarrow \quad \bS_2 \text{ is not in } \V(\bA)
    \uncover<2->{ \\   &{\mathbf \Longrightarrow} \quad \bA \text{ has a cube term} }
    \uncover<3->{ \\   &{\mathbf \Longrightarrow} \quad \V(\bA) \text{ is CP } }
  \end{align*}

  \begin{columns}
    \begin{column}{0.4\textwidth}
      \begin{overprint}
        \onslide<1|handout:0>
        \begin{center}\includegraphics[height=1.25in]{figures/NoSL-cropped.png}\end{center}
        \onslide<2|handout:0>
        \begin{center}\includegraphics[height=1.25in]{figures/CubeEquiv-cropped.png}\end{center}
        \onslide<3|handout:1>
        \begin{center}\includegraphics[height=1.25in]{figures/CPequiv-cropped.png}\end{center}
      \end{overprint}
      \vskip1cm
    \end{column}
    \begin{column}{0.6\textwidth}
        \begin{itemize}
        \item<2-> 1st reduction by cube-term blockers.
        \end{itemize}
        \begin{itemize}
        \item<3-> 2nd reduction by Kearnes-Tschantz.
        \end{itemize}
    \end{column}
  \end{columns}
}


%%%%%%%%%%%%%%%%%%%%%%%%%%%%%%%%%%%%%%%%%%%%%%%%%%%%%%%%%%%
%% 13
\frame[label=remaining]{
  \frametitle{Remaining Questions for Finite CIBs}

  \begin{block}{Conclusion}
    Let $\bA$ be a finite CIB. Then 
\[
\bS_2 \notin \sansH \sansS (\bA) \text{ if and
    only if } \V(\bA) \text{ is congruence permutable.}
\]
\onslide<2->(so $\CSP(\bA)$ tractable in this case)
  \end{block}

  \onslide<3->
  \begin{block}{Open Question}
    Let $\bA$ be a finite CIB with $\bS_2$ in $\sansH \sansS (\bA)$.  Is $\CSP(\bA)$ tractable?
  \end{block}

\onslide<4->  Recall, 
    if $\V(\bA)$ is  $\mathrm{SD}_\wedge$, then $\CSP(\bA)$ is
    tractable.

%% %% for every $\bA$, 
%%   %% \begin{itemize}
%%   %% \item<3->if $\bS_2 \in \V(\bA)$, then $\V(\bA)$ is not CM;
%%   %% \item<4->
%%     if $\V(\bA)$ is  $\mathrm{SD}_\wedge$, then $\CSP(\bA)$ is
%%     tractable.
%%   %% \end{itemize}

  \onslide<5->
  \begin{block}{Revised Question}
    Let $\bA$ be a finite CIB with $\bS_2$ in $\sansH \sansS (\bA)$,
    and $\V(\bA)$ not $\mathrm{SD}_\wedge$.\\[4pt] 
    Is $\CSP(\bA)$ tractable?
  \end{block}

}

%%%%%%%%%%%%%%%%%%%%%%%%%%%%%%%%%%%%%%%%%%%%%%%%%%%%%%%%%%%
%% 14
\frame[label=examples1]{
  \frametitle{Example 1}

  \begin{columns}
    \begin{column}{0.4\textwidth}
  \begin{tabular}{c|cccc}
    $\cdot$ & 0 & 1 & 2 & 3\\
    \hline
    0 & 0 & 0 & 0 & 1\\
    1 & 0 & 1 & 3 & 2\\
    2 & 0 & 3 & 2 & 1\\
    3 & 1 & 2 & 1 & 3\\
  \end{tabular}
    \end{column}
    \begin{column}{0.6\textwidth}
      \onslide<2->
      \emph{Cliff's trick:} replace binary operation with a term from
      $\sansClo(\bA)$, say 
      \[x \ast y = (x\cdot(x\cdot y)) \cdot (y\cdot(x\cdot y))
      \]
      \\[6pt] 
      If $\<A, \ast\>$ tractable, then so is $\bA = \<A, \cdot\>$.
      \\[6pt] 
      \onslide<3->
      \begin{align*}
      \{\ast\} \subseteq \sansClo(\bA) \quad &\Longrightarrow \quad \sansRel(\sansClo(\bA)) 
      \subseteq \sansRel(\{\ast\})\\
      &\Longrightarrow \quad \CSP(\bA) \leq_P \CSP\<A, \ast\>
      \end{align*}
    \end{column}
  \end{columns}
  \onslide<4->
  \begin{columns}
    \begin{column}{0.4\textwidth}
      \begin{tabular}{c|cccc}
        $\ast$ & 0 & 1 & 2 & 3\\
        \hline
        0 & 0 & 0 & 0 & 0\\
        1 & 0 & 1 & 3 & 2\\
        2 & 0 & 3 & 2 & 1\\
        3 & 0 & 2 & 1 & 3\\
        \end{tabular}
    \end{column}
    \begin{column}{0.6\textwidth}
      $\<A, \ast\> \text{  tractable } \quad  \Longrightarrow \quad \bA 
\text{  tractable }$ 
    \end{column}
  \end{columns}
}



%%%%%%%%%%%%%%%%%%%%%%%%%%%%%%%%%%%%%%%%%%%%%%%%%%%%%%%%%%%
%% 15
\frame[label=examples2]{
  \frametitle{Example 2}

  \begin{columns}
    \begin{column}{0.4\textwidth}
  \begin{tabular}{c|cccc}
    $\cdot$ & 0 & 1 & 2 & 3\\
    \hline
    0 & 0 & 0 & 1 & 1\\
    1 & 0 & 1 & 3 & 2\\
    2 & 1 & 3 & 2 & 1\\
    3 & 1 & 2 & 1 & 3\\
  \end{tabular}
    \end{column}
    \begin{column}{0.6\textwidth}
      Let $t(x,y) = x\cdot(x\cdot(x\cdot y)) \cdot y\cdot (y\cdot(x\cdot y))$.
    \end{column}
  \end{columns}

\vskip3mm

  \onslide<3->
  \begin{columns}
    \begin{column}{0.4\textwidth}
      \begin{tabular}{c|cccc}
        $t$ & 0 & 1 & 2 & 3\\
        \hline
        0 & 0 & 0 & 0 & 1\\
        1 & 0 & 1 & 3 & 2\\
        2 & 0 & 3 & 2 & 1\\
        3 & 1 & 2 & 1 & 3\\
        \end{tabular}
    \end{column}
    \begin{column}{0.6\textwidth}
      $\<A, t\> \text{  tractable }$ 
    \end{column}
  \end{columns}
}

%%%%%%%%%%%%%%%%%%%%%%%%%%%%%%%%%%%%%%%%%%%%%%%%%%%%%%%%%%%
%% 16
\frame[label=examples3]{
  \frametitle{Example 3}

  \begin{columns}
    \begin{column}{0.4\textwidth}
  \begin{tabular}{c|cccc}
    $\cdot$ & 0 & 1 & 2 & 3\\
    \hline
    0 & 0 & 0 & 2 & 1\\
    1 & 0 & 1 & 3 & 2\\
    2 & 2 & 3 & 2 & 1\\
    3 & 1 & 2 & 1 & 3\\
  \end{tabular}
    \end{column}
    \begin{column}{0.6\textwidth}
\onslide<2->      Let $t_2(x,y) = \dots$ ?
\vskip2mm
\onslide<3->      Let $t_3(x,y,z) = \dots$ ?
    \end{column}
  \end{columns}
}
%%%%%%%%%%%%%%%%%%%%%%%%%%%%%%%%%%%%%%%%%%%%%%%%%%%%%%%%%%%
%% 17
\frame[label=others]{
  \frametitle{}
...and about 25 others.

\begin{center}\includegraphics[height=2in]{figures/UACalcBergman.png}
%% \includegraphics[height=1.5in]{figures/BergmanDirectory.png}
\end{center}

To see them, load UACalc with files 
from the \alert{Bergman} directory at
\begin{center}
{\bf \url{https://github.com/UACalc/AlgebraFiles} }
\\[6pt]
\onslide<2->Thank you for listening!
\end{center}

}

\end{document}

%%%%%%%%%%%%%%%%%%%%%%%%%%%%%%%%%%%%%%%%%%%%%%%%%%%%%%%%%%%
%% 18
\frame[label=goodbye]{
  \frametitle{}
\begin{center}
Thank you for listening!
\end{center}
}





Question: Does the converse of the last implication hold in general?
That is, if A is an finite idempotent algebra, then is it true that

S not in V(A)  =>  V(A) is CM?

This is certainly not true. For example, take A to be a 2-element set. However,
even if you omit type 1 it fails. Example 2.2 in the Freese-McKenzie paper on
Robust Maltsev conditions is an example. (It is actually an example due to
Matt.) You can probably find more examples by hunting through the Berman-Burris
catalog of 3-element binars:
http://www.math.uic.edu/~berman/1994-Groupoid-Catalog-Preprint.pdf 

It seems to me this is what Prop 3.9 and Cor 3.10 of Freese-Valeriote
says.  (If not, do you know a counter-example?)
No, the stuff about the tails is subtle. That's what Matt's example was designed to show.

I should know counter-examples for each of the converses that don't
hold.  Right now, I only know of one for "CM => cube term".  Namely,
the algebra Kearnes4.ua available here:

https://github.com/UACalc/AlgebraFiles/tree/master/Kearnes

has no cube term, but V(A) is CM.
This is the only example I am aware of.

What are examples of algebras with a cube term but no Malcev term.  (I
should know this! ...Jiali, feel free to jump in here!)
Every near-unanimity term is a cube term. So Lattices, for example, have cube terms but not Malcev terms. Also, the gmm terms of Dalmau are cube terms. I presume they are not always Malcev terms.

Now, back to the CIB case.  Once we prove (for CIBs) that

S not in V(A)  =>   V(A) is CP

then all the conditions above are equivalent.  That is,

For a finite CIB A, TFAE

1. V(A) is CP
2. A has a Malcev term
3. A has a cube term
4. V(A) is CM
5. S is not in V(A)
Correct. By the way, do we need A finite for this? I suspect not







