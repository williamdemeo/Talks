%\documentclass[xcolor=dvipsnames,9pt]{beamer} 
\documentclass[xcolor=dvipsnames,9pt,hide notes,mathserif]{beamer}

\usepackage{pgfpages}
\usepackage{listings}
%\usepackage{enumitem}

%% For creating a handout:
%\pgfpagesuselayout{4 on 1}[border shrink=5mm]
%\mode<handout>{\setbeamercolor{background canvas}{bg=black!5}}

%% For creating notes for the speaker:
%\setbeameroption{notes on second screen}
%\setbeameroption{show notes}

\setbeamerfont{structure}{family=\rmfamily,shape=\scshape} 
\usepackage{graphicx}
\usepackage{tikz}
\usepackage{scalefnt}
\usepackage{amsmath}%
\usepackage{amsfonts}%
\usepackage{amssymb}%
%(wjd) added stmaryrd and enumerate packages
\usepackage{stmaryrd,enumerate}
\usepackage{graphicx}
\usepackage{comment}
\usetikzlibrary{matrix,arrows}

\usepackage{mathrsfs,textcomp}
\setbeamertemplate{navigation symbols}{}
\usepackage{verbatim}
\usepackage[mathcal]{euscript}

% This changes the color of alerted text to blue:
\definecolor{MyDarkBlue}{rgb}{0.2,0.2,0.7}
\definecolor{olivegreen}{cmyk}{0.64,0,0.95,0.40} % PANTONE 582
\setbeamercolor{alerted text}{fg=blue}
\newcommand{\emphcyan}[1]{\textcolor{MyDarkBlue}{\textbf{#1}}}
%\renewcommand{\alert}[1]{\textcolor{olivegreen}{\emph{#1}}}
\renewcommand{\alert}[1]{\textcolor{olivegreen}{#1}}
%\renewcommand{\alert}[1]{\textbf{{\emph{#1}}}}
% (default is red, but my slides are green and I don't like red and green together)

%\usecolortheme[named=OliveGreen]{structure} 
\usecolortheme[named=olivegreen]{structure} 
\setbeamertemplate{items}[square] 
\setbeamertemplate{blocks}[rounded][shadow=false]


% Commands for creating the ROTATING RECTANGLE
% Pass in a number which will be used to calculate the rotation angle.
% Example: Inside a tikzpicture environment, I would call 
%          \foreach \i in {0,...,11} { \eImageOfBZero{\i}  }
\newcommand{\eImageOfBZero}[1]{
  \pgfmathtruncatemacro{\r}{15*#1}
  \foreach \j in {1,2} {
    \draw[rotate around={\r:(-1,0.5)}] (\j -1, 0.5) node {$\j$};
    \pgfmathtruncatemacro{\x}{\j+3}
    \draw[rotate around={\r:(-1,0.5)}] (\j -1, -0.5) node {$\x$};
  }
  \draw[rotate around={\r:(-1,0.5)}] (-1, -0.5) node {$3$};
  \draw[rounded corners, dotted, rotate around={\r:(-1,0.5)}] (-1.5,-1) rectangle (1.5,1);
}

\newcommand{\eImageOfBOne}[1]{
  \pgfmathtruncatemacro{\r}{-15*#1}
  \foreach \j in {0,1,2} {
    \pgfmathtruncatemacro{\x}{10-\j}
    \draw[rotate around={\r:(-1,0.5)}] (\j -3, 1.5) node {$\x$};
  }
  \draw[rotate around={\r:(-1,0.5)}] (-3, .5) node {$7$} (-2, .5) node {$6$};
  \draw[rounded corners, dotted, rotate around={\r:(-1,0.5)}] (-3.5,0) rectangle (-0.5,2);
}

\newcommand{\eImageOfBTwo}[1]{
  \pgfmathtruncatemacro{\r}{15*#1}
  \foreach \j in {0,1,2} {
    \pgfmathtruncatemacro{\x}{15-\j}
    \draw[rotate around={\r:(1,0.5)}] (\j+1,1.5) node {$\x$};
  }
  \draw[rotate around={\r:(1,0.5)}] (3, .5) node {$11$} (2, .5) node {$12$};
  \draw[rounded corners, dotted,rotate around={\r:(1,0.5)}] (3.5,0) rectangle (0.5,2);
}


\newcommand{\Hawaii}{Hawai\kern.05em`\kern.05em\relax i}
\newcommand{\Manoa}{M\=anoa}
\newcommand{\dotsize}{.8pt}
\newcommand{\FLRP}{{\small FLRP}}
\newcommand{\ssubnormal}{\ensuremath{\vartriangleleft}}
\newcommand{\core}{\ensuremath{\mathrm{core}}}
\newcommand{\bs}{\ensuremath{\backslash}}
\newcommand{\GAP}{{\small GAP}}
\newcommand{\cd}{\ensuremath{\otimes}}
\newcommand{\con}[1]{\ensuremath{\langle #1 \rangle}}
\newcommand{\ii}[1]{{\it #1}}
\newcommand{\power}[1]{\ensuremath{\mathscr{P}(#1)}}
\newcommand{\scrA}{\ensuremath{\mathscr{A}}}
\newcommand{\bM}{\ensuremath{\mathbf{M}}}
\newcommand{\Mn}{\ensuremath{\mathbf{M}_n}}
\newcommand{\bF}{\ensuremath{\mathbf{F}}}
\newcommand{\bE}{\ensuremath{\mathbf{E}}}
\newcommand{\bR}{\ensuremath{\mathbf{R}}}
\newcommand{\bA}{\ensuremath{\mathbf{A}}}
\newcommand{\bG}{\ensuremath{\mathbf{G}}}
\newcommand{\bH}{\ensuremath{\mathbf{H}}}
\newcommand{\bK}{\ensuremath{\mathbf{K}}}
\newcommand{\bL}{\ensuremath{\mathbf{L}}}
\newcommand{\bB}{\ensuremath{\mathbf{B}}}
\newcommand{\svert}{\ensuremath{\; \vert \; }}
\newcommand{\Z}{\ensuremath{\mathbb{Z}}}
\newcommand{\sE}{\ensuremath{\mathcal{E}}}
\newcommand{\sO}{\ensuremath{\mathcal{O}}}
\newcommand{\sH}{\ensuremath{\mathcal{H}}}
\newcommand{\sS}{\ensuremath{\mathcal{S}}}
\newcommand{\sL}{\ensuremath{\mathcal{L}}}
\newcommand{\bN}{\ensuremath{\mathbf{N}}}
\newcommand{\bX}{\ensuremath{\mathbf{X}}}
\newcommand{\resB}{\ensuremath{|_{_B}}}
\newcommand{\eps}{\ensuremath{\varepsilon}}
\newcommand{\sA}{\ensuremath{\mathcal{A}}}
\newcommand{\sB}{\ensuremath{\mathcal{B}}}
\newcommand{\sC}{\ensuremath{\mathcal{C}}}
\newcommand{\SSS}{\text{\emphslb{S}}}
\newcommand{\id}{\mbox{id}}
\newcommand{\Hom}{\mbox{Hom}}
\newcommand{\End}{\ensuremath{\mathrm{End}}}
\newcommand{\bEnd}{\ensuremath{\mathbf{End}}}
\newcommand{\Aut}{\ensuremath{\mathrm{Aut}}}
\newcommand{\bAut}{\ensuremath{\mathbf{Aut}}}
\newcommand{\Cg}{\ensuremath{\mathrm{Cg}}}
\newcommand{\Con}{\ensuremath{\mathrm{Con}\,}}
\newcommand{\bCon}{\ensuremath{\mathbf{Con}\,}}
\newcommand{\Sub}{\mbox{Sub}}
\newcommand{\bSub}{\ensuremath{\mathbf{Sub}}}
\newcommand{\CSub}[1]{\ensuremath{\mathbf{CSub}[#1]}}
\newcommand{\csub}{\ensuremath{\mbox{CSub}}}
\newcommand{\Stab}{\mbox{Stab}}
\newcommand{\bStab}{\ensuremath{\mathbf{Stab}}}
\newcommand{\X}{\ensuremath{\mathbf{X}}}
\newcommand{\image}{\mbox{im}}
\newcommand{\Eq}{\mbox{Eq}}
\newcommand{\bEq}{\ensuremath{\mathbf{Eq}}}
\newcommand{\bEqX}{\ensuremath{\mathbf{Eq}(X)}}
\newcommand{\idemdec}{\ensuremath{\mbox{Idemdec}(X)}}
\newcommand{\EqX}{\ensuremath{\mbox{Eq}(X)}}
\newcommand{\upalpha}{\ensuremath{\alpha^{\uparrow}}}
\newcommand{\downalpha}{\ensuremath{\alpha^{\downarrow}}}
\newcommand{\upbeta}{\ensuremath{\beta^{\uparrow}}}
\newcommand{\downbeta}{\ensuremath{\beta^{\downarrow}}}
\newcommand{\meet}{\ensuremath{\wedge}}
\newcommand{\join}{\ensuremath{\vee}}
\newcommand{\Meet}{\ensuremath{\bigwedge}}
\renewcommand{\Join}{\ensuremath{\bigvee}}
\renewcommand{\leq}{\ensuremath{\leqslant}}
\renewcommand{\nleq}{\ensuremath{\nleqslant}}
\renewcommand{\geq}{\ensuremath{\geqslant}}
\newcommand{\pb}{\ensuremath{\protect{|}}}

\newcommand{\code}[1]{{\small {\tt #1}}}
\newcommand{\<}{\langle}	     %% left angle for order sequences <a,b>
\renewcommand{\>}{\rangle}	     %% right angle
\newcommand{\lb}{\ensuremath{\llbracket}}
\newcommand{\rb}{\ensuremath{\rrbracket}}

\newcommand{\Palfy}{P\'alfy}
\newcommand{\Pudlak}{Pudl\'ak}
\newcommand{\PAP}{P\'alfy-Pudl\'ak}
\newcommand{\Tuma}{T\r{u}ma}
\newcommand{\res}{\ensuremath{\upharpoonright}}  % restriction


\mode<presentation>{\usetheme{boxes}}  %boxes,Pittsburgh JuanLesPins, PaloAlto, Singapore, Szeged, Warsaw, Boadilla
%\usetheme{Madrid}}
%\usetheme{boxes}  %boxes,Pittsburgh JuanLesPins, PaloAlto, Singapore, Szeged, Warsaw, Boadilla

\usepackage[english]{babel}
\usepackage[latin1]{inputenc}
\usepackage{times}
\usepackage[T1]{fontenc}
% Or whatever. Note that the encoding and the font should match. If T1
% does not look nice, try deleting the line with the fontenc.

\title{CSP Theory of Commutative Idempotent Binars}

\author[William DeMeo]{William DeMeo\\
  {\small \url{williamdemeo@gmail.com}}\\
  {\small Iowa State University}\\[4pt]
  {\footnotesize joint work with}\\[4pt] 
  Cliff Bergman\\
  Jiali Li
}
%\institute[]{

%% \date[BLAST 2013]{ % (optional, should be abbreviation of conference name)
%%   BLAST Conference\\{\small University of North Texas}\\
%%   {\small June 7--12, 2015}}

\subject{Universal Algebra; Lattice Theory.}% (optional) inserted into PDF info catalog.

% TOC pops up at the beginning of each subsection:
\AtBeginSubsection[]{
  \begin{frame}<beamer>
    \frametitle{Outline}
    \tableofcontents[currentsection,currentsubsection]
  \end{frame}
}

% If you wish to uncover everything in a step-wise fashion, uncomment the following command: 
% \beamerdefaultoverlayspecification{<+->}

\begin{document}
\thicklines

%% \includeonlyframes{titlepage,problem,milestones,methods,knownresults,filterideal,MO,freese,OA,OAcong,OAEx2,PAP1,OAresults,OAextension,Limitations,OAextension2,conclusion,MO,Conclusion}


\frame[label=titlepage]{
  \titlepage
}


%%%%%%%%%%%%%%%%%%%%%%%%%%%%%%%%%%%%%%%%%%%%%%%%%%%%%%%%%%%
%% 1: CSP dichotomy conjecture
\frame[label=problem]{
  \frametitle{~}
%  \framesubtitle{CSP dichotomy conjecture}

      \alert{General Problem:} Find Maltsev conditions that characterize complexity of
      CSPs of universal algebras.
      \vskip3mm

      \begin{overprint}
        \onslide<1->
          \begin{block}{csp dichotomy conjecture}
            For a (finite, idempotent) algebra $\mathbf A$...
            \[
            \CSP (\mathbf A) \text{ is tractable } \; \Longleftrightarrow \;  \bA
            \text{ has a wnu term operation}
            \]
          \end{block}
        \onslide<3->
          \begin{block}{csp dichotomy conjecture}
            For a (finite, idempotent) algebra $\mathbf A$...
            \[
            \CSP (\mathbf A) \text{ is tractable } \; \Longleftarrow \;  \bA
            \text{ has a wnu term operation}
            \]
          \end{block}
      \end{overprint}

      \vskip3mm

      \onslide<2->{ The left-to-right direction is known, the converse is open.} \\[10pt]

      \begin{overprint}
        \onslide<4->
          A term $t(x_1, \dots, x_n)$ is a \alert{weak near unanimity}
          term operation if it satisfies 
          \begin{align*}
          t(x, x, \dots, x)&\approx x \quad \text{ (idempotent)}\\[4pt]
          t(y, x, \dots, x) &\approx t(x, y, \dots, x) \approx \dots \approx
          t(x, x, \dots, y).
          \end{align*}

        \onslide<5->\begin{center}\includegraphics[height=1.5in]{figures/wnu-only-cropped.png}\end{center}
        \onslide<6->\begin{center}\includegraphics[height=1.5in]{figures/Taylor-cropped.png}\end{center}
        \onslide<7->\begin{center}\includegraphics[height=1.5in]{figures/Cyclic-cropped.png}\end{center}
        \onslide<8->\begin{center}\includegraphics[height=1.5in]{figures/Siggers-cropped.png}\end{center}
        \onslide<9->\begin{center}\includegraphics[height=1.5in]{figures/NP-cropped.png}\end{center}
        \onslide<10->\begin{center}\includegraphics[height=1.5in]{figures/P-cropped.png}\end{center}
      \end{overprint}
}

%%%%%%%%%%%%%%%%%%%%%%%%%%%%%%%%%%%%%%%%%%%%%%%%%%%%%%%%%%%
%% 2: Commutative Idempotent Binars
\frame[label=problem]{
  \frametitle{Commutative Idempotent Binars}
Some more definitions.
  \begin{itemize}
  \item A set $A$ together with a single binary operation is called a \alert{binar}.

  \item A \alert{commutative idempotent binar} is an algebra 
    $\bA = \<A, \cdot\>$ satisfying $x\cdot y \approx y\cdot x$ and $x\cdot x \approx x$.

  \item A binary operation $x\cdot y = t(x,y)$ is a WNU term if and only if it is idempotent and 
    commutative. This suggests the following 
  \end{itemize}
\onslide<2->{
  \begin{block}{Question}
    Is every finite commutative idempotent binar tractable?
  \end{block}
  If the dichotomy conjecture is to hold, then the answer must be ``yes.''
  % We let **\cib** denote the variety of **commutative idempotent binars.**
\\[4pt]
\onslide<3->{
  A semilattice is an associative CIB.\\[4pt]
  Semilattices are tractable (in fact, they have \emph{finite width}).  
}}

}

%%%%%%%%%%%%%%%%%%%%%%%%%%%%%%%%%%%%%%%%%%%%%%%%%%%%%%%%%%%
%% 3: More well known facts

\frame[label=knownfigs]{
  \frametitle{Some well known facts}
  Let $\bA$ be a finite idempotent algebra. Let $\mathbf S_2$ be the 2-elt semilattice.
  %% \begin{block}{}
  \begin{align*}
    \V(\bA) \text{ is CP } &\Longleftrightarrow \quad \bA \text{ has Malcev term}\\
    \onslide<3->{        &\Longrightarrow \quad \bA \text{ has cube term}\\
      \onslide<4->{        &\Longrightarrow \quad \V(\bA) \text{ is CM}\\
        \onslide<5->{        &\Longrightarrow \quad \bS_2 \text{ is not in }
          \V(\bA)
        }
      }
    }
  \end{align*}
  %% \end{block}

  \begin{overprint}
    \onslide<2->
    \begin{center}\includegraphics[height=2in]{figures/CP-cropped.png}\end{center}
    \onslide<3->
    \begin{center}\includegraphics[height=2in]{figures/Cube-cropped.png}\end{center}
    \onslide<4->
    \begin{center}\includegraphics[height=2in]{figures/CM-cropped.png}\end{center}
    \onslide<5->
    \begin{center}\includegraphics[height=2in]{figures/NoSL-cropped.png}\end{center}
  \end{overprint}
}

%%%%%%%%%%%%%%%%%%%%%%%%%%%%%%%%%%%%%%%%%%%%%%%%%%%%%%%%%%%
%% 4: Some well known facts
\frame[label=known]{
  \frametitle{Some well known facts}

  $\bA=$ a finite idempotent algebra\\[4pt]
  $\mathbf S_2=$ the 2-elt semilattice.

  \begin{align*}
    \V(\bA) \text{ is CP } &\Longleftrightarrow \quad \bA \text{ has a Malcev term}\\
    &\Longrightarrow \quad \bA \text{ has a cube term}\\
    &\Longrightarrow \quad \V(\bA) \text{ is CM}\\
    &\Longrightarrow \quad \bS_2 \text{ is not in } \V(\bA)
  \end{align*}

  \begin{columns}
    \begin{column}{0.4\textwidth}
      \begin{center}\includegraphics[height=1.25in]{figures/NoSL-cropped.png}\end{center}
    \end{column}
    \begin{column}{0.6\textwidth}
      \begin{itemize}
      \item<2-> cube term $\Longrightarrow$ CM\\[4pt]
        \emph{Proof:} few subalgebras of powers\\[5pt]
        Berman, Idziak, Markovi{\'c}, McKenzie, Valeriote, Willard (BIMMVW) 2010.
        %% ``Varieties with few subalgebras of powers'' 
        \\[10pt]
      \item<3-> CM $\Longrightarrow \; \bS_2$ is not in $\V(\bA)$\\[4pt]
        \emph{Proof:} $\bS_2 \in \V(\bA) \; \Rightarrow\; \bS_2^2 \in \V(\bA)$;\\[4pt]
        ~ \phantom{\emph{Proof:}} $\Con(\bS_2^2)$ is not modular.
      \end{itemize}
    \end{column}
  \end{columns}
}


%%%%%%%%%%%%%%%%%%%%%%%%%%%%%%%%%%%%%%%%%%%%%%%%%%%%%%%%%%%
%% 5
\frame[label=known]{
  \frametitle{Commutative Idempotent Binars (CIBs)}
  Let $\bA$ be a CIB.
  \begin{align*}
    \V(\bA) \text{ is CP } &\Longleftrightarrow \quad \bA \text{ has Malcev term}\\
    &\Longrightarrow \quad \bA \text{ has cube term}\\
    &\Longrightarrow \quad \V(\bA) \text{ is CM}\\
    &\Longrightarrow \quad \bS_2 \text{ is not in } \V(\bA)
  \end{align*}

  \begin{overprint}
    \onslide<1->
    \begin{center}\includegraphics[height=2in]{figures/NoSL-cropped.png}\end{center}
    \onslide<2->
    \begin{center}\includegraphics[height=2in]{figures/CubeEquiv-cropped.png}\end{center}
    \onslide<3->
    \begin{center}\includegraphics[height=2in]{figures/CPequiv-cropped.png}\end{center}
  \end{overprint}
}




%%%%%%%%%%%%%%%%%%%%%%%%%%%%%%%%%%%%%%%%%%%%%%%%%%%%%%%%%%%
%% 6
\frame[label=cube]{
  \frametitle{Cube Terms}
  \begin{block}{}
    A \alert{cube operation} is
    a function $c: A^n \rightarrow A$ satisfying
    for each $1 \leq i \leq n$ 
    $c(w_1, \dots, w_n) = x$ where 
    $\{w_1, \dots, w_n\} \subseteq \{x, y\}$ and 
    $w_i = y$.
    \\[4pt]
    Here $x$ and $y$ are distinct variables.
    \\[6pt]
    An algebra $\bA$ is said to have a \alert{cube term} if its clone
    of term operations contains a cube operation. 
  \end{block}
  \vfill
  \onslide<2->{
    Cube terms were introduced in... ?\\[8pt]
    Berman, Idziak, Markovi{\'c}, McKenzie, Valeriote, Willard,\\
    ``Varieties with few subalgebras of powers,'' 2010. \\[10pt]
    Markovi{\'c}, Mar{\'o}ti, McKenzie,\\
    ``Finitely related clones \& algebras with cube terms,'' 2012.}
}


%%%%%%%%%%%%%%%%%%%%%%%%%%%%%%%%%%%%%%%%%%%%%%%%%%%%%%%%%%%
%% 4
\frame[label=cube]{
  \frametitle{Cube Term Blockers}
  \begin{block}{}
    %% \begin{definition}
    A \alert{cube term blocker} (CTB) for $\bA$ is a pair $(C, B)$ of subuniverses
    of $\bA$ satisfying $\emptyset < C < B \leq A$ and for every term 
    $t(x_1, \dots, x_n)$ of $\bA$ there is an index $i \in [n]$ such that 
    \[
    (\forall (b_1, \dots, b_n) \in B^n) (b_i \in C \longrightarrow t(b_1, \dots, b_n)\in C).
    \]
  \end{block}
  %% \end{definition}
  %% We call a set $D$ a (proper) \defn{ideal} of a \cib $\bA = \< A, \cdot\>$
  %%   if $D$ is a (proper) subset of $A$ satisfying $d\cdot a \in D$ for all 
  %% $d\in D$ and $a \in A$.
  Markovi{\'c}, Mar{\'o}ti, McKenzie proved that a finite idempotent algebra has
  a cube term iff it possesses no CTB.  

  \vfill
  \onslide<2->{
    \begin{lemma}
      A finite CIB $\bA = \<A, \cdot\>$ has a CTB if and only if 
      $\bS_2 \in \sansH \sansS (\bA)$.
    \end{lemma}
    \onslide<3->{
      \begin{proof}
        If $(C, B)$ is a CTB, then $\theta = C^2 \cup (B- C)^2$ is a congruence of
        $\bB = \<B, \cdot\>$ and $\bB/\theta \cong \bS_2$. 
        \\[5pt]
        Conversely, suppose $\bS_2 \in \sansH \sansS (\bA)$, and $\bB$ is 
        a subalgebra of $\bA$ with $\bB/\theta$ a meet-SL for some $\theta$. 
        Let $C/\theta$ be the bottom of $\bB/\theta$, then $(C, B)$ is a CTB.
      \end{proof}
    }
}}


}



  %%%%%%%%%%%%%%%%%%%%%%%%%%%%%%%%%%%%%%%%%%%%%%%%%%%%%%%%%%%
  %% 4
  \frame[label=cube]{
    \frametitle{Collapse for CIBs}

    Kearnes and Tschantz,
    ``Automorphism groups of squares and of free algebras,'' 2007.

    \begin{lemma}
      If $V$ is an idempotent variety that is not congruence permutable, then there
      are subuniverses $U$ and $W$ of $\bF := \bF_V\{x, y\}$ (the 2-generated free
      algebra) satisfying 
      \begin{enumerate}[1.]
      \item $x\in U \cap W$
      \item $y \in U^c \cap W^c$
      \item $(U \times F) \cup (F \times W) \leq \bF^2$
      \end{enumerate}
    \end{lemma}
    \onslide<2->{
      For CIB's, $U$ or $W$ will be an ideal.\\[4pt]
      This implies a CTB and a semilattice.}
  }

  %%%%%%%%%%%%%%%%%%%%%%%%%%%%%%%%%%%%%%%%%%%%%%%%%%%%%%%%%%%
  %% 4
  \frame[label=examples]{
    \frametitle{Remaining Question for CIBs}

    \begin{block}{Conclusion}
      Let $\bA$ be a CIB and $\bS_2 \notin \V(\bA)$. Then $\CSP(\bA)$ is tractable.
    \end{block}

    \onslide<2->
    \begin{block}{Open Question}
      Let $\bA$ be a CIB and $\bS_2 \in \V(\bA)$.  Is $\CSP(\bA)$ tractable?
    \end{block}

    Recall, for every $\bA$, 
    \begin{itemize}
    \item<3->if $\bS_2 \in \V(\bA)$, then $\V(\bA)$ is not CM;

    \item<4-> if $\V(\bA)$ is  $\mathrm{SD}_\wedge$, then $\CSP(\bA)$ is
    tractable (in fact, always has a solution).
    \end{itemize}

    \onslide<5->
    \begin{block}{Revised Question}
    Let $\bA$ be a CIB with $\bS_2$ in $\V(\bA)$, not $\mathrm{SD}_\wedge$.
    Is $\CSP(\bA)$ tractable?
    \end{block}

  }

  %%%%%%%%%%%%%%%%%%%%%%%%%%%%%%%%%%%%%%%%%%%%%%%%%%%%%%%%%%%
  %% 4
  \frame[label=examples]{
    \frametitle{Examples}

    \begin{tabular}{c|cccc}
$\cdot$ & 0 & 1 & 2 & 3\\
\hline
      0 & 0 & 0 & 0 & 1\\
      1 & 0 & 1 & 3 & 2\\
      2 & 0 & 3 & 2 & 1\\
      3 & 1 & 2 & 1 & 3\\
    \end{tabular}
\hskip1cm
    \begin{tabular}{c|cccc}
$\ast$ & 0 & 1 & 2 & 3\\
\hline
      0 & 0 & 0 & 1 & 1\\
      1 & 0 & 1 & 3 & 2\\
      2 & 1 & 3 & 2 & 1\\
      3 & 1 & 2 & 1 & 3\\
    \end{tabular}
\vskip1cm
  \begin{columns}
    \begin{column}{0.4\textwidth}
    \begin{tabular}{c|cccc}
$\circ$ & 0 & 1 & 2 & 3\\
\hline
      0 & 0 & 0 & 2 & 1\\
      1 & 0 & 1 & 3 & 2\\
      2 & 2 & 3 & 2 & 1\\
      3 & 1 & 2 & 1 & 3\\
    \end{tabular}
    \end{column}
    \begin{column}{0.5\textwidth}
    Maroti's idea:
    \vskip1cm
    Bergman's idea: replace basic binary operation with a term from
    $\Clo(\bA)$, say $t(x,y) = (x\ast y) \ast x$.\\[4pt] If $\<A, t\>$ tractable,
    then so is $\<A, \ast\>$
    \end{column}
  \end{columns}
  }

    \end{document}




    Question: Does the converse of the last implication hold in general?
    That is, if A is an finite idempotent algebra, then is it true that

    S not in V(A)  =>  V(A) is CM?

    This is certainly not true. For example, take A to be a 2-element set. However,
    even if you omit type 1 it fails. Example 2.2 in the Freese-McKenzie paper on
    Robust Maltsev conditions is an example. (It is actually an example due to
    Matt.) You can probably find more examples by hunting through the Berman-Burris
    catalog of 3-element binars:
    http://www.math.uic.edu/~berman/1994-Groupoid-Catalog-Preprint.pdf 

    It seems to me this is what Prop 3.9 and Cor 3.10 of Freese-Valeriote
    says.  (If not, do you know a counter-example?)
    No, the stuff about the tails is subtle. That's what Matt's example was designed to show.

    I should know counter-examples for each of the converses that don't
    hold.  Right now, I only know of one for "CM => cube term".  Namely,
    the algebra Kearnes4.ua available here:

    https://github.com/UACalc/AlgebraFiles/tree/master/Kearnes

    has no cube term, but V(A) is CM.
    This is the only example I am aware of.

    What are examples of algebras with a cube term but no Malcev term.  (I
    should know this! ...Jiali, feel free to jump in here!)
    Every near-unanimity term is a cube term. So Lattices, for example, have cube terms but not Malcev terms. Also, the gmm terms of Dalmau are cube terms. I presume they are not always Malcev terms.

    Now, back to the CIB case.  Once we prove (for CIBs) that

    S not in V(A)  =>   V(A) is CP

    then all the conditions above are equivalent.  That is,

    For a finite CIB A, TFAE

    1. V(A) is CP
    2. A has a Malcev term
    3. A has a cube term
    4. V(A) is CM
    5. S is not in V(A)
    Correct. By the way, do we need A finite for this? I suspect not







