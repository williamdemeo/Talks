% ICMC 2001 paper template for Latex2e.

% Points to note:
%    Please use \paragraph instead of \subsubsection -- see discussion below.
%    See comments in the References section on how to do citations.

\documentclass[10pt,a4paper]{article}
\usepackage{fullpage}
\usepackage{times}
\usepackage{chicago}                    % for "(Author, year)" cite style.
\usepackage{indentfirst}                % indent para after headings.
% \usepackage{url}                      % (handy if you reference a URL.)

% \setlength{\oddsidemargin}{-0.25in}     % Latex has one-inch "driver margin"
% \setlength{\evensidemargin}{-0.25in}  % (shouldn't be necessary)
\setlength{\textwidth}{160mm}             % 
\setlength{\textheight}{247mm} 
\setlength{\columnsep}{0.25in}
\setlength{\parindent}{0.2in}

\raggedbottom                           % better than inter-para spaces, I say.


\begin{document}

\twocolumn

\title{\textbf{The Resource-Instance Model of Music Representation}}
\author{
Roger B. Dannenberg, Dean Rubine, Tom Neuendorffer \\
School of Computer Science, Carnegie Mellon University \\
email:dannenberg@cs.cmu.edu
}
\date{}     % no date
\maketitle

\pagestyle{empty}          % no page numbers.
\thispagestyle{empty}      % yes, that includes not on the first page, either.

% do abstract by hand -- \begin{abstract} wouldn't conform to the guidelines.

\begin{center}
\large{\textbf{Abstract}}
\end{center}
% there's this annoying little indent here I can't properly eliminate, so...
\hspace*{-0.1in}                       % hackily cancel it out.
\noindent
\textit{
Traditional software synthesis systems, such as Music V, utilize an
instance model of computation in which each note instantiates a new
copy of an instrument. An alternative is the resource model,
exemplified by MIDI "mono mode," in which multiple updates can modify
a sound continuously, and where multiple notes share a single
instrument. We have developed a unified, general model for describing
combinations of instances and resources. Our model is a hierarchy in
which resource-instances at one level generate output, which is
combined to form updates to the next level. The model can express
complex system configurations in a natural way.
}

\section{Introduction}

Two opposing formalisms are prevalent in music representations.  In
the resource model, sounds or notes are produced by controlling an
instrument (the resource).  In the instance model, sounds or notes are
considered to be independent and isolated.  Resource and instance
models can be seen in traditional music notation, computer music
scores, score languages, MIDI, synthesis hardware, and synthesis
software.  Although the distinction between resource and instance
models is fundamental, it is not often made (perhaps because the
implications of the distinction are not well understood).

Once the distinction is made, it can be seen that virtually every
music representation system exhibits both formalisms.  In other words,
music representations have aspects of both the resource and instance
models. Furthermore, these seemingly mutually exclusive models can be
combined to create a comprehensive formalism.  Armed with this new
formalism, we can shed new light on existing representation schemes,
exposing hidden assumptions, revealing subtle ambiguities, and
unmasking limitations.

We will begin by explaining the instance and resource models in
greater detail. We then describe our new formalism, which integrates
the two models. The new "resource-instance" formalism is then applied
to MIDI and Music V to illustrate particular characteristics of these
representation systems.  Then, we describe how we are applying the
formalism in a new system for music representation and synthesis.

\section{A Section}

Sections are numbered.  Subsections are formatted as in the following
subsection.

\subsection{A Subsection}

The Latex source, \textsf{icmc2001template.tex}, contains comments
explaining how you should use Latex so as to conform to the paper
format guidelines.

% \paragraph conforms to the guidelines; \subsubheading doesn't.
% I'll grant you that \paragraph is logically wrong, but it's easy.
\paragraph{Third-order heading.}
If you wish to use third-order headings, format them like this.  They
start an unindented paragraph.

Additional paragraphs are indented as usual.

\subsection{References}

% We use chicago.(bst|sty), which supports (Author, year) cites and
% can be made to look _almost_ the way we wanted.  See chicago.sty
% for a list of all the flavors of citation you can do.
% \cite            (Laurel and Hardy 1932)
% \shortcite       (Laurel et al. 1932)
% \citeN           Laurel and Hardy (1932)
% \shortciteN      Laurel et al. (1932)

Bibliographical references appear in parentheses; there is an example
at the end of this sentence~\cite{canon}.  References with up to
three authors include all the authors \cite{fugue}, but references
with more than three authors use ``et al.''
% that is, you use \shortcite
\shortcite{ircam-mw}.\footnote{the Word template specifies an
italicized ``\emph{et al.}'', but our Latex style doesn't support that.}
% (My naive attempt at hacking chicago.bst to do it had the side effect
% of leaving no space between the author and the "et".  We decided the
% italics were less than utterly indispensable.  If you know how to
% get them, though, drop me a note at eli@cs.cmu.edu.)

A reference is not a subject or object. When you want to use the
referenced work as part of a sentence, use the author or authors and
use the year only for the reference, as in the following sentence:
% \citeN is "cite as noun"
\citeN{mathews} includes a manual for Music V\@.  
% or if you want to, you can say
% Mathews~\citeyear{mathews} includes a manual for Music V\@.  

Just for variety, this is a reference to an ICMC
paper~\cite{superscalar}.

\subsection{Figures and Captions}

If possible, place figures in-line with the text.  Please include an
explanatory caption.

\begin{figure}[htbp]
  \begin{center}

    \framebox[7cm]{\rule[-5mm]{0cm}{5cm} }

    \caption{This figure intentionally left blank.} 
    \label{fig:emptybox}
  \end{center}
\end{figure}

\section{Copyright Notices}

You may wish to add a copyright notice to the bottom of the first
column of your paper.  All copyrights remain with the authors.
Authors will be asked to sign a form that gives ICMA, ICMC, and IEEE
rights to sell the ICMC proceedings.  Typically, ICMC papers carry no
explicit copyright notice.

\bibliographystyle{chicago}
\small{
\bibliography{icmc2001template}
}

\end{document}
