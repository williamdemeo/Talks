% -*- mode: LaTeX; tex-main-file: "../notes.tex"; -*-
%\begin{itemize}
%\item Synthesis -- Sethares (p.15), comp book (00.07.17), beige book (00.07.27)
%\item Tonal Fusion -- comp book (00.07.10)
%\item Spectral Dominance -- Hartmann (p.298) and beige book (00.07.11)
%\item Tonal Dissonance -- comp book (00.07.21)
%\item Wavelet Theory -- beige book (00.07.31, 00.08.01, 00.08.02)
%\end{itemize}
%\input{\HOME/intro/quote.tex}
\section{Introduction}
\subsection{DSP for Perception Analysis}
We begin by stating the two main objectives of this work.  Given a musical
signal, $x(t)$, we wish to:
\begin{enumerate}
\item \label{item:spectral} find useful time-frequency representations for
  analyzing the information content of $x(t)$, with the goal of characterizing
  perceptual properties of the signal;
\item \label{item:dissonance} %using the results of~\ref{item:spectral},
  find measures of qualities related to human perception of the signal;
  in particular derive a ``dissonance signature'' of $x(t)$. 
\end{enumerate}

In addressing (\ref{item:spectral}), we use the \emph{matching pursuit}
algorithm \cite{Mallat:1993}
to perform an atomic decomposition of the signal.  We then use this
decomposition as the basis for an energy characterization of the signal, given
by the \emph{Wigner-Ville distribution}.  This approach is not new.  However,
the literature employing this strategy ignores the interference structure of
the Wigner-Ville distribution.  We retain these interference terms as they are
the focus of our approach to the second objective stated above.

The novel contribution of this paper is consideration of how the interference
terms of the Wigner-Ville decomposition can be used as the basis for a
dissonance measure of a musical signal.  For a simple composition of two pure
tones, there is a well known relation between the interference terms and
the sensory notion of ``beating'' -- i.e.~the effect caused by amplitude
modulations resulting from the composition of tones.  Since some measures of
sensory dissonance are motivated by the rate of such beating, this
suggests basing our dissonance characterization on the interference structure
of a musical signal.

\subsection{Sensory Dissonance}
The concept of \emph{sensory dissonance} was originally proposed
by Helmholtz~\citeyear{Helmholtz:1877}, and further developed by Plomp and 
Levelt~\citeyear{Plomp:1965}, and Sethares~\citeyear{Sethares:1997}.
%Though we consider these studies in greater detail in
%sections~\ref{sec:consonance} and~\ref{sec:disscurves}, 
What follows is a %description that motivates our alternative treatment of
description of dissonance that motivates our alternative treatment of
this concept.

In order to assess the intrinsic dissonance of a musical signal over a
small time interval, the aforementioned studies employ a function of
the signal's estimated frequency components over that interval.
This often provides a useful quantitative measure.  However, such a
function makes no attempt to account for other widely accepted notions
of dissonance.   
Perhaps the most obvious short-coming results from the point-wise
nature of this dissonance function.  That is, because it is
well localized in time, there is no way for the dissonance function to account
for \emph{melodic dissonance} of the signal.  The melodic
dissonance of a given segment of music depends on that 
segment's relation to its context.  In our present work, we
consider signal analysis methods that provide for
%estimates of
%instantaneous frequencies and analytic amplitudes of a signal, and use 
%these estimates to measure the dissonance properties of the
%signal.  A primary objective is the development of methods providing
more dynamic dissonance measures.  In particular, we wish to simultaneously
account for local, point-wise dissonance, as well as dissonance resulting from
the melodic contour of the signal.  

%\subsection{Spectral Analysis of Musical Signals}
%There is a vast literature on the spectral analysis of musical signals. The
%broad class of analysis methods that we consider here 
%consists of {\it time-frequency representations}.  The type of signal
%which concerns us is a {\it time series}, $x(t)$.  Usually we assume that
%periodicities play an important role in the structure of $x(t)$.  To study the
%signal from this point of view, we decompose it as a function of both time and
%frequency. Such a decomposition is what we mean by a time-frequency
%representation (TFR). 
%
%The two broad classes of TFR's which we consider are {\it atomic
%decompositions} and {\it energy distributions}.  The first decomposes a
%signal by ``projecting'' it onto the time-frequency space, thereby equating
%it with a weighted sum of basic elements in that space.
%Some more detailed discussion of atomic decompositions appears is
%section~\ref{sec:atomic}. 
%The second class of TFR's are the energy distributions, the primary
%objective of which is to represent a distribution of the signal's energy
%across the time-frequency plane.  We consider the energy distributions in
%section~\ref{sec:energy}. 
%
%Though atomic decompositions, such as wavelet and windowed Fourier techniques,
%can be effectively applied to non-stationary signals, they suffer from the
%adverse consequences of Heisenberg's uncertainty principle, manifested in a
%time-frequency resolution trade-off.  The more general distributions of
%section~\ref{sec:energy} can help us overcome this limitation.
%
