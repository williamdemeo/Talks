% -*- mode: LaTeX; tex-main-file: "../notes.tex"; -*-
% (begin: insert file ambiguity.tex)
%\subsection{Quadratic Time-Frequency Energy\protect\footnotemark}
%\footnotetext{Mallat~\cite{Mallat:1998} (p.107),
%Mertins~\cite{Mertins:1999} (p.265)} 
\section{Energy Distributions}
\label{sec:energy}
Wavelet and windowed Fourier transforms are computed by
correlating the signal with families of time-frequency atoms.  The
time and frequency resolution of these transforms is thus limited by
the time-frequency resolution of the corresponding atoms.  Ideally,
one would like to define a density of energy in a time-frequency
plane with no loss of resolution.  This section presents a
different class of time-frequency representation (TFR) which is not
restricted by the uncertainty principle.  

%\footnotetext{Mertins~\cite{Mertins:1999} (p.265)}
The \emph{Wigner-Ville TFR} is computed by
correlating $x$ with a time and frequency translation of itself.
(Below we refer to the Wigner-Ville TFR simply as 
% the ``\WV'' and, sometimes, as 
the ``\WT.'')  Though it yields some remarkable
properties, the quadratic from of this representation is also 
considered a drawback which limits its application because of the inevitable
cross terms that appear in quadratic forms.  An attempt is usually made to 
attenuate these so-called
``interference terms'' by performing a time-frequency averaging, but this 
procedure results in a loss of resolution. It is not hard to show
that the spectrogram, the scalogram, and all squared time-frequency
decompositions can be written as time-frequency averagings of the
\WT; see, e.g., Mallat~\citeyear{Mallat:1998}.

%Although the \WT\ does not yield positive distributions, it allows extremely
%good insight into signal properties within certain applications.

\subsection{Wigner Transform}
%\ifthenelse{\boolean{nofootnotes}}{\subsection{Wigner Transform}}
%{\subsection{Wigner Transform\protect\footnotemark}
%\footnotetext{Mertins~\cite{Mertins:1999} (p.265)}}
%The ambiguity function represents the signal energy %of the signal 
%in the {\it frequency-time} plane.  We now transform this
%representation into the {\it time-frequency} plane.
\label{sec:wigner}
The quadratic form\footnote{Integrals are over the entire real line unless
  otherwise noted.}
\begin{equation*} %\label{eqn:WignerVille}
\W_x(t,\nu) = 
%\integral 
\int \xtpull \xtpushconj e^{-i2\pi \tau\nu}\,d\tau
\end{equation*}
is known as the \emph{Wigner-Ville distribution}, or \emph{\WT}.
It is the one-dimensional \FT\ of
$\phi_x(t,\tau) = \xtpushconj \xtpull$, with respect to $\tau$.
The function $\phi_x$ has a Hermitian symmetry in $\tau$, so the 
\WT\ is real valued. 
Also, as the two-dimensional \FT\ of the so called \emph{ambiguity function},  
\[
\A_{x}(\xi,\tau) = \int \xtpull \xtpushconj e^{i2\pi \xi t}\,dt
\]
the \WT\ satisfies
\begin{equation}\label{eq:ambig-ft}
\W_x(t,\nu) = %  \iint\limits_{\R^2} 
\int \A_x(\xi,\tau) e^{-i2\pi(\xi t + \nu\tau)}\,d\xi\,d\tau
\end{equation}

The \WT\ localizes the time-frequency structures of $x$.  If
the energy of $x$ is well concentrated in time around $t_0$ and in
frequency around $\nu_0$ then $\W_x$ has its energy centered at
$(t_0,\nu_0)$, with a spread equal to the time and frequency spread of
$x$. 
%\end{define}

\subsection{Interference Structure}% of Bilinear Transforms}
Because the \WV\ transform is a sesquilinear form of the signal, it
does not submit to the principle of linear superposition.  Instead, as in the
quadratic equation, $(a+b)^2 = a^2 + b^2 + ab + ba$, it is easy to verify that
\begin{align}
  \W_{x+y}(t,\nu) &= \W_x(t,\nu) + \W_y(t,\nu)\nonumber \\
      &+ \W_{xy}(t,\nu) +\W_{yx}(t,\nu) \label{eqn:Wignersum} 
\end{align}
where $\W_{xy}$ is the \emph{cross \WT\ } of the signals $x$ and $y$, which is
defined by 
%\begin{eqnarray*}
\begin{equation}
\label{eqn:crossWigner}
\W_{xy}(t,\nu) = \int \xtpull \ytpushconj e^{-i2\pi\nu\tau}\,d\tau 
\end{equation}
%\begin{align}\label{eqn:crossWigner}
%  \W_{xy}(t,\nu) &= \\
%& \integral \xtpull \ytpushconj e^{-i2\pi\nu\tau}\,d\tau \nonumber%\\
%  &=& \integral \Xtpush \Ytpullconj e^{-i2\pi\xi t}\,d\xi\nonumber
%\end{eqnarray*}\end{align}

%%It can also be shown that~(\ref{eqn:crossWigner}) is the
%%two-dimensional \FT\ of the \emph{cross ambiguity function},
%%which is defined by 
%%%\begin{eqnarray}\label{eqn:crossAmbiguity}
%%\[  \A_{xy}(\xi,\tau) = \integral \xtpull \ytpushconj e^{i2\pi \xi t}\,dt\]%\\
%%%   &=& \integral \Xtpush \Ytpullconj e^{i2\pi \xi t}\,dt\nonumber \\
%%%\end{eqnarray}

We define the \emph{interference term} of
equation~(\ref{eqn:Wignersum}) by
%\begin{eqnarray*}
%  I_{xy}(t,\nu) &=& \W_{xy}(t,\nu) + \W_{yx}(t,\nu) \\
%                 &=& 2\,\Real\left[\W_{xy}(t,\nu)\right]
%\end{eqnarray*}
\begin{align*}
  I_{xy}(t,\nu) &= \W_{xy}(t,\nu) + \W_{yx}(t,\nu)\\
 &= 2\,\Real\left[\W_{xy}(t,\nu)\right]
\end{align*}
This real valued function creates non-zero values at
interesting locations of the time-frequency %$(t,\nu)$ 
plane. 

More generally, for any linear combination of signal components,
\[
  x(t) = \sum_{n=1}^N a_n x_n(t)
\]
the \WT\ is
%\begin{equation}
\begin{align}
  \W_{x} &= \sum_{n=1}^N|a_n|^2 \W_{x_n}(t,\nu) \label{eq:WignerSum} \\
 &+ 2\sum_{n=1}^{N-1}\sum_{k=n+1}^N\,
                 \Real\left[a_n a_k^* \W_{x_nx_k}(t,\nu)\right]
\nonumber
%\end{equation}
\end{align}
Hence, for a signal with $N$ components, the \WT\ contains 
$N(N-1)/2$
additional components.  They result from the interaction of different
components of the signal, and are called ``interference terms'' for
two reasons.  First, the mechanism of their creation is analogous to
the usual interferences that can be observed for physical waves.  A
second reason for this terminology lies in the
effect that these terms can have on the %concerning the readability of a
time-frequency diagram of the signal energy.  As they amount to a
combinatorial proliferation of additional, ``specious'' signal
components, they can inhibit our ability to discern ``true'' signal 
components in the diagram.  

The presence of cross terms in a \WT\ can be regarded as a
natural consequence of its bilinear structure.  This very structure is also
what leads to most of the good properties of the transform (such as
localization).    
No matter whether one views the cross terms as helpful or hindering,
it is important to understand fully the mechanism of their creation.
This is indispensable for drawing the correct interpretation from the 
representation of an unknown signal, and for separating signal component
terms from interference terms if desired \cite{Flandrin:1999}.

In the present work, we study the
cross terms in order to understand how this measure of signal interference
relates to ``musical interference,'' i.e.~dissonance. 

%As a simple example, consider a well localized time-frequency
%atom, $x(t)$, centered at $t=0$, and the two atoms which are time and 
%frequency shifted versions of $x(t)$: 
%\[
%\xpull(t) = 
% \apull \xtpull e^{-i2\pi (\halfxi) t} \quad \text{ and }\quad  
%\xpush(t) = \apush \xtpush e^{i2\pi (\halfxi) t}
%\]
\subsection{Examples}
\label{sec:examples}
\def\a{\mathbf{a}}
\def\b{\mathbf{b}}
%We will start with some general terminology, but quickly proceed to
%specifics relevant to our application.  For a lucid treatment of the
%general theory of time-frequency representations on finite abelian groups,
%see~\cite{Tolimieri:1998}.  

%For some finite abelian group $A$, with dual $A^*$, consider the signal $x \in
%L(A)$, and define the \emph{Weyl-Heisenberg operator}
%$H : A \times A^* \rightarrow  L(A)$ by
%\[ 
%H(\a)x(t) = x_\a(t) = x(t-a)\langle t,a^* \rangle, \qquad t\in A
%\]
%for any pair $\a = (a,a^*) \in A\times A^*$.
%This simply provides a convenient notation for the common translation and
%modulation operation, a special case of which appears in~(\ref{eqn:transmod})
%below.\\ 
%\\
%{\bf Example}\\
For simplicity, suppose that $x \in L(\Z/N)$ represents an elementary signal
component, so that $x$ is a discrete periodic function of period $N$, defined
on the group of integers $\Z/N \simeq \{0,1,\ldots,N-1\}$.  In this
special case\footnote{For a more general, lucid treatment of the theory of
  time-frequency representations on finite abelian groups,
  see~\citeN{Tolimieri:1998}.} 
the so called \emph{Weyl-Heisenberg} operator is
$H : \Z /N \times \Z /N \rightarrow  L(\Z /N)$,
and is defined by
\begin{align*}
H(\a)x(n) &= x_\a(n) \nonumber\\
&= x(n-a_1)\;e^{i2\pi a_2n/N},
\quad x \in L(\Z/N) %\label{eqn:transmod}
\end{align*}
for any $\a = (a_1,a_2) \in \Z /N \times \Z /N$.

The canonical example used to describe the structure of the 
%\WV\ transform
cross terms of the \WT\ begins with a well localized time-frequency atom
$x(t)$ centered at $t=0$. %\cite{Flandrin:1999}. 
From this we construct two atoms
which are time 
and frequency shifted versions of $x(t)$.  In particular, let $\a = (a_1,a_2)$
and $\b = (b_1,b_2)$ and consider
\begin{alignat*}{2}%\[
\alpha\; x_\a(t)&=\alpha \;x(t-a_1)\; e^{i2\pi a_2t}, \qquad \alpha\; &\geq 0\\
\beta\; x_\b(t)&=\beta\;x(t-b_2)\;e^{i2\pi b_2t}, \qquad \beta &\geq 0
\end{alignat*}%\]
The \WT\ of the composite signal $x_\a(t)+x_\b(t)$ is
\[
  \W_{x_\a+x_\b}(t,\nu) 
    = \W_{x_\a}(t,\nu)+\W_{x_\b}(t,\nu) + I_{x_\a x_\b}(t,\nu)
\]
The \emph{covariance property} of the \WT\ ensures that
the shifted atoms, taken individually, have Wigner representations
given by  
\begin{align*}
\W_{x_\a}(t,\nu) &= \alpha^2\W_x(t-a_1,\nu-a_2)\\
\W_{x_\b}(t,\nu) &= \beta^2\W_x(t-b_1,\nu-b_2)
\end{align*}
Since the energy of $\W_x$ is centered at $(0,0)$, the energy of
$\W_{x_\a}$ and $\W_{x_\b}$ is concentrated in neighborhoods of
$\a = (a_1,a_2)$ and $\b = (b_1,b_2)$, respectively.  A direct calculation
verifies that the interference term is
%\[I_{x_1x_2}(t,\nu)=2a_1a_2\W_x(t-t_m,\nu-\nu_m)
%\cos[\vartheta(t_1,t_2,\nu_1,\nu_2,\phi_1,\phi_2)]\]
%\cos\vartheta\]
%where
%\[\vartheta \equiv (t-t_m)\Delta \nu-(\nu-\nu_m)\Delta t +\Delta \phi\]
%\left[(t-t_m)\Delta \nu-(\nu-\nu_m)\Delta t +\Delta \phi \right]\]
\begin{align*}
  I_{x_\a x_\b}(t,\nu)&=
    2\alpha \beta \W_x(t-t_m,\nu-\nu_m)\\
    & \times \cos\left\{2\pi \left[(t-t_m)\Delta\nu 
               -(\nu-\nu_m)\Delta t \right]\right\}
\end{align*}
where %and
\begin{alignat*}{2}%\[%begin{eqnarray*}
  t_m    &= \frac{a_1+b_1}{2}, \quad 
  \nu_m &&= \frac{a_2+b_2}{2}\\
  \Delta t    &= a_1-b_1,\quad
  \Delta \nu &&= a_2-b_2
\end{alignat*}
This is an oscillatory waveform concentrated in a neighborhood of the
point in the time-frequency plane that is the geometric midpoint 
%, $(t_m,\nu_m)$,
between the individual components.  The frequency of the oscillations
is proportional to the Euclidean distance
$\sqrt{\Delta \nu^2 + \Delta t^2}$ that separates the points
$\a $ and $\b$, where the individual atoms are
concentrated.  The direction of these oscillations is perpendicular to
the line that joins these two center points.
% $(t_1,\nu_1)$ and $(t_2,\nu_2)$.

%\begin{define}{\bf Physical Interpretation. }  It is possible to attach
\paragraph{Physical Interpretation.} 
It is possible to attach physical meaning to the interference structure of the
\WT. For the most basic case, in which the signal is a simple superposition
of pure frequencies, the cross term % of the \WV\ 
can be regarded as a signature of the \emph{beat frequency} resulting
from the interaction between %coexistence of 
the individual frequencies.  
To see this from the preceeding example, let
$x$ be a pure sinusoidal wave, $x(t) = e^{i2\pi\nu_m t}$ at the (mid-point)
frequency $\nu_m$, and suppose $\a = (0,-\halfDnu)$, $\b = (0,\halfDnu)$.
Then $x_\a$ and $x_\b$ are the frequency shifted versions of $x$,
\[%\begin{align*}
x_\a(t)%=\xfpull(t) = x(t)e^{-i2\pi(\halfDnu)t}\\
      =e^{i2\pi(\nu_m-\halfDnu)t},\;
x_\b(t)%=\xfpush(t) = x(t)e^{i2\pi(\halfDnu)t}\\
      = e^{i2\pi(\nu_m+\halfDnu)t}
\]%\end{align*}
%\end{define}
The \WT\ of the composite signal $x_\a + x_\b$ is given by
%\begin{align}
%\begin{equation}
\[  \W_{x_\a+x_\b}(t,\nu)
    = \W_{x_\a}(t,\nu)+\W_{x_\b}(t,\nu) + I_{x_\a x_\b}(t,\nu)\]%\nonumber\\
\[ =\delta(\nu-(\nu_m-\halfDnu))+\delta(\nu-(\nu_m+\halfDnu))\]%\nonumber\\
\begin{equation}
+ \delta(\nu-\nu_m)\,2\cos(2\pi \Delta\nu t) \label{eq:WignerBeats}
\end{equation}
Now let us relate this expression to the physical phenomenon of
beats, which are perceived most easily when the distance between
signal components %, $\Delta\nu$, 
is small.  To do so, we write the signal as follows:
\begin{align}
x_\a(t)+x_\b(t) &=
\half \,e^{i2\pi(\nu_m-\halfDnu)t}
      + \half \,e^{i2\pi(\nu_m+\halfDnu)t}\nonumber\\
      &+ \cos(2\pi\halfDnu t)\,e^{i2\pi\nu_m t}\label{eq:sigbeats}
\end{align}
%\begin{align}
%x_\a(t)+x_\b(t) 
%    &= e^{i2\pi(\nu_m-\halfDnu)t} 
%       + e^{i2\pi(\nu_m+\halfDnu)t}\nonumber \\
%    &= (e^{-i2\pi\halfDnu t} + e^{i2\pi\halfDnu t})\,
%       e^{i2\pi\nu_m t}\nonumber \\ 
%    &= 2\,\cos(2\pi\halfDnu t)\,e^{i2\pi\nu_m t}\label{eq:sigbeats}
%\end{align}
%This expresses the signal as a tone with frequency $\nu_m$ and
%modulated amplitude $2\cos(2\pi\halfDnu t)$.  
%This is a sinusoidal of frequency $\nu_m$ with 
When the components $x_\a(t)$ and $x_\b(t)$ are close together in
frequency -- that is, when $\Delta\nu$ is small -- the cosine term is
slowly varying as compared to the exponential term, and the resulting
signal can be viewed as a simple tone of frequency $\nu_m$ with
%slowly varying 
a modulated amplitude envelope, with modulation frequency $\Delta\nu$.
The term ``beating'' refers to such
amplitude modulations. % due to interaction of signal components.

Comparing~(\ref{eq:sigbeats}) with~(\ref{eq:WignerBeats}), it is
clearly the interference term of the \WT\ which specifies
the existence and nature of beats in the composite signal.
% of interacting signal components.
%existence of the beat frequency component 
%and stands for its realization in the time-frequency plane.   
%To make for easier comparison, we can also write this signal as
%\begin{align*}
%x_\a(t)+x_\b(t) &=
%\half \,e^{i2\pi(\nu_m-\halfDnu)t}
%      + \half \,e^{i2\pi(\nu_m+\halfDnu)t}\\
%      &+ \cos(2\pi\halfDnu t)\,e^{i2\pi\nu_m t}
%\end{align*}

%%\begin{define}{\bf Interferences in the Ambiguity Plane. }
%%The ambiguity function is also a bilinear form, has its own
%%interference structure, and represents both signal and interference
%%components in the frequency-time, or ``Doppler-delay,'' plane.
%%A superposition of two signal components, 
%%$x_1(t) + x_2(t)$, 
%%yields the following frequency-time representation:
%%\[
%%  \A_{x_1+x_2}(\xi,\tau) 
%%    = \A_{x_1}(\xi,\tau)+\A_{x_2}(\xi,\tau)
%%      + \A_{x_1x_2}(\xi,\tau) + \A_{x_2x_1}(\xi,\tau) 
%%\]
%%\end{define}
%%Let the interference term be denoted
%%\[
%% J_{x_1x_2}(\xi,\tau) = \A_{x_1x_2}(\xi,\tau) + \A_{x_2x_1}(\xi,\tau) 
%%\]
%%The Fourier relation of equation~(\ref{eq:ambig-ft}), between the \WT\
%%and the ambiguity function, proves that 
%%\[
%% I_{x_1x_2} = \iint\limits_{\R^2} J_{x_1x_2}(\xi,\tau) 
%%              e^{-i2\pi(\xi t + \nu\tau)}\,d\xi\,d\tau
%%\]
%%
%%For the previous example, in which 
%%\[%\begin{align*}
%%x_1(t)%\xfpull(t)% = x(t)e^{-i2\pi(\halfDnu)t}\\
%%      =e^{i2\pi(\nu_m-\halfDnu)t},\quad \text{ and } \quad
%%x_2(t)%\xfpush(t)% = x(t)e^{i2\pi(\halfDnu)t}\\
%%      = e^{i2\pi(\nu_m+\halfDnu)t}
%%\]%\end{align*}
%%direct calculation yield
%%\[%\begin{alignat*}{2}
%%\A_{x_1}(\xi,\tau)=\delta(\xi)\,e^{i2\pi(\nu_m-\halfDnu)\tau},\quad
%%\A_{x_1x_2}(\xi,\tau)=\delta(\xi+\Delta\nu)\,e^{i2\pi\nu_m\tau}\]%\\
%%\[
%%\A_{x_2}(\xi,\tau)=\delta(\xi)\,e^{i2\pi(\nu_m+\halfDnu)\tau},\quad
%%\A_{x_2x_1}(\xi,\tau)= \delta(\xi-\Delta\nu)\,e^{i2\pi\nu_m\tau}
%%\]%\end{alignat*}
%%Summing these and simplifying yields
%%\begin{align*}
%%%\label{eq:ambig-beats}
%%\A_{x_1+x_2}(\xi,\tau) 
%%%  &= \delta(\xi)(e^{i2\pi(\nu_m-\halfDnu)\tau} 
%%%                +e^{i2\pi(\nu_m+\halfDnu)\tau})
%%%   + (\delta(\xi+\Delta\nu)+\delta(\xi-\Delta\nu))e^{i2\pi\nu_m\tau}\\
%%  &= [2\cos(2\pi\halfDnu\tau)\,\delta(\xi)
%%   +  \delta(\xi+\Delta\nu)+\delta(\xi-\Delta\nu)]\,e^{i2\pi\nu_m\tau}
%%\end{align*}
%%%\begin{define}{\bf Physical Interpretations in the Ambiguity Plane. }
%%The interference term for this example is
%%\[
%%J_{x_1x_2}(\xi,\tau) =    
%%  [\delta(\xi+\Delta\nu)+\delta(\xi-\Delta\nu)]\,e^{i2\pi\nu_m\tau}
%%\]
%%which demonstrates that interferences appear in the ambiguity plane
%%where the Doppler variable $\xi$ is equal to $\pm\Delta\nu$.  This
%%quantity represents the frequency difference between the two signal
%%components.   
%%
%%In the context of a dissonance analysis, one could argue that
%%interferences appearing near the origin are responsible for beating or
%%``roughness'' as it is here that signal components are %found to be
%%closest in frequency.  However, such an analysis relies on our ability
%%to separate signal and interference components when computing the
%%time-frequency representation.  This is crucial, particularly for the
%%ambiguity function in which the true signal components appear
%%near the origin.  In that case it is impossible to discern
%%interference terms resulting from interaction among components with
%%small frequency differences, unless we study the interferences terms
%%by themselves.  We consider one method of separating signal and
%%interference energy in the following section.




%The instantaneous power of the signal is 
%\[\left|x_1(t)+x_2(t)\right|^2 = 2 [1+\cos(2\pi\xi t)]\]
%Hence, it is governed by fluctuations that have a longer period if the
%two frequencies are closer to each other.  Owing to the correct
%marginal energy densities of the \WV\ transform, this value must
%coincide with the sum of all amplitudes of the signal terms
%\emph{and} the cross terms at this instant.  

