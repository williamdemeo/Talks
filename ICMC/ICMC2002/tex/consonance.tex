% -*- mode: LaTeX; tex-main-file: "../notes.tex"; -*-
\section{Measures of Consonance}
In later sections, we consider signal analysis methods that are
particularly well suited to the type of musical analysis we wish to
perform.  Therefore, we should first consider %crucial that we have a good grasp of the
some of the musical ideas underlying and motivating our work.
The next section presents one such notion --
musical \emph{consonance}.  We also briefly discuss some existing
quantitative measures of this concept; the reader can find a more detailed
treatment in Sethares \citeyear{Sethares:1997}.
% (end: insert file synthesis.tex)

% (begin: insert file consonance.tex)
\subsection{Consonance and Dissonance}
\label{sec:consonance}
According to Tenney~\citeyear{Tenney:1988} and Sethares
\citeyear{Sethares:1997}, the historical usage of the term {\it
consonance} 
%(and its antonym, {\it dissonance}) 
can be classified according to five distinct categories:
%(see~\citeN{Sethares:1997} for more details):
\emph{melodic consonance} (CDC-1), \emph{polyphonic consonance} (CDC-2),
\emph{contrapuntal consonance} (CDC-3), \emph{functional consonance} (CDC-4),
\emph{sensory consonance} (CDC-5).  

In this paper, the focus is on CDC-1 and CDC-5, so we describe only these.
Briefly, \emph{melodic consonance} applies to successive melodic intervals and
describes these intervals as either consonant or dissonant depending on the
surrounding melodic context; it refers to relatedness of pitches sounded
successively, or the \emph{melodic contour}.  \emph{Sensory consonance}
equates consonance with smoothness and the absence of beats, and equates
dissonance with roughness and the presence of beats.

The definition of sensory consonance is based on the  phenomenon of
beats.  If two pure sine tones are sounded at almost the same
frequency, then beating occurs due to the interference between the
tones.  The beating becomes slower as the two frequencies approach
each other and disappears when they coincide.  Typically, slower  
beats are perceived as gentle and pleasant while fast beats are
perceived as rough and unpleasant. Observing that any sound can be
decomposed into sinusoidal partials, Helmholtz~\citeyear{Helmholtz:1877}
theorized that the perception of dissonance in a musical tone is
determined by the presence and quality of beats among the tone's
interacting partials. 

The present research effort is directed at the discovery of a
measure which might simultaneously quantify multiple notions of consonance.
In particular, we  would like to exploit the theory of \emph{sensory}
and \emph{tonal} consonance (CDC-2 and CDC-5)
of Plomp and Levelt~\citeyear{Plomp:1965} as well as its elaboration
in~\citeN{Sethares:1997}.  
%Some of this theory is described below in
%section~\ref{sec:disscurves}.  
Briefly, this theory employs functions called
\emph{dissonance curves} which measure the ``sensory'' dissonance, of
a complex tone at each particular instant in time.\footnote{Really, a
small interval of time is required to ascertain what pseudo-periodic
frequencies are present at a particular instant.}   
This provides a useful point-wise measure.
However, we would also like a measure that is dynamic and appeals to a
melodic sense of consonance, as in CDC-1.  For example, a dissonance curve
does not account for dissonance due to melodic changes from one complex tone
to the next.  

%\subsection{Dissonance Curves}
%\label{sec:disscurves}
%Two pure sine waves are both perceptible if the frequencies are well
%separated.  When the frequencies are close together, only one sine
%wave is heard (though possibly with beats).  By varying the frequency
%of one of the waves while keeping the other wave fixed, Plomp and
%Levelt~\cite{Plomp:1965} gathered data revealing subject's preferences
%for various frequency intervals.  An example of what these data
%indicate appears as a \emph{dissonance curve} in 
%Figure~\ref{fig:1DissCurve}. Methods for deriving such curves in
%practice are described in~\cite{Sethares:1997}.  These curves are 
%in stark contrast to existing musical theory.  In particular they imply 
%that, in terms of sensory dissonance, intervals such as the major 7th and
%minor 9th are almost indistinguishable from the octave.  This leads some 
%to argue that these data bear no relation to our notions of musical
%consonance. Regardless of whether they have a direct influence on our 
%ideas about musical intervals, dissonance curves provide a way to measure
%the intervals among \emph{pure sine waves}.  Therefore, they can be 
%useful when analyzing a musical signal that has been decomposed into 
%its spectral components.
%
%\begin{figure}
%\ifthenelse{\boolean{nofigures}}{}{
%%             \pdfimage
%%             width 120 mm 
%%             height 70 mm
%%             {\HOME/figures/1DissCurve.png}
%\centering  
%\includegraphics[width=120mm, height=70mm]{\HOME/figures/1DissCurve}
%}
%\caption{{\small Two sine waves are sounded simultaneously.  Typical
%perceptions include pleasant beating (at small frequency ratios),
%roughness (at middle ratios), and separation into two tones (at first
%with roughness, and later without) for larger ratios.  the horizontal
%axis represents the frequency interval between the two sine waves,
%and the vertical axis is a normalized measure of ``sensory''
%dissonance.  The frequency of the lower sine wave is 400 Hz}}
%\label{fig:1DissCurve}
%\end{figure}
%
%% Leaving out this figure for now:
%%\begin{figure}
%\ifthenelse{\boolean{nofigures}}{}{
%%             \pdfimage
%%             width 120 mm 
%%             height 70 mm
%%             {\HOME/figures/3DissCurves.png}
%\centering  
%\includegraphics[width=120mm, height=70mm]{\HOME/figures/3DissCurves}
%}
%\caption{{\small Two sine waves are sounded simultaneously.  The
%horizontal axis represents the frequency interval between the two sine
%waves, and the vertical axis is a normalized measure of ``sensory''
%dissonance. The plot shows how the sensory consonance and dissonance
%change depending on the frequency of the lower tone.}}
%\label{fig:3DissCurves}
%\end{figure}
%% (end: insert file consonance.tex)

%%% NEWER (UNDERDEVELOPED) IDEAS %%%
%Recall equation~(\ref{eqn:sumpartials}) of section~\ref{sec:synthesis}
%in which a complex tone $x$ is modeled as a weighted sum of sinusoidal
%partials: 
%\[x(t) = \sum_{k=1}^K a_k(t) \cos(\phi_k(t))\]
%where the instantaneous frequency of the $k$th partial is
%%(cf.~\S~\ref{sec:instfreq}, equation~\ref{eqn:sumpartials}): 
%\[\omega_k(t) = \phi'_k(t) \geq 0.\]
%Suppose for simplicity that the set of $K$ instantaneous frequencies
%$\{\omega_k\}$ are known and constant over a small interval (instant)
%of time.  Then...
%%the relative dissonances among the frequencies in a complex
%%tone
%
%It is possible that a particular \emph{instant} in a piece of music
%will have a high sensory dissonance value but, because of this
%instant's relation to its context, the section of the piece containing
%this instant has a low total dissonance value.  We can create measures
%that act in this way.  Consider a simple succession of 3 instants of a
%musical piece represented by $\{S_0, S_1, S_2\}.$  Let $\|\cdot\|$ be
%the measure of dissonance of $S_k.$  It is possible ...

