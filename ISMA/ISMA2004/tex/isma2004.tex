%% LaTeX file 'isma2004.tex'

%% latexfile{
%% author = {William DeMeo},
%% filename = {isma2004.tex},
%% date = {2004.06.02},
%% text = {Main LaTeX input file for isma paper}
%% }
%============================================================
\ismasec{Notation and Background}
%============================================================
This section summarizes the notations,
definitions, and important facts needed below.
The presentation style is terse since the goal %of this section 
is to distill from the %more general 
literature only those results that are most relevant for DSP
applications. 
The books~\cite{An:2003} and~\cite{Tolimieri:1998} treat the
same material in a more thorough and rigorous manner.

Throughout, $\C$ denotes complex numbers, 
$G$ an arbitrary (nonabelian) group, %of order $N$, 
and $\LG$ the collection of complex valued functions on $G$.

%---
\ismasubsec{Cyclic Groups}
%---
A group $C$ is called a \emph{cyclic group} if there exists
$x\in C$ such that every $y\in C$ has the form $y=x^n$ for
some integer $n$.  In this case, $x$ is a
\emph{generator} of $C$. 

%Cyclic groups are frequently constructed as special
%subgroups of arbitrary groups, but it is convenient to have
%notation for a cyclic group of 
%order $N$ without reference to a particular underlying group.
Let the set
\begin{equation}\label{eq:cyclicGroup}
C_N(x) = \{ x^n : 0 \leq n < N\}
\end{equation} 
denote the cyclic group of order $N$ with
generator $x$, and define binary composition by
\begin{equation}\label{eq:binarycomp}
x^m x^n = x^{m+n}, \quad 0\leq m, n < N,
\end{equation}
where $m+n$ is addition modulo $N$.  
%Then $C_N(x)$ is a cyclic group of order $N$ having generator $x$.
In $C_N(x)$, the identity element %of $C_N(x)$ 
is $x^0 = 1$, and the inverse of $x^n$ is $x^{N-n}$.

%Most of the groups discussed above have elements
%in the set $\Z/N$, and we wrote the binary composition on
%the group as addition modulo $N$.   
To say that a group is \emph{abelian} is to specify that the binary
composition of the group is commutative, in which case the
symbol $+$ is usually used to represent this operation.
For nonabelian groups, we write the binary
composition as multiplication.  Since our work involves
both abelian and nonabelian groups, it is notationally
cleaner to write the binary operations of all groups --
whether abelian or not -- as multiplications. As the
following discussion illustrates, groups such as $\Z/N$ with
addition modulo $N$ have a simple multiplicative
representation.

%~~~
\begin{example}\label{ex:ZN}
%~~~
Let $\Z/N = \{0, 1, \ldots, N-1\}$,
and let addition modulo $N$ be the binary composition on $\Z/N$.
This group is isomorphic to the cyclic group $C_N(x)$; %\ie
%\begin{eqnarray}
%\Z/N &=& \{ n : 0 \leq n < N \} \nonumber\\
%&\simeq& \{ x^n : 0 \leq n < N\}= C_N(x).\label{eq:ZNisoCN}
%\end{eqnarray}
indeed, it is by this identification %~(\ref{eq:ZNisoCN}) 
that the binary composition on $\Z/N$ can be written as multiplication. 
More precisely, uniquely identifying each $m \in \Z/N$
with $x^m \in C_N(x)$, the binary composition
$m+n$ is replaced with that of~(\ref{eq:binarycomp}).  
\end{example}

%-----------------------------------------------------------------------
\ismasubsec{Group of Units}
%-----------------------------------------------------------------------
%Multiplication modulo $N$ is a ring product on the group of
%integers $\Z/N$. 
An element $m\in \Z/N$ is called a {\it unit} if
there exists an $n\in \Z/N$ such that $mn = 1$.  The set
$U(N)$ of all units in $\Z/N$ is a group with respect to
multiplication modulo $N$, and is called the 
\emph{group of units}.%unit group} of $\Z/N$. 
The group of units can be described %characterized 
as the set of all integers $0<m<N$ such that $m$ and $N$ are
relatively prime.  
\begin{example}
For $N=8$, $U(8) = \{1, 3, 5, 7\}$.
%\begin{equation}\label{eq:unitGroup}
%U(8) = \{1, 3, 5, 7\}.
%\end{equation}
\end{example}


%-----------------------------------------------------------------------
\ismasubsec{Generalized Translation and Convolution}
%-----------------------------------------------------------------------

%\ismasubsubsec{General definition of translation}
For 
$y\in G$, 
the mapping 
$\lt{T}(y)$ 
of 
$\LG$ 
defined 
by 
\begin{equation}\label{eq:trans}
(\lt{T}(y)f)(x) = f(y^{-1}x), \quad x \in G,
\end{equation}
is a linear operator of $\LG$ called 
\emph{left translation by} $y$.

%---
%\ismasubsubsec{General definition of convolution}
%---
The mapping $\lt{C}(f)$ of $\LG$ defined by 
\begin{equation}
\lt{C}(f) = \sum_{y\in G} f(y) \lt{T}(y), \quad f\in \LG,
\end{equation}
is a linear operator of $\LG$ called 
\emph{left convolution by} $f$.  By definition, for $x\in G$,
\begin{equation}\label{eq:conv}
(\lt{C}(f)g)(x) = \sum_{y\in G} f(y) g(y^{-1}x), \quad g \in \LG.
\end{equation}
%The collection of all left convolutions of $\LG$ is
%$\vs{C}(G) = \left\{\lt{C}(f) : f \in G \right\}$.

For $f, g \in \LG$, the composition
$f * g  = \lt{C}(f)g$
is called the \emph{convolution product}.
The vector space $\LG$ paired with the convolution product
is an algebra, the \emph{convolution algebra over} $G$.
%---
\begin{example}
%---
To gain some familiarity with the general 
definitions of translation %~(\ref{eq:trans}) 
and convolution, %~(\ref{eq:conv}), 
it helps to verify 
that these definitions agree with what we expect 
when $G$ is the familiar abelian group $\Z/N$. 
Indeed, for this special case,~(\ref{eq:trans}) becomes
%translation of $\LG$ by $y\in G$ is defined by 
\begin{equation}
(\lt{T}(y)f)(x) = f(x-y),  \qquad x \in G,
\end{equation}
and~(\ref{eq:conv}) becomes
%convolution of $\LG$ by $g\in \LG$ is defined by
%\begin{equation}
%\lt{C}(g)f = \sum_{y \in G}g(y)\lt{T}(y)f, \qquad f \in \LG,
%\end{equation}
%which, at $x \in G$, is
\begin{equation}
(\lt{C}(g)f)(x) =\sum_{y \in G} g(y)f(x-y).
\end{equation}
\end{example}



%-----------------------------------------------------------------------
\ismasubsec{The Group Algebra $\CG$}
%-----------------------------------------------------------------------
\label{sec:groupalgebra}
The \emph{group algebra} $\CG$ is the space of all formal sums
\begin{equation}
f = \sum_{x\in G} f(x) x, \quad f(x) \in \C,
\end{equation}
with the following operations for $f, g \in \CG$:
\begin{equation}
f+g = \sum_{x\in G} (f(x) + g(x))x,
\end{equation}
\begin{equation}
\alpha f = \sum_{x\in G} (\alpha f(x)) x, 
           \quad \alpha \in \C,
\end{equation}
\begin{equation}
fg = \sum_{x\in G}\left(\sum_{y\in G} f(y)g(y^{-1}x)\right)x. 
\end{equation}

The mapping $\lt{L}(g)$ of $\CG$ defined by 
$\lt{L}(g)f = gf$
is a linear operator on the space $\CG$ called 
\emph{left multiplication by} $g$.  
Since $y\in G$ can be identified with the formal
sum $e_y \in \CG$ consisting of a single nonzero term,
\begin{equation}\label{eq:leftmult}
yf = \lt{L}(e_y)f = \sum_{x\in G}f(y^{-1}x) x.
\end{equation}
In relation to translation of $\LG$, (\ref{eq:leftmult}) is the
$\CG$ analog. 

The mapping $\varTheta: \LG \to \CG$ defined by
\begin{equation}\label{eq:iso}
\varTheta(f) = \sum_{x\in G} f(x) x, \quad f\in \LG,
\end{equation}
is an algebra isomorphism of the convolution algebra $\LG$
onto the group algebra $\CG$.  
Thus we can identify
$\varTheta(f)$ with $f$, using  context to decide whether
$f$ refers to the function in $\LG$ or the formal sum in
$\CG$.  

An important aspect of the foregoing isomorphism is the
correspondence between the translations of the spaces.
Translation of $\LG$ by $y\in G$ %$\T(y)$ 
corresponds to left multiplication of $\CG$ by $y\in G$.
%$\lt{L}(y)$ 
Convolution of $\LG$ by $f\in \LG$ corresponds to
left multiplication of $\CG$ by $f\in \CG$. 
%We state 
%these relations symbolically as follows:
%\begin{center}
%\begin{tabular}{ccc}
%  $\LG$ & $\simeq$ & $\CG$ \\
%  $\lt{T}(y)$ & $\leftrightarrow$ & $\lt{L}(y)$\\
%  $\lt{C}(f)$ & $\leftrightarrow$ & $\lt{L}(f)$
%\end{tabular}
%\end{center}

\begin{figure}
  \centerline{\epsfig{figure=figures/t_cyclicshift,width=70mm, height=50mm}}
  \caption{{\small {\it An impulse
      $f\in \CA$ and a few abelian group translates, $x^2f, x^9f,
      x^{-7}f$. }}}
  \label{fig:cyclicshift}
\end{figure}




%------------------------------------------------------------------------
\ismasubsec{Translation-Invariant Subspaces}
%------------------------------------------------------------------------
A subspace $\vs{V}$ of the space $\CG$ is called a
\emph{left ideal} if 
\begin{equation}
u\vs{V} = \{uf : f \in \vs{V}\} \subset \vs{V}, \quad u \in G. 
\end{equation}
A left ideal of $\CG$ corresponds to a subspace of $\LG$
invariant under all left translations.  If $\vs{V}$ is a
left ideal, then, by linearity, 
$g\vs{V} \subset \vs{V}$ for all $g \in \CG$.
The set $\CG g$, defined by 
$\{fg : f \in \CG\}$, is a left ideal of $\CG$,
called \emph{the left ideal generated by} $g$ in $\CG$. 
%$\CG g = \CG$ if and only if $g$ is an invertible element in $\CG$. 
A left ideal $\vs{V}$ of $\CG$ is called \emph{irreducible}
if the only left ideals of $\CG$ contained in $\vs{V}$ are
$\{0\}$ and $\vs{V}$. The sum of two distinct, irreducible
left ideals is always a direct sum. 
% (\cite{An:2003}, p.~129).

For \emph{abelian} group $A$, the group algebra $\C A$ of
signals is decomposed into a direct sum of irreducible ideals.  
Since multiplication of $\C A$ by elements of $G$
corresponds to translation, ideals represent
translation-invariant subspaces.  Furthermore, in the 
abelian case, such translation-invariant subspaces are
one-dimensional.   

Similarly, for \emph{nonabelian} group $G$, the group algebra
$\CG$ is decomposed into a direct sum of left ideals
and, again, the ideals are translation-invariant
subspaces.  However, some of them must now be multi-dimensional,
and herein lies the potential advantage of using nonabelian
groups for indexing the data. The left translations
are more general and represent a broader class of 
transformations. Therefore, projections of data into the
resulting left ideals can reveal more complicated partitions
and structures as compared with the Fourier components in
the abelian group case. 



%=======================================================================
\ismasec{Nonabelian Group DSP}
%=======================================================================
This section presents some basic theory of digital 
signal processing (DSP), but relies on a more general 
mathematical formalism than that employed by the 
standard textbooks on the subject.\footnote{A few notable
  exceptions are 
  \cite{{An:2003}, {Tolimieri:1998}}, {Chirikjian:2002}.}
%\cite{{An:2003},{Chirikjian:2001},{Tolimieri:1998},{Tolimieri:1997}}}.
%the book by Chirikjian and Kyatkin~  %and the books by Tolimieri and An

%------------------------------------------------------------------------
\ismasubsec{Main Theorems}
%------------------------------------------------------------------------
A \emph{character} of $G$ is a group homomorphism of $G$
into $\C^{\times}$, where $\C^{\times} = \C \setminus \{0\}$.
In other words, the mapping $\varrho: G \to \C^{\times}$ is 
a character of $G$ if it satisfies 
$\varrho(xy) = \varrho(x)\varrho(y), \, x, y \in G$.
%There is always at least one character, the 
%\emph{trivial character}, which is 1 for all $y\in G$. 
Let $G^*$ denote the set of all characters of $G$.

By the identification~(\ref{eq:iso}) between $\LG$ and
$\CG$, a character $\varrho \in G^*$ can be viewed as
a formal sum,
\begin{equation}
\varrho = \sum_{x\in G}\varrho(x)x.
\end{equation}
Therefore, $G^* \subset \CG$. 
Expressing the characters as formal sums 
leads to simple proofs of important DSP results.
\begin{theorem}\label{thm:char-action}
If $\varrho$ is a character of $G$, then
\begin{equation}
y\varrho = \varrho y = \varrho(y^{-1})\varrho, \quad y\in G.
\end{equation}
\end{theorem}
%\begin{proof}
%By a change of variables,
%\begin{equation}
%\varrho y = \sum_{x\in G}\varrho(x)xy = \sum_{x\in G}\varrho(xy^{-1})x, \quad y\in G.
%\end{equation}
%By homomorphism property, $\varrho(xy^{-1}) =
%\varrho(x)\varrho(y^{-1})$.  Therefore, 
%$\varrho y = \varrho(y^{-1})\varrho$, for all $\quad y\in G$.
%The same change of variables argument works for $y\varrho$.
%\qed
%\end{proof}
\begin{theorem}\label{thm:eigenvector}
%If $\varrho\in G^*$ is a nontrivial character of $G$, then
For $\varrho\in G^*$,
\begin{equation}
\frac{1}{|G|} \sum_{x\in G} \varrho(x) = 
\begin{cases}  1, & \varrho(x)=1, \forall x\in G,\\
0, & \text{otherwise.}
\end{cases}
\end{equation}
\end{theorem}
where $|G|$ is the order of $G$.
%\begin{proof}
%%Let $y\in G$ be such that $\varrho(y) \neq 0$, and $\varrho(y) \neq 1$.  (If no
%%such $y$ exists, the result is obvious.)
%By a change of variables,
%\begin{equation}
%\varrho(y) \sum_{x\in G} \varrho(x) = \sum_{x\in G}
%\varrho(yx) = \sum_{x\in G} \varrho(x), \quad y\in G.
%\end{equation}
%Therefore, either (a) $\varrho(x)=1, \forall x\in G$, 
%or (b) $\sum \varrho(x)=0$.
%\qed 
%\end{proof}
Theorem~\ref{thm:char-action} shows that every character is
an eigenvector of left-multiplication by elements of the
group $G$, so we call them $L(G)$-eigenvectors.
Therefore, by linearity, the characters are eigenvectors
of left-multiplication by $f\in \CG$ (convolution by
$f\in \LG$).  This is re-stated more formally as the following
formula for the $G$-\emph{spectral components} of $f$:
\begin{corollary}\label{cor:FT}
If $\varrho\in G^*$ and $f\in \CG$, then
\begin{equation}\label{eq:FT}
f\varrho = \varrho f = \hat{f}(\varrho)\varrho,
\end{equation}
where $\hat{f}(\varrho) = \sum_{y\in G} f(y) \varrho(y^{-1})$.
%\end{cor}
\end{corollary}
\begin{proof}
By Theorem~\ref{thm:char-action},
\begin{equation}
f\varrho = \sum_{y\in G} f(y) y\varrho =  \sum_{y\in G} f(y)\varrho(y^{-1}) \varrho
\end{equation}
Similarly for $\varrho f$, mutatis mutandis.
\qed
\end{proof}

The functions which make up the standard Fourier basis
% -- the exponential functions -- 
are eigenvectors of standard convolution.  
As seen in the proof of~\ref{thm:eigenvector}, 
this is merely a consequence of the fact that
exponential functions satisfy properties which allow us to
call them characters.  The notion of a character basis
generalizes the Fourier basis to include bases which
can diagonalize any linear combination of 
left group multiplications.
%%% LEFT OFF with notes on p. 132 %%%
\begin{corollary}\label{cor:idemp}
If $\lambda, \tau \in G^*$, then
\begin{equation}
\lambda \tau = 
\begin{cases}  |G|\lambda, & \tau=\lambda,\\
0, & \tau\neq\lambda.
\end{cases}
\end{equation}
\end{corollary}
%\begin{proof}
%Suppose $\tau = \lambda$; then,
%\begin{eqnarray*}
%\lambda \tau %= \lambda\lambda &=& \sum_{x\in G} \tau(x)x\tau\\
%= \sum_{x\in G} \lambda(x)\lambda(x^{-1})\lambda
%= \sum_{x\in G} \lambda(1)\lambda = |G|\lambda
%\end{eqnarray*}
%Suppose $\tau \neq \lambda$. By definition,
%\begin{equation}
%\hat{\lambda}(\tau) = \sum_{x\in G}\lambda(x)\tau(x^{-1}) =
%\sum_{y\in G}\lambda(y^{-1})\tau(y) = \hat{\tau}(\lambda)
%\end{equation}
%By~(\ref{eq:FT}), 
%$\hat{\lambda}(\tau)\tau = \lambda\tau = \tau\lambda =\hat{\tau}(\lambda)\lambda $.  
%Since $\hat{\lambda}(\tau) = \hat{\tau}(\lambda)$ and $\tau
%\neq \lambda$, it must be the case that
%$\hat{\tau} = 0$ and $\lambda\tau = 0$.
%\qed
%\end{proof}



%----------------------------------------------------------------------
\ismasec{Semidirect Product Groups}
%----------------------------------------------------------------------
%To determine whether a particular group is useful for a DSP
%application, we must specify exactly how this group
%represents the data.
%The group representation may reduce computational
%complexity, or it may simply make it easier to state,
%understand, or model a given problem.

This section describes %procedures for specifying and studying 
a simple class of nonabelian groups that have
proven useful in applications -- 
\emph{abelian by abelian semidirect products}. 
%These are
%perhaps the simplest extension of abelian groups
%% case, these are
%%groups of the form $G = A \sdp B$, where $A$ and $B$
%%are abelian groups. Not surprisingly, 
%and DSP over such groups closely resembles that over abelian
%groups.  However, the resulting processing tools can have
%vastly different characteristics. 

%%% LEFT OFF HERE --> resume from top of page 109
%%% LEFT OFF with notes on p. 132 %%%

%-----------------------------------------------------------------------
%\ismasubsec{Action Group}
%-----------------------------------------------------------------------
Let $G$ be a finite group of order $N$, $K$ a subgroup of $G$,
and $H$ a normal subgroup of $G$. If $G = HK$ and $H \cap
K = \{1\}$, then we say that $G$ is the 
%\emph{internal semidirect product}
\emph{semidirect product} $G = H \sdp K$. 
It can be shown that $G = H \sdp K$ if and only if every $x \in
G$ has a unique representation of the form $x = yz, \; y\in H,
z\in K$.

Denote by $Aut(H)$ the set of all \emph{automorphisms} of
$H$. The mapping $\Psi:K\rightarrow Aut(H)$ defined by  
\begin{equation}\label{eq:homo}
\Psi_z(x) = zxz^{-1}, \quad z\in K, x\in H
\end{equation}
is a group homomorphism. 
Define the binary composition in $G$
%relative to the representation $G= H\sdp K$
in terms of $\Psi$ as follows:
\begin{equation}\label{eq:PsiProd}
x_1x_2 = (y_1z_1)(y_2z_2)= y_1\Psi_{z_1}(y_2)z_1z_2,
\end{equation}
\[
y_1, y_2 \in H,\; z_1, z_2 \in K. 
\]

If $K$ is a normal subgroup of $G$,
then $y^{-1}Ky = K$ for all $y\in G$, 
%(by definition of \emph{normal subgroup}) in which case, %$[H, K] = \{1\}$, 
and $G$ is simply the cartesian product $H\times K$ 
with component-wise multiplication. 
What is new in the semidirect product is
the possibility that $K$ acts nontrivially on $H$. 
For this reason, $K$ is sometimes called the ``action group.''

%-----------------------------------------------------------------------
\ismasubsec{Simplest Nonabelian Example}
%-----------------------------------------------------------------------
If the mapping $\Psi$ given in~(\ref{eq:homo}) is defined over
$K=U(N)$, then $\Psi$ is a group isomorphism.
Under this identification, we can form the semidirect
product $G = H\sdp K$, with $H = C_N(x)$ and $K$ a
subgroup of $U(N)$.  Throughout this section, $G$ will
denote such a semidirect product group.

The elements $u\in K$ are integers.
However, we follow~\cite{An:2003} and 
use $k_u$ to denote the element
$u\in K$ as this avoids confusion that can arise 
on occasion.\footnote{This notation is especially useful when $K$ 
is a cyclic group with generator $u$.  If  we denote elements of $K$ by $k_u^j$, instead of 
by $u^j$, it is easier to distinguish them from elements of the abelian group $C_N(x)$.}

Without loss of generality, assume the action 
group $K$ is a cyclic  group of order $J = |K|$ with generator
$u$. We identify each element of $K$ with an index, and denote 
the set of elements by $K = \{k_u^j: 0\leq j < J\}$.
Thus, to each $k_v \in K$, there corresponds a $j\in \Z$ 
such that $k_u^j=k_v$.
We use $x^n k_v$ and $x^n k_u^j$ to denote  
typical points of $G=C_N(x)\sdp K$.

Given two points in $G$, say $z = x^m k_u$
and $y=x^n k_v$, define multiplication %on $G$ 
according
to~(\ref{eq:PsiProd}) as follows:
\begin{equation}\label{eq:prod}
  zy = (x^m k_u)(x^n k_v) = x^{m+u n} k_u k_v,
\end{equation}
where %In~(\ref{eq:prod}), 
$m+ u n$ is taken modulo $N$.
Since $k_v=k_u^j$ for some $j\in \Z$, then 
$k_u k_v = k_u^{1+j}$, and $zy = x^{m+u n} k_u^{j+1}$.

Let $z = x^m k_v$ and suppose $k_w$ is the inverse 
of $k_v$ in $K$.  Then the inverse of $z$ must be
$z^{-1}=x^{N-wm} k_w$, since this satisfies
%\begin{equation}\label{eq:inv}z^{-1}=x^{N-wm} k_w.\end{equation}
$\inverse{z}z \equiv 1$.

Suppose $K \subset U(N)$ has order $|K|=J$, 
and consider the semidirect product group
with elements
\begin{equation}
  G   = \{x^n k_u^j : 0 \leq n < N, 0 \leq j < J\}.\label{eq:sdp}
\end{equation}
%\ismasubsec{Translations on Semidirect Product Groups}
For $f\in \CG$, %the formal sum can be expressed in the following ways:
\begin{equation}
  f = \sum_{y\in G} f(y)y= \sum_{n,j} f(x^n k_u^j)x^n k_u^j,
\end{equation}
%where $0\leq n <N$ and $0\leq j < J$.

As above, translations of $\CG$ are defined as
left-multiplication by elements of $G$.  
For semidirect product~(\ref{eq:sdp}) 
there is a simple dichotomy of translation types that arise
from left-multiplication by elements of $G$.  %Translations of the
First, the familiar ``abelian translates'' 
are obtained upon left-multiplication by powers of $x$
(Figure~\ref{fig:cyclicshift}).  
By change of variables, 
\begin{equation}
x^mf = \sum_{n,j} f(x^{n-m} k_u^j)x^n k_u^j,
\end{equation}
which is simply a ``right shift'' of $f$ by $m$ units.
Similarly, left-multiplication by powers of
$x^{-1}$ effects ``left shift'' of $f$. 
(Recall, $x^{-1} \equiv x^{N-1}$
and $x^{-m} \equiv x^{N-m}$.)  

Of the second type are the ``nonabelian translates,'' 
obtained upon left-multiplication by $k_v \in K$.
\begin{equation}
  k_vf %&=& \sum_{n,j} f(x^n k_u^j)k_v x^n k_u^j\nonumber\\
  = \sum_{n,j} f(k_v^{-1}x^n k_u^j)x^n k_u^j.
\end{equation}
Suppose that $k_w = k_u^\ell$ is the inverse of $k_v$ in $K$.  Then,
\begin{equation}
  k_vf %  &=& \sum_{n,j} f(k_w x^{n}k_u^j)x^n k_u^j\nonumber\\
  = \sum_{n,j} f(x^{wn}k_u^{\ell + j})x^n k_u^j\label{eq:nonabtrans}
\end{equation}
%As usual, summation is over $0\leq n <N$ and $0\leq j < J$.
From equation~(\ref{eq:nonabtrans}) it is clear that $k_vf$ 
results in a more complex transformation than $x^m f$.

%% \newcommand{\zmv}{\ensuremath{x^m k_v}}
%% \newcommand{\zmvInv}{\ensuremath{x^{N-wm} k_w}}

For the general element $z = \zmv \in G$ with 
inverse $z^{-1}=\zmvInv$ %(equation~(\ref{eq:inv})), 
we derive rules for generalized translations.
% with respect to $z$ and $z^{-1}$.
\begin{equation*}
  zf = \sum_{y\in G} f(z^{-1}y)y= \sum_{n,j} f(x^{N-w(m-n)} k_w k_u^j)x^n k_u^j
%  &=& \sum_{y\in G} f(y)zy = \sum_{y\in G} f(z^{-1}y)y\nonumber \\  
%  &=& \sum_{n,j} f(\zmvInv \, x^n k_u^j) x^n k_u^j\nonumber\\
%  &=& \sum_{n,j} f(x^{N-w(m-n)} k_w k_u^j)x^n k_u^j
\end{equation*}
\begin{equation*}
  z^{-1}f = \sum_{y\in G} f(zy)y= \sum_{n,j} f(x^{m+vn} k_vk_u^j)x^n k_u^j
%  &=& \sum_{y\in G} f(y)z^{-1}y = \sum_{y\in G} f(zy)y\nonumber \\  
%  &=& \sum_{n,j} f(\zmv x^n k_u^j)x^n k_u^j\nonumber\\
%  &=& \sum_{n,j} f(x^{m+vn} k_vk_u^j)x^n k_u^j
\end{equation*}



%=======================================================================
\ismasec{Examples}
%=======================================================================
As seen above, when varying group structures are placed on indexing sets,
and products in the resulting group algebra are computed, 
interesting signal transforms obtain.  In this section,
we elucidate the nature of these operations by
examining some simple concrete examples in detail.

\ismasubsec{Semidirect Product Example\protect\footnotemark}
\footnotetext{An and Tolimieri (2003), page 125.}
Let $G_2$ be the \emph{dihedral group} %$\vs{D}_{2N}(x, k_{N-1})$,
%and suppose the elements of $G_2$ are indexed as follows:
with elements
\begin{eqnarray*}
G_2&=& C_N(x) \sdp \{1, k_{N-1}\}\\
&=& \{x^n k_{N-1}^j : 0 \leq n < N, 0 \leq j < 2\}.
\end{eqnarray*}
We order the elements of $G_2$ as follows:
\[
\{1, x, \ldots, x^{N-1}, k_{N-1}, xk_{N-1}, \ldots, x^{N-1}
k_{N-1}\}
\]
Thus, $G_2$ is divided into two blocks of $N$-samples.

By describing the translations of functions in $\CG_2$, 
we will see that the nonabelian translates of
$\CG_2$ are ``intra-block time-reversal'' operations.

Multiplication on $G_2$ obeys the following relations:
\begin{equation}\label{eq:id0}
  x^N = k_{N-1}^2 = 1,
\end{equation}
\begin{equation}
  x^mk_{N-1}^{j+1} \; x^nk_{N-1}^j = 
  \begin{cases} 
    x^{m-n}, & j=0,\\
    x^{m+n}, & j=1.
  \end{cases}
\end{equation}
If $z=x^mk_{N-1}$, then $z^2=1$, thus $\inverse{z}=z$.

For $f\in \CG_2$, 
\begin{equation}\label{eq:f}
  f = \sum_n f(x^n)x^n + f(x^n k_{N-1})x^n k_{N-1}.
\end{equation}
By~(\ref{eq:id0}), the nonabelian translate $k_{N-1}f$
is given by
\[
\sum_n f(k_{N-1}x^n)x^n + f(k_{N-1}x^n k_{N-1})x^n k_{N-1}
\]
which is equivalent to 
%  &=& \sum_n f(x^{(N-1)n} k_{N-1})x^n + f(x^{(N-1)n}) x^n k_{N-1},\nonumber\\
\begin{equation}\label{eq:natran}
\sum_n f(x^{N-n} k_{N-1})x^n + f(x^{N-n}) x^n k_{N-1}.
\end{equation}
Comparing (\ref{eq:f}) and (\ref{eq:natran}), we see that
the nonabelian translate of $f\in \CG_2$ swaps the first $N$
samples of $f$ with the remaining $N$ samples, and performs
a time-reversal within each sub-block.
For a simple linear function, this special translation is
illustrated in Figure~\ref{fig:G2trans}.

\begin{figure}
\centerline{\epsfig{figure=figures/G2transV,width=80mm, height=55mm}}
\caption{{\small {\it A linear signal $f\in \CG_2$, where $N
    = 8$ (left); the element $k_{N-1} \in G_2$ (middle) --
    as an element of the group algebra, $k_{N-1}$ is the ``impulse
    function'' $g \in \CG_2$ with one nonzero coefficient,
    $g(k_{N-1}) =1$; the product $gf = k_{N-1}f$ (right) is,
    in general, the convolution product and is implemented
    by appealing to the convolution theorem and using a 
    generalized FFT algorithm.}}}
    \label{fig:G2trans}
\end{figure}



%\begin{figure}[t]
%\centerline{\epsfig{figure=figures/fline2,width=70mm}}
%\caption{{\it Figures 10.4.4--10.4.6 of An (2003)
%    re-produced with fline.m program}}  
%% (see figures 10.4.4--10.4.6 in An & Tolimieri, pages 216--218)
%\label{fig:10.4.4}
%\end{figure}




