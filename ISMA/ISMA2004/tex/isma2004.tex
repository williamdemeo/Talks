%% LaTeX file 'isma2004.tex'

%% latexfile{
%% author = {William DeMeo},
%% filename = {isma2004.tex},
%% date = {2004.06.02},
%% text = {Main LaTeX input file for isma paper}
%% }
%============================================================
\ismasec{Notation and Background}
%============================================================
This section summarizes the notations,
definitions, and important facts needed below.
The presentation style is terse since the goal %of this section 
is to distill from the %more general 
literature only those results that are most relevant for DSP
applications. 
The books~\cite{An:2003} and~\cite{Tolimieri:1998} treat the
same material in a more thorough and rigorous manner.

Throughout, $\C$ denotes complex numbers, 
$G$ an arbitrary (nonabelian) group, %of order $N$, 
and $\LG$ the collection of complex valued functions on $G$.

%---
\ismasubsec{Cyclic Groups}
%---
A group $C$ is called a \emph{cyclic group} if there exists
$x\in C$ such that every $y\in C$ has the form $y=x^n$ for
some integer $n$.  In this case, $x$ is a
\emph{generator} of $C$. 

%Cyclic groups are frequently constructed as special
%subgroups of arbitrary groups, but it is convenient to have
%notation for a cyclic group of 
%order $N$ without reference to a particular underlying group.
Let the set
\begin{equation}\label{eq:cyclicGroup}
C_N(x) = \{ x^n : 0 \leq n < N\}
\end{equation} 
denote the cyclic group of order $N$ with
generator $x$, and define binary composition by
\begin{equation}\label{eq:binarycomp}
x^m x^n = x^{m+n}, \quad 0\leq m, n < N,
\end{equation}
where $m+n$ is addition modulo $N$.  
%Then $C_N(x)$ is a cyclic group of order $N$ having generator $x$.
In $C_N(x)$, the identity element %of $C_N(x)$ 
is $x^0 = 1$, and the inverse of $x^n$ is $x^{N-n}$.

%Most of the groups discussed above have elements
%in the set $\Z/N$, and we wrote the binary composition on
%the group as addition modulo $N$.   
To say that a group is \emph{abelian} is to specify that the binary
composition of the group is commutative, in which case the
symbol $+$ is usually used to represent this operation.
For nonabelian groups, we write the binary
composition as multiplication.  Since our work involves
both abelian and nonabelian groups, it is notationally
cleaner to write the binary operations of all groups --
whether abelian or not -- as multiplications. As the
following discussion illustrates, groups such as $\Z/N$ with
addition modulo $N$ have a simple multiplicative
representation.

%~~~
\begin{example}\label{ex:ZN}
%~~~
Let $\Z/N = \{0, 1, \ldots, N-1\}$,
and let addition modulo $N$ be the binary composition on $\Z/N$.
This group is isomorphic to the cyclic group $C_N(x)$; %\ie
%\begin{eqnarray}
%\Z/N &=& \{ n : 0 \leq n < N \} \nonumber\\
%&\simeq& \{ x^n : 0 \leq n < N\}= C_N(x).\label{eq:ZNisoCN}
%\end{eqnarray}
indeed, it is by this identification %~(\ref{eq:ZNisoCN}) 
that the binary composition on $\Z/N$ can be written as multiplication. 
More precisely, uniquely identifying each $m \in \Z/N$
with $x^m \in C_N(x)$, the binary composition
$m+n$ is replaced with that of~(\ref{eq:binarycomp}).  
\end{example}

%-----------------------------------------------------------------------
\ismasubsec{Group of Units}
%-----------------------------------------------------------------------
%Multiplication modulo $N$ is a ring product on the group of
%integers $\Z/N$. 
An element $m\in \Z/N$ is called a {\it unit} if
there exists an $n\in \Z/N$ such that $mn = 1$.  The set
$U(N)$ of all units in $\Z/N$ is a group with respect to
multiplication modulo $N$, and is called the 
\emph{group of units}.%unit group} of $\Z/N$. 
The group of units can be described %characterized 
as the set of all integers $0<m<N$ such that $m$ and $N$ are
relatively prime.  
\begin{example}
For $N=8$, $U(8) = \{1, 3, 5, 7\}$.
%\begin{equation}\label{eq:unitGroup}
%U(8) = \{1, 3, 5, 7\}.
%\end{equation}
\end{example}


%-----------------------------------------------------------------------
\ismasubsec{Generalized Translation and Convolution}
%-----------------------------------------------------------------------

%\ismasubsubsec{General definition of translation}
For 
$y\in G$, 
the mapping 
$\lt{T}(y)$ 
of 
$\LG$ 
defined 
by 
\begin{equation}\label{eq:trans}
(\lt{T}(y)f)(x) = f(y^{-1}x), \quad x \in G,
\end{equation}
is a linear operator of $\LG$ called 
\emph{left translation by} $y$.

%---
%\ismasubsubsec{General definition of convolution}
%---
The mapping $\lt{C}(f)$ of $\LG$ defined by 
\begin{equation}
\lt{C}(f) = \sum_{y\in G} f(y) \lt{T}(y), \quad f\in \LG,
\end{equation}
is a linear operator of $\LG$ called 
\emph{left convolution by} $f$.  By definition, for $x\in G$,
\begin{equation}\label{eq:conv}
(\lt{C}(f)g)(x) = \sum_{y\in G} f(y) g(y^{-1}x), \quad g \in \LG.
\end{equation}
%The collection of all left convolutions of $\LG$ is
%$\vs{C}(G) = \left\{\lt{C}(f) : f \in G \right\}$.

For $f, g \in \LG$, the composition
$f * g  = \lt{C}(f)g$
is called the \emph{convolution product}.
The vector space $\LG$ paired with the convolution product
is an algebra, the \emph{convolution algebra over} $G$.
%---
\begin{example}
%---
To gain some familiarity with the general 
definitions of translation %~(\ref{eq:trans}) 
and convolution, %~(\ref{eq:conv}), 
it helps to verify 
that these definitions agree with what we expect 
when $G$ is the familiar abelian group $\Z/N$. 
Indeed, for this special case,~(\ref{eq:trans}) becomes
%translation of $\LG$ by $y\in G$ is defined by 
\begin{equation}
(\lt{T}(y)f)(x) = f(x-y),  \qquad x \in G,
\end{equation}
and~(\ref{eq:conv}) becomes
%convolution of $\LG$ by $g\in \LG$ is defined by
%\begin{equation}
%\lt{C}(g)f = \sum_{y \in G}g(y)\lt{T}(y)f, \qquad f \in \LG,
%\end{equation}
%which, at $x \in G$, is
\begin{equation}
(\lt{C}(g)f)(x) =\sum_{y \in G} g(y)f(x-y).
\end{equation}
\end{example}



%-----------------------------------------------------------------------
\ismasubsec{The Group Algebra $\CG$}
%-----------------------------------------------------------------------
\label{sec:groupalgebra}
The \emph{group algebra} $\CG$ is the space of all formal sums
\begin{equation}
f = \sum_{x\in G} f(x) x, \quad f(x) \in \C,
\end{equation}
with the following operations for $f, g \in \CG$:
\begin{equation}
f+g = \sum_{x\in G} (f(x) + g(x))x,
\end{equation}
\begin{equation}
\alpha f = \sum_{x\in G} (\alpha f(x)) x, 
           \quad \alpha \in \C,
\end{equation}
\begin{equation}
fg = \sum_{x\in G}\left(\sum_{y\in G} f(y)g(y^{-1}x)\right)x. 
\end{equation}

The mapping $\lt{L}(g)$ of $\CG$ defined by 
$\lt{L}(g)f = gf$
is a linear operator on the space $\CG$ called 
\emph{left multiplication by} $g$.  
Since $y\in G$ can be identified with the formal
sum $e_y \in \CG$ consisting of a single nonzero term,
\begin{equation}\label{eq:leftmult}
yf = \lt{L}(e_y)f = \sum_{x\in G}f(y^{-1}x) x.
\end{equation}
In relation to translation of $\LG$, (\ref{eq:leftmult}) is the
$\CG$ analog. 

The mapping $\varTheta: \LG \to \CG$ defined by
\begin{equation}\label{eq:iso}
\varTheta(f) = \sum_{x\in G} f(x) x, \quad f\in \LG,
\end{equation}
is an algebra isomorphism of the convolution algebra $\LG$
onto the group algebra $\CG$.  
Thus we can identify
$\varTheta(f)$ with $f$, using  context to decide whether
$f$ refers to the function in $\LG$ or the formal sum in
$\CG$.  

An important aspect of the foregoing isomorphism is the
correspondence between the translations of the spaces.
Translation of $\LG$ by $y\in G$ %$\T(y)$ 
corresponds to left multiplication of $\CG$ by $y\in G$.
%$\lt{L}(y)$ 
Convolution of $\LG$ by $f\in \LG$ corresponds to
left multiplication of $\CG$ by $f\in \CG$. 
%We state 
%these relations symbolically as follows:
%\begin{center}
%\begin{tabular}{ccc}
%  $\LG$ & $\simeq$ & $\CG$ \\
%  $\lt{T}(y)$ & $\leftrightarrow$ & $\lt{L}(y)$\\
%  $\lt{C}(f)$ & $\leftrightarrow$ & $\lt{L}(f)$
%\end{tabular}
%\end{center}

\begin{figure}
  \centerline{\epsfig{figure=figures/t_cyclicshift,width=70mm, height=50mm}}
  \caption{{\small {\it An impulse
      $f\in \CA$ and a few abelian group translates, $x^2f, x^9f,
      x^{-7}f$. }}}
  \label{fig:cyclicshift}
\end{figure}




\input{ideals}

%=======================================================================
\ismasec{Nonabelian Group DSP}
%=======================================================================
This section presents some basic theory of digital 
signal processing (DSP), but relies on a more general 
mathematical formalism than that employed by the 
standard textbooks on the subject.\footnote{A few notable
  exceptions are 
  \cite{{An:2003}, {Tolimieri:1998}}, {Chirikjian:2002}.}
%\cite{{An:2003},{Chirikjian:2001},{Tolimieri:1998},{Tolimieri:1997}}}.
%the book by Chirikjian and Kyatkin~  %and the books by Tolimieri and An

\input{nonabelianDSP}

\input{sdp}

%=======================================================================
\ismasec{Examples}
%=======================================================================
As seen above, when varying group structures are placed on indexing sets,
and products in the resulting group algebra are computed, 
interesting signal transforms obtain.  In this section,
we elucidate the nature of these operations by
examining some simple concrete examples in detail.

\input{examples}

