\ismasubsec{Semidirect Product Example\protect\footnotemark}
\footnotetext{An and Tolimieri (2003), page 125.}
Let $G_2$ be the \emph{dihedral group} %$\vs{D}_{2N}(x, k_{N-1})$,
%and suppose the elements of $G_2$ are indexed as follows:
with elements
\begin{eqnarray*}
G_2&=& C_N(x) \sdp \{1, k_{N-1}\}\\
&=& \{x^n k_{N-1}^j : 0 \leq n < N, 0 \leq j < 2\}.
\end{eqnarray*}
We order the elements of $G_2$ as follows:
\[
\{1, x, \ldots, x^{N-1}, k_{N-1}, xk_{N-1}, \ldots, x^{N-1}
k_{N-1}\}
\]
Thus, $G_2$ is divided into two blocks of $N$-samples.

By describing the translations of functions in $\CG_2$, 
we will see that the nonabelian translates of
$\CG_2$ are ``intra-block time-reversal'' operations.

Multiplication on $G_2$ obeys the following relations:
\begin{equation}\label{eq:id0}
  x^N = k_{N-1}^2 = 1,
\end{equation}
\begin{equation}
  x^mk_{N-1}^{j+1} \; x^nk_{N-1}^j = 
  \begin{cases} 
    x^{m-n}, & j=0,\\
    x^{m+n}, & j=1.
  \end{cases}
\end{equation}
If $z=x^mk_{N-1}$, then $z^2=1$, thus $\inverse{z}=z$.

For $f\in \CG_2$, 
\begin{equation}\label{eq:f}
  f = \sum_n f(x^n)x^n + f(x^n k_{N-1})x^n k_{N-1}.
\end{equation}
By~(\ref{eq:id0}), the nonabelian translate $k_{N-1}f$
is given by
\[
\sum_n f(k_{N-1}x^n)x^n + f(k_{N-1}x^n k_{N-1})x^n k_{N-1}
\]
which is equivalent to 
%  &=& \sum_n f(x^{(N-1)n} k_{N-1})x^n + f(x^{(N-1)n}) x^n k_{N-1},\nonumber\\
\begin{equation}\label{eq:natran}
\sum_n f(x^{N-n} k_{N-1})x^n + f(x^{N-n}) x^n k_{N-1}.
\end{equation}
Comparing (\ref{eq:f}) and (\ref{eq:natran}), we see that
the nonabelian translate of $f\in \CG_2$ swaps the first $N$
samples of $f$ with the remaining $N$ samples, and performs
a time-reversal within each sub-block.
For a simple linear function, this special translation is
illustrated in Figure~\ref{fig:G2trans}.

\begin{figure}
\centerline{\epsfig{figure=figures/G2transV,width=80mm, height=55mm}}
\caption{{\small {\it A linear signal $f\in \CG_2$, where $N
    = 8$ (left); the element $k_{N-1} \in G_2$ (middle) --
    as an element of the group algebra, $k_{N-1}$ is the ``impulse
    function'' $g \in \CG_2$ with one nonzero coefficient,
    $g(k_{N-1}) =1$; the product $gf = k_{N-1}f$ (right) is,
    in general, the convolution product and is implemented
    by appealing to the convolution theorem and using a 
    generalized FFT algorithm.}}}
    \label{fig:G2trans}
\end{figure}



%\begin{figure}[t]
%\centerline{\epsfig{figure=figures/fline2,width=70mm}}
%\caption{{\it Figures 10.4.4--10.4.6 of An (2003)
%    re-produced with fline.m program}}  
%% (see figures 10.4.4--10.4.6 in An & Tolimieri, pages 216--218)
%\label{fig:10.4.4}
%\end{figure}


