%-----------------------------------------------------------------------
\ismasubsec{The Group Algebra $\CG$}
%-----------------------------------------------------------------------
\label{sec:groupalgebra}
The \emph{group algebra} $\CG$ is the space of all formal sums
\begin{equation}
f = \sum_{x\in G} f(x) x, \quad f(x) \in \C,
\end{equation}
with the following operations for $f, g \in \CG$:
\begin{equation}
f+g = \sum_{x\in G} (f(x) + g(x))x,
\end{equation}
\begin{equation}
\alpha f = \sum_{x\in G} (\alpha f(x)) x, 
           \quad \alpha \in \C,
\end{equation}
\begin{equation}
fg = \sum_{x\in G}\left(\sum_{y\in G} f(y)g(y^{-1}x)\right)x. 
\end{equation}

The mapping $\lt{L}(g)$ of $\CG$ defined by 
$\lt{L}(g)f = gf$
is a linear operator on the space $\CG$ called 
\emph{left multiplication by} $g$.  
Since $y\in G$ can be identified with the formal
sum $e_y \in \CG$ consisting of a single nonzero term,
\begin{equation}\label{eq:leftmult}
yf = \lt{L}(e_y)f = \sum_{x\in G}f(y^{-1}x) x.
\end{equation}
In relation to translation of $\LG$, (\ref{eq:leftmult}) is the
$\CG$ analog. 

The mapping $\varTheta: \LG \to \CG$ defined by
\begin{equation}\label{eq:iso}
\varTheta(f) = \sum_{x\in G} f(x) x, \quad f\in \LG,
\end{equation}
is an algebra isomorphism of the convolution algebra $\LG$
onto the group algebra $\CG$.  
Thus we can identify
$\varTheta(f)$ with $f$, using  context to decide whether
$f$ refers to the function in $\LG$ or the formal sum in
$\CG$.  

An important aspect of the foregoing isomorphism is the
correspondence between the translations of the spaces.
Translation of $\LG$ by $y\in G$ %$\T(y)$ 
corresponds to left multiplication of $\CG$ by $y\in G$.
%$\lt{L}(y)$ 
Convolution of $\LG$ by $f\in \LG$ corresponds to
left multiplication of $\CG$ by $f\in \CG$. 
%We state 
%these relations symbolically as follows:
%\begin{center}
%\begin{tabular}{ccc}
%  $\LG$ & $\simeq$ & $\CG$ \\
%  $\lt{T}(y)$ & $\leftrightarrow$ & $\lt{L}(y)$\\
%  $\lt{C}(f)$ & $\leftrightarrow$ & $\lt{L}(f)$
%\end{tabular}
%\end{center}

\begin{figure}
  \centerline{\epsfig{figure=figures/t_cyclicshift,width=70mm, height=50mm}}
  \caption{{\small {\it An impulse
      $f\in \CA$ and a few abelian group translates, $x^2f, x^9f,
      x^{-7}f$. }}}
  \label{fig:cyclicshift}
\end{figure}


