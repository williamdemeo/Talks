% This is the file named 'intro-short.tex'
%% @LaTeX-file{
%% author = {William DeMeo},
%% filename = {'intro-short.tex'},
%% date = {2004/05/30},
%% text = {latex file for short version of intro to isma2004 paper}
%% }

\ismasec{Introduction}
The translation-invariance of most classical signal
processing transforms and filtering operations is largely
responsible for their widespread use, and is crucial for
efficient algorithmic implementation and interpretation 
of results \cite{An:2003}. 

DSP on \emph{finite abelian groups} such as $\Z/N$ is
well understood and has great practical utility.  
Translations are defined using addition modulo $N$, and 
basic operations, including convolutions and Fourier 
expansions, are developed relative to these translations \cite{Tolimieri:1998}. 
%An
%excellent treatment that is applications oriented while
%remaining fairly abstract and general, is provided by 
%%\citeN{Tolimieri:1998}.
%Tolimieri and An in \cite{Tolimieri:1998}.
Recently, however, interest in the practical utility of
\emph{finite nonabelian groups} has grown
significantly. Although the theoretical foundations of
nonabelian groups is well established, application of the
theory to DSP has yet to  become common-place.
%; cf.~the NATO
%ASI ``Computational Non-commutative Algebras,'' Italy,
%2003. Another
A notable exception is \cite{An:2003},
which develops theory and algorithms for
indexing data with nonabelian groups, defining translations
with a non-commutative group multiply operation, and
performing typical DSP operations relative to these
translations. 
%The work %of An and Tolimieri 
%demonstrates that including nonabelian groups among the
%possible data indexing strategies significantly broadens the
%range of useful signal processing techniques.

This paper describes the use of nonabelian groups
for indexing 1-dimensional signals, and discusses 
computational advantages and insights thus gained. 
A simple but instructive class of 
nonabelian groups 
%-- the \emph{semidirect product} groups --
is examined.  When elements of such
groups are used to index the data, and standard DSP
operations are defined with respect to special group
binary operators, more general and interesting
signal transformations are possible.
%For a standard noise reduction problem, we show
%how the new framework simplifies the algorithm, while
%simultaneously generalizing it to handle more complex noise
%reduction tasks. Numerical results are provided along with
%the Matlab source code for reproducing them. 

%%=======================================================================
%\section{Nonabelian Group DSP}
%=======================================================================
\ismasubsec{Two Distinctions of Consequence}
Abelian group DSP can be completely described in terms of a
special class of signals called the \emph{characters} of the
group. (For $\Z/N$, the characters are simply the exponentials.)
Each character of an abelian group represents a one-dimensional
translation-invariant subspace, and the set of all characters
spans the space of signals indexed by the group; any such
signal can be uniquely expanded as a linear combination over
the characters.

In contrast, the characters of a nonabelian group $G$, %on the other hand,
do not determine a basis for expanding signals indexed by
$G$.  However, a basis can be constructed by extending the
characters of an abelian subgroup $A$ of $G$, and then taking
certain translations of these extensions.  Some of the
characters of $A$ cannot be extended to characters of $G$,
but only to proper subgroups of $G$.  This presents some
difficulties involving the underlying translation-invariant
subspaces, some of which are now multi-dimensional.
However, it also presents opportunities for alternative
views of local signal domain information on these
translation-invariant subspaces. 

The other abelian/nonabelian distinction of primary importance 
concerns translations defined on the group.  In the abelian group
case, translations represent simple linear shifts in space or time.
When nonabelian groups index the data, however, translations are no
longer so narrowly defined. 
