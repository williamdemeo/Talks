Underlying most digital signal processing (DSP) algorithms
is the group $\Z/N$ of integers modulo $N$, which is
taken as the data indexing set. 
%Translations are defined
%using addition modulo $N$, and DSP operations, including
%convolutions and Fourier expansions, are then developed
%relative to these translations. 
Recently, An and
Tolimieri~\cite{An:2003} considered a different class of
index set mappings, which arise when the underlying group is
nonabelian, and successfully apply them to 2D image data. 

Advantages of indexing signals with nonabelian groups
are not limited to image data, but extend to audio signals
as well.  
The present work provides an overview of DSP on finite
groups and group algebras, 
%I present the basic nonabelian group theory relevant to DSP, and define 
``generalized translations,'' and their consequences, 
``generalized convolutions.'' Thereafter, some specific,
simple examples of nonabelian-group indexing sets 
are discussed along with their practical implications.

%Thereafter I discuss implications for audio
%filtering applications, spectral analysis and synthesis.

%In the present work a standard noise reduction
%problem is examined in the nonabelian framework.  The
%result is an algorithm that is simpler yet more
%general. Numerical results are provided along with the
%Matlab source code for reproducing them.
