%---
\ismasubsec{Cyclic Groups}
%---
A group $C$ is called a \emph{cyclic group} if there exists
$x\in C$ such that every $y\in C$ has the form $y=x^n$ for
some integer $n$.  In this case, $x$ is a
\emph{generator} of $C$. 

%Cyclic groups are frequently constructed as special
%subgroups of arbitrary groups, but it is convenient to have
%notation for a cyclic group of 
%order $N$ without reference to a particular underlying group.
Let the set
\begin{equation}\label{eq:cyclicGroup}
C_N(x) = \{ x^n : 0 \leq n < N\}
\end{equation} 
denote the cyclic group of order $N$ with
generator $x$, and define binary composition by
\begin{equation}\label{eq:binarycomp}
x^m x^n = x^{m+n}, \quad 0\leq m, n < N,
\end{equation}
where $m+n$ is addition modulo $N$.  
%Then $C_N(x)$ is a cyclic group of order $N$ having generator $x$.
In $C_N(x)$, the identity element %of $C_N(x)$ 
is $x^0 = 1$, and the inverse of $x^n$ is $x^{N-n}$.

%Most of the groups discussed above have elements
%in the set $\Z/N$, and we wrote the binary composition on
%the group as addition modulo $N$.   
To say that a group is \emph{abelian} is to specify that the binary
composition of the group is commutative, in which case the
symbol $+$ is usually used to represent this operation.
For nonabelian groups, we write the binary
composition as multiplication.  Since our work involves
both abelian and nonabelian groups, it is notationally
cleaner to write the binary operations of all groups --
whether abelian or not -- as multiplications. As the
following discussion illustrates, groups such as $\Z/N$ with
addition modulo $N$ have a simple multiplicative
representation.

%~~~
\begin{example}\label{ex:ZN}
%~~~
Let $\Z/N = \{0, 1, \ldots, N-1\}$,
and let addition modulo $N$ be the binary composition on $\Z/N$.
This group is isomorphic to the cyclic group $C_N(x)$; %\ie
%\begin{eqnarray}
%\Z/N &=& \{ n : 0 \leq n < N \} \nonumber\\
%&\simeq& \{ x^n : 0 \leq n < N\}= C_N(x).\label{eq:ZNisoCN}
%\end{eqnarray}
indeed, it is by this identification %~(\ref{eq:ZNisoCN}) 
that the binary composition on $\Z/N$ can be written as multiplication. 
More precisely, uniquely identifying each $m \in \Z/N$
with $x^m \in C_N(x)$, the binary composition
$m+n$ is replaced with that of~(\ref{eq:binarycomp}).  
\end{example}

%-----------------------------------------------------------------------
\ismasubsec{Group of Units}
%-----------------------------------------------------------------------
%Multiplication modulo $N$ is a ring product on the group of
%integers $\Z/N$. 
An element $m\in \Z/N$ is called a {\it unit} if
there exists an $n\in \Z/N$ such that $mn = 1$.  The set
$U(N)$ of all units in $\Z/N$ is a group with respect to
multiplication modulo $N$, and is called the 
\emph{group of units}.%unit group} of $\Z/N$. 
The group of units can be described %characterized 
as the set of all integers $0<m<N$ such that $m$ and $N$ are
relatively prime.  
\begin{example}
For $N=8$, $U(8) = \{1, 3, 5, 7\}$.
%\begin{equation}\label{eq:unitGroup}
%U(8) = \{1, 3, 5, 7\}.
%\end{equation}
\end{example}
