%=======================================================================
%\section{Nonabelian Group DSP}
%=======================================================================
\ismasubsec{Two Distinctions of Consequence}
Abelian group DSP can be completely described in terms of a
special class of signals called the \emph{characters} of the
group. (For $\Z/N$, the characters are simply the exponentials.)
Each character of an abelian group represents a one-dimensional
translation-invariant subspace, and the set of all characters
spans the space of signals indexed by the group; any such
signal can be uniquely expanded as a linear combination over
the characters.

In contrast, the characters of a nonabelian group $G$, %on the other hand,
do not determine a basis for expanding signals indexed by
$G$.  However, a basis can be constructed by extending the
characters of an abelian subgroup $A$ of $G$, and then taking
certain translations of these extensions.  Some of the
characters of $A$ cannot be extended to characters of $G$,
but only to proper subgroups of $G$.  This presents some
difficulties involving the underlying translation-invariant
subspaces, some of which are now multi-dimensional.
However, it also presents opportunities for alternative
views of local signal domain information on these
translation-invariant subspaces. 

The other abelian/nonabelian distinction of primary importance 
concerns translations defined on the group.  In the abelian group
case, translations represent simple linear shifts in space or time.
When nonabelian groups index the data, however, translations are no
longer so narrowly defined. 
