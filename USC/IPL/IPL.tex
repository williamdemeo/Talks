% arara: pdflatex
% arara: bibtex
% arara: pdflatex
% arara: pdflatex
%\documentclass[11pt]{article}
\documentclass{article}

\usepackage{lmodern}
\usepackage{microtype}

% Package for customizing page layout
%\usepackage[letterpaper]{geometry}
\usepackage[
top    = 3cm,
bottom = 3cm,
left   = 4cm,
right  = 4cm]{geometry}
%\usepackage[letterpaper]{geometry}
% \usepackage{fullpage}

\newcommand{\etaR}{\ensuremath{\tilde{\eta}}}
\newcommand{\hto}{\ensuremath{\rightharpoonup}}
\newcommand{\bool}{\ensuremath{\mathtt{Bool}}}
\newcommand{\true}{\ensuremath{\mathtt{true}}}
\newcommand{\false}{\ensuremath{\mathtt{false}}}
\newcommand{\ar}{\ensuremath{\operatorname{ar}}}
\newcommand{\CSP}{\ensuremath{\operatorname{CSP}}}
\newcommand{\dom}{\ensuremath{\operatorname{dom}}}
\newcommand{\cod}{\ensuremath{\operatorname{cod}}}
\newcommand{\im}{\ensuremath{\operatorname{im}}}
\newcommand{\slt}{\ensuremath{\mathbf S_2}}
\newcommand{\wnu}{\acs{wnu}\xspace}
\newcommand{\tct}{\acs{tct}\xspace}
\newcommand{\ctb}{\acs{ctb}\xspace}
\newcommand{\lics}{\acs{lics}\xspace}
\newcommand{\csp}{\acs{CSP}\xspace}
%% \newcommand{\mas}{\acs{MAS}\xspace}
\newcommand{\mas}{mass\xspace}
\newcommand{\masses}{masses\xspace}
%% \newcommand{\mas}{\acs{MA}\xspace}
%% \newcommand{\masses}{\acsp{MA}\xspace}
\newcommand{\sat}{\acs{sat}\xspace}
\newcommand{\nae}{\acs{nae}\xspace}
\newcommand{\csps}{\acsp{CSP}\xspace}
\newcommand{\sd}{\acs{sd}\xspace}
\newcommand{\cib}{\acs{cib}\xspace}
\newcommand{\cibs}{\acsp{cib}\xspace}
\newcommand{\PeqNP}{\acs{PeqNP}\xspace}
\newcommand{\PneqNP}{\acs{PneqNP}\xspace}
\newcommand{\NPcomplete}{\acs{NP}-complete\xspace}
\newcommand{\NP}{\acs{NP}\xspace}
\renewcommand{\P}{\acs{P}\xspace}
\newcommand{\mfA}{\ensuremath{\mathfrak{A}}}
\newcommand{\mfX}{\ensuremath{\mathfrak{X}}}
\newcommand{\lb}{\ensuremath{\llbracket}}
\newcommand{\rb}{\ensuremath{\rrbracket}}
\newcommand{\id}{\ensuremath{\operatorname{id}}}
\newcommand{\fin}{\ensuremath{\operatorname{fin}}}
\newcommand{\Eq}{\ensuremath{\operatorname{Eq}}}
\newcommand{\Rel}{\ensuremath{\operatorname{Rel}}}
\newcommand{\Proj}{\ensuremath{\operatorname{Proj}}}
\newcommand{\arity}{\ensuremath{\operatorname{arity}}}
\newcommand{\Op}{\ensuremath{\operatorname{Op}}}
\newcommand{\alg}[1]{\ensuremath{\mathbf{#1}}}
\newcommand{\Sub}{\ensuremath{\operatorname{Sub}}}
\newcommand{\Con}{\ensuremath{\operatorname{Con}}}
\newcommand{\Cg}{\ensuremath{\operatorname{Cg}}}
\newcommand{\Clo}{\ensuremath{\operatorname{\mathsf{Clo}}}}
\newcommand{\typ}{\ensuremath{\operatorname{typ}}}
\newcommand{\Pol}{\ensuremath{\operatorname{Pol}}}
\newcommand{\Poly}{\ensuremath{\operatorname{Poly}}}
%% \renewcommand{\V}{\ensuremath{\operatorname{V}}}
\newcommand{\sdp}{\ensuremath{\leq_{\mathrm{sd}}}}
%\newcommand{\defn}[1]{{\bf #1}}
\newcommand{\defn}[1]{\textit{#1}}
\newcommand{\N}{\ensuremath{\mathbb{N}}}
\newcommand{\SDwedge}{\ensuremath{\mbox{SD}_\wedge}}
\newcommand{\<}{\ensuremath{\langle}}
\renewcommand{\>}{\ensuremath{\rangle}}
\newcommand{\bR}{\ensuremath{\mathbf{R}}}
\newcommand{\bn}{\ensuremath{\mathbf{n}}}
\newcommand{\sansA}{\ensuremath{\mathsf{A}}}
\newcommand{\sansComp}{\ensuremath{\mathsf{comp}}}
\newcommand{\sansC}{\ensuremath{\mathsf{C}}}
\newcommand{\sansH}{\ensuremath{\mathsf{H}}}
\newcommand{\sansO}{\ensuremath{\mathsf{O}}} % ALL FINITARY OPERATIONS
\newcommand{\sansP}{\ensuremath{\mathsf{P}}}
\newcommand{\sansPoly}{\ensuremath{\mathsf{Pol}}}
\newcommand{\sansPol}{\ensuremath{\mathsf{Pol}}}
\newcommand{\sansRel}{\ensuremath{\mathsf{Inv}}}
\newcommand{\sansInv}{\ensuremath{\mathsf{Inv}}}
\newcommand{\sansClo}{\ensuremath{\operatorname{\mathsf{Clo}}}}
\newcommand{\sansR}{\ensuremath{\mathsf{R}}} % ALL FINITARY RELATIONS
\newcommand{\sansS}{\ensuremath{\mathsf{S}}}
\newcommand{\sansT}{\ensuremath{\mathsf{T}}}
\newcommand{\bA}{\ensuremath{\mathbf{A}}}
\newcommand{\bQ}{\ensuremath{\mathbf{Q}}}
\newcommand{\bT}{\ensuremath{\mathbf{T}}}
\newcommand{\ba}{\ensuremath{\mathbf{a}}}
\newcommand{\bb}{\ensuremath{\mathbf{b}}}
\newcommand{\bt}{\ensuremath{\mathbf{t}}}
\newcommand{\bu}{\ensuremath{\mathbf{u}}}
\newcommand{\bv}{\ensuremath{\mathbf{v}}}
\newcommand{\bw}{\ensuremath{\mathbf{w}}}
\newcommand{\vf}{\ensuremath{\mathbf{f}}}
\newcommand{\onlyif}{\ensuremath{\Longrightarrow}}
\newcommand{\bD}{\ensuremath{\mathbf{D}}}
\newcommand{\bF}{\ensuremath{\mathbf{F}}}
\newcommand{\bP}{\ensuremath{\mathbf{P}}}
\newcommand{\br}{\ensuremath{\mathbf{r}}}
\newcommand\ubA{\ensuremath{\underline{\mathbf{A}}}}
\newcommand\uA{\ensuremath{\underline{A}}}
\newcommand{\bbA}{\ensuremath{\mathbb{A}}}
\newcommand{\bbB}{\ensuremath{\mathbb{B}}}
\newcommand{\bbC}{\ensuremath{\mathbb{C}}}
\newcommand{\bbD}{\ensuremath{\mathbb{D}}}
\newcommand{\bbS}{\ensuremath{\mathbb{S}}}
\newcommand{\bbV}{\ensuremath{\mathbb{V}}}
\newcommand{\bs}{\ensuremath{\mathbf{s}}}
\newcommand{\sA}{\ensuremath{\mathcal{A}}}
\newcommand{\sV}{\ensuremath{\mathcal{V}}}
\newcommand{\sB}{\ensuremath{\mathcal{B}}}
\newcommand{\sS}{\ensuremath{\mathcal{S}}}
\newcommand{\bB}{\ensuremath{\mathbf{B}}}
\newcommand{\bc}{\ensuremath{\mathbf{c}}}

\newcommand{\nn}{\ensuremath{\underline{n}}}
\newcommand{\mm}{\ensuremath{\underline{m}}}
\newcommand{\pp}{\ensuremath{\underline{p}}}
\newcommand{\kk}{\ensuremath{\underline{k}}}
\newcommand{\ul}{\ensuremath{\underline{\ell}}}
%% \newcommand{\vzero}{\ensuremath{\mathbf{0}}}
\newcommand{\vzero}{\ensuremath{\eta}}
\newcommand{\uzero}{\ensuremath{\underline{0}}}
\newcommand{\uone}{\ensuremath{\underline{1}}}
\newcommand{\utwo}{\ensuremath{\underline{2}}}
\newcommand{\uthree}{\ensuremath{\underline{3}}}

\newcommand{\bC}{\ensuremath{\mathbf{C}}}
\newcommand{\cC}{\ensuremath{\mathcal{C}}}
\newcommand{\cS}{\ensuremath{\mathcal{S}}}
\newcommand{\sC}{\ensuremath{\mathcal{C}}}
\newcommand{\sI}{\ensuremath{\mathcal{I}}}
\newcommand{\sM}{\ensuremath{\mathcal{M}}}
\newcommand{\sW}{\ensuremath{\mathcal{W}}}
\newcommand{\sR}{\ensuremath{\mathcal{R}}}
\newcommand{\sF}{\ensuremath{\mathcal{F}}}
\newcommand{\bS}{\ensuremath{\mathbf{S}}}
\newcommand{\bx}{\ensuremath{\mathbf{x}}}
\newcommand{\by}{\ensuremath{\mathbf{y}}}
\newcommand{\bz}{\ensuremath{\mathbf{z}}}
\newcommand{\power}[1]{\ensuremath{\mathscr{P}(#1)}}
%% \newcommand{\sP}{\ensuremath{\power{P}}}
\newcommand{\sP}{\ensuremath{\mathcal{P}}}
\newcommand{\balpha}{\ensuremath{\boldsymbol{\alpha}}}
\newcommand{\bbeta}{\ensuremath{\boldsymbol{\beta}}}
\newcommand{\bgamma}{\ensuremath{\boldsymbol{\gamma}}}
\newcommand{\bdelta}{\ensuremath{\boldsymbol{\delta}}}
\newcommand{\bepsilon}{\ensuremath{\boldsymbol{\epsilon}}}
\newcommand{\AAn}{\ensuremath{A^{(A^n)}}}
\newcommand{\AAm}{\ensuremath{A^{(A^m)}}}
\newcommand{\AAk}{\ensuremath{A^{(A^k)}}}
\newcommand{\meet}{\ensuremath{\wedge}}
\newcommand{\Meet}{\ensuremath{\bigwedge}}
\renewcommand{\Join}{\ensuremath{\bigvee}}
\newcommand{\join}{\ensuremath{\vee}}
\newcommand{\onto}{\ensuremath{\twoheadrightarrow}}
\newcommand{\into}{\ensuremath{\rightarrowtail}}
\newcommand{\fst}{\ensuremath{\operatorname{fst}}}
\newcommand{\snd}{\ensuremath{\operatorname{snd}}}
\newcommand{\absorbing}{\ensuremath{\mathrel{\triangleleft}}}
\newcommand{\minabsorbing}{\ensuremath{\mathrel{\triangleleft\triangleleft}}}
%% \newcommand{\myprod}{\ensuremath{\prod}}
\newcommand{\myprod}{\ensuremath{\Pi}}

%% \newcommand\map{\ensuremath{\operatorname{map}}}
\newcommand\map{\ensuremath{map}}
\def\makecs#1#2{\makecsX {#1}#2,.}
\def\makecsX#1#2#3.{\onecs{#1}{#2} \ifx#3,\let\next\eatit
   \else\let\next\makecsX\fi 
   \next{#1}#3.}
\def\onecs#1#2{\expandafter\gdef\csname #2\endcsname{#1{#2}}}
\def\eatit#1#2.{\relax}


\usepackage{proof-dashed}
\usepackage{tikz-cd}

\metadata{IPL Notes}{Feb 28, 2014}
\begin{document}
This note summarizes the rules of inference for \ac{IPL} as I understand them.
This is based on the lectures given by Professor Robert Harper in September
2013 at CMU~\cite{Harper2012}. 
Notes for Harper's lectures were transcribed by his students and
this summary is based on the recorded lectures and the notes. 

As advanced by Per Martin-L\"{o}f, a modern presentation of \ac{IPL}
distinguishes the notions of \vocab{judgment} and \vocab{proposition}. A
judgment is something that may be known, whereas a proposition is something that
sensibly may be the subject of a judgment. For instance, the statement ``Every
natural number larger than $1$ is either prime or can be uniquely factored into
a product of primes\@.'' is a proposition because it sensibly may be subject to
judgment. That the statement is in fact true is a judgment.
Only with a proof, however, is it evident that the judgment indeed holds.

Thus, in \ac{IPL}, the two most basic judgments are $A \prop$ and $A \true$:
\begin{alignat*}{2}
  A \prop &&\quad& \text{$A$ is a well-formed proposition} \\
  A \true &&& \text{\begin{tabular}[t]{@{}l@{}}
                Proposition $A$ is intuitionistically true, i.e., has a proof.
              \end{tabular}}
\end{alignat*}
The inference rules for the $\prop$ judgment are called \vocab{formation rules}.
The inference rules for the $\true$ judgment are divided into classes:
\vocab{introduction rules} and \vocab{elimination rules}. 

The meaning of a proposition $A$ is given by the introduction rules for the
judgment $A \true$. The elimination rules are dual and  describe what may be
deduced from a proof of $A \true$.  The principle of \vocab{internal coherence},
also known as \emph{Gentzen's principle of inversion}, is that the introduction
and elimination rules for a proposition $A$ fit together properly.  The
elimination rules should be strong enough to deduce all information that was
used to introduce $A$ (\vocab{local completeness}), but not so strong as to
deduce information that might not have been used to introduce $A$ (\vocab{local
  soundness}). 

\medskip 

\hrule

\begin{center}
CONJUNCTION
\end{center}
\begin{itemize}
\item[(formation)] 
If $A$ and $B$ are well-formed propositions, then so is
their \emph{conjunction}, which we write as $A \conj B$.
\begin{equation*}
  \infer[{\conj}\mathsf{F}]{A \conj B \prop}{
    A \prop & B \prop}
\end{equation*}

\item[(introduction)]
%\paragraph{Introduction.}
To give meaning to conjunction, we must say how
to introduce the judgment $A \conj B \true$.
A verification of $A \conj B$ requires a proof of $A$ and
a proof of $B$.
\begin{equation*}
  \infer[{\conj}\mathsf{I}]{A \conj B \true}{
    A \true & B \true}
\end{equation*}

\item[(elimination)]
%\paragraph{Elimination.}
Because every proof of $A \conj B$ comes from a pair of proofs, one of $A$ and
one of $B$, we are justified in deducing $A \true$ and $B \true$ from a proof of
$A \conj B$: 
\begin{mathpar}
  \infer[{\conj}\mathsf{E}_1]{A \true}{
    A \conj B \true}
  \and
  \infer[{\conj}\mathsf{E}_2]{B \true}{
    A \conj B \true}
\end{mathpar}
\end{itemize}

\medskip 

\hrule
%\newpage
\begin{center}
TRUTH
 \end{center}
%We denote by $\truth$ the \emph{trivially true} proposition. 
\begin{itemize}
\item[(formation)]
The formation rule serves as immediate evidence for the judgment that $\truth$ is a
well-formed proposition.
\begin{equation*}
  \infer[{\truth}\mathsf{F}]{\truth \prop}{
    }
\end{equation*}

\item[(introduction)]
Since $\truth$ is a trivially true proposition, its introduction rule makes the
judgment $\truth \true$ immediately evident.

\begin{equation*}
  \infer[{\truth}\mathsf{I}]{\truth \true}{
    }
\end{equation*}

\item[(elimination)]
Since $\truth$ is trivially true, an elimination rule should not increase
our knowledge---we put in no information when we introduced $\truth \true$, so,
by the principle of conservation of proof, we should get no information out. Thus,
there is no elimination rule for $\truth$.
\end{itemize}

%% \medskip 

%% \hrule
\newpage

\begin{center}
ENTAILMENT
 \end{center}
\emph{Entailment} is a judgment and is written as 
\begin{equation*}
  A_1 \true, \dotsc, A_n \true \entails A \true
\end{equation*}
This expresses the judgment that $A \true$ follows from 
$A_1 \true, \dotsc, A_n \true$. 
One can view $A_1 \true, \dotsc, A_n \true$ as being assumptions from which
the conclusion $A \true$ may be deduced. 
We assume that the entailment judgment satisfies several \emph{structural
  properties}: reflexivity, transitivity, weakening, contraction, and
permutation. 
\begin{itemize}
\item[Reflexivity:] An assumption is enough to conclude the same judgment.
\begin{equation*}
  \infer[\text{\textsf{R}}]{A \true \entails A \true}{
    }
\end{equation*}

\item[Transitivity:]
If you prove $A \true$, then you are justified in using it in a proof.
\begin{equation*}
  \infer[\text{\textsf{T}}]{C \true}{
    A \true &
    A \true \entails C \true}
\end{equation*}

Reflexivity and transitivity are undeniable since without them it would be
unclear what is meant by an \emph{assumption}.  An assumption should be strong enough
to prove conclusions (reflexivity), and only as strong as the proofs they stand for
(transitivity). 
The remaining structural properties---weakening, contraction, and
permutation---could be denied.  Logics that deny any of these properties are
called \emph{substructural logics}. 

\item[Weakening:]
We can add assumptions to a proof without invalidating that proof.
\begin{equation*}
  \infer[\text{\textsf{W}}]{B \true \entails A \true}{
    A \true}
\end{equation*}
\item[Contraction:]
The number of copies of an assumption does not matter.
\begin{equation*}
  \infer[\text{\textsf{C}}]{A \true \entails C \true}{
    A \true, A \true \entails C \true}
\end{equation*}
\item[Permutation:]
aka ``exchange;'' the order of assumptions does not matter.
\begin{equation*}
  \infer[\text{\textsf{P}}]{\pi(\ctx) \entails C \true}{
    \ctx \entails C \true}
\end{equation*}
\end{itemize}


\medskip 

\hrule

\begin{center}
IMPLICATION
\end{center}
\begin{itemize}
\item[(formation)] 
\begin{equation*}
  \infer[{\imp}\mathsf{F}]{A \imp B \prop}{
    A \prop & B \prop}
\end{equation*}
\item[(introduction)]
\begin{equation*}
  \infer[{\imp}\mathsf{I}]{A \imp B \true}{
    A \true \entails B \true}
\end{equation*}
In this way, implication internalizes the entailment judgment as a proposition,
while we nonetheless maintain the distinction between propositions and
judgments.
%% (As an aside for those familiar with category theory, the relationship between
%% entailment and implication is analogous to the relationship between a mapping
%% and a collection of mappings internalized as an object in some category.) 
\item[(elimination)]
\begin{equation*}
  \infer[{\imp}\mathsf{E}]{B \true}{
    A \imp B \true & A \true} \,.
\end{equation*}
This rule is sometimes referred to as \latin{modus ponens}.
\end{itemize}

%% \medskip

%% \hrule
\newpage
\begin{center}
DISJUNCTION
\end{center}
\begin{itemize}
\item[(formation)]
\begin{equation*}
  \infer[{\disj}\mathsf{F}]{A \disj B \prop}{
    A \prop & B \prop}
\end{equation*}
\item[(introduction)]
\begin{mathpar}
  \infer[{\disj}\mathsf{I_1}]{A \disj B \true}{
    A \true}
  \and
  \infer[{\disj}\mathsf{I_2}]{A \disj B \true}{
    B \true}
\end{mathpar}
\item[(elimination)]
\begin{equation*}
  \infer[{\disj}\mathsf{E}]{C \true}{
    A \disj B \true &
    A \true \entails C \true & B \true \entails C \true}
\end{equation*}
\end{itemize}

\medskip 

\hrule
\begin{center}
FALSEHOOD
\end{center}
\begin{itemize}
\item[(formation)]
The unit of disjunction is falsehood, the proposition that is trivially never
true, which we write as $\falsehood$.  Its formation rule is immediate evidence
that $\falsehood$ is a well-formed proposition. 
\begin{equation*}
  \infer[{\falsehood}\mathsf{F}]{\falsehood \prop}{
    }
\end{equation*}
\item[(introduction)]
Because $\falsehood$ should never be true, it has no introduction rule.
\item[(elimination)]
\begin{equation*}
  \infer[{\falsehood}\mathsf{E}]{C \true}{
    \falsehood \true}
\end{equation*}
The elimination rule captures \latin{ex falso quodlibet}: from a proof of $\falsehood \true$, we may deduce that \emph{any} proposition $C$ is true because there is ultimately no way to introduce $\falsehood \true$.
Once again, the rules cohere.
The elimination rule is very strong, but remains justified due to the absence of any introduction rule for falsehood.
\end{itemize}

\medskip 

\hrule
%% \begin{center}
%% NEGATION
%% \end{center}
%% \begin{itemize}
%% \item[(formation)]
%% \item[(introduction)]
%% \item[(elimination)]
%% \end{itemize}

%% \nocite{Pfenning2009a, Pfenning2009b}
\begin{thebibliography}{1}


\bibitem{Harper2012}
Robert Harper.
\newblock Carnegie Mellon University course: 15-819 Homotopy Type Theory.
\newblock
  \url{http://www.cs.cmu.edu/~rwh/courses/hott/},
  Fall 2012.

\bibitem{Pfenning2009b}
Frank Pfenning.
\newblock Lecture notes on harmony.
\newblock
  \url{http://www.cs.cmu.edu/~fp/courses/15317-f09/lectures/03-harmony.pdf},
  September 2009.

\bibitem{Pfenning2009a}
Frank Pfenning.
\newblock Lecture notes on natural deduction.
\newblock
  \url{http://www.cs.cmu.edu/~fp/courses/15317-f09/lectures/02-natded.pdf},
  August 2009.

\end{thebibliography}

%% \bibliographystyle{plain}
%% \bibliography{hott_references}

\end{document}
