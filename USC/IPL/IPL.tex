% IPL.tex
% Notes on rules of inference for IPL.
% Author: William DeMeo <williamdemeo@gmail.com>
% Date: 28 Feb 2014
% Updated: 30 Jan 2018
% Copyright: William DeMeo 2014

\documentclass{article}
%% If you want bigger font, replace the line above with
%%     \documentclass[12pt]{article}  


%%%%%%%%%%%%%%%%%%%%%%%%%%%%%%%%%%%%%%%%
% Basic packages
%%%%%%%%%%%%%%%%%%%%%%%%%%%%%%%%%%%%%%%%
\usepackage{amsmath,amsthm,amssymb,mathtools}
\usepackage{etoolbox}
\usepackage{mathpartir}
\usepackage{xcolor}
\usepackage{xspace}
\usepackage{url}



%%%%%%%%%%%%%%%%%%%%%%%%%%%%%%%%%%%%%%%%
% Acronyms -- add your own if you want!
%%%%%%%%%%%%%%%%%%%%%%%%%%%%%%%%%%%%%%%%
\usepackage[acronym, shortcuts]{glossaries}
\newacronym{HoTT}{HoTT}{homotopy type theory}
\newacronym{IPL}{IPL}{intuitionistic propositional logic}
\newacronym{TT}{TT}{intuitionistic type theory}
\newacronym{LEM}{LEM}{law of the excluded middle}
\newacronym{ITT}{ITT}{intensional type theory}
\newacronym{ETT}{ETT}{extensional type theory}
\newacronym{NNO}{NNO}{natural numbers object}
\robustify{\ac}
%% Use acronyms like this: \ac{IPL}
%% If the acronym has already been used, then \ac{IPL} just prints IPL; 
%% However, if it's the first occurrence, then we'll get:
%%   intuitionistic propositional logic (IPL)


\newrobustcmd*{\vocab}[1]{\emph{#1}}
\newrobustcmd*{\latin}[1]{\textit{#1}}


%%%%%%%%%%%%%%%%%%%%%%%%%%%%%%%%%%%%%%%%%%%%%%%%%%%%%%%%%%%%%%%%%%%
% `comment` is a package that enables you to omit sections by simplly
% enclosing the unwanted stuff inside `\begin{comment}  \end{comment}`
\usepackage{comment}  % This package lets you omit sections of code


%%%%%%%%%%%%%%%%%%%%%%%%%%%%%%%%%%%%%%%%%%%%%%%%%%%%%%%%%%%%%%%%%%%
% `hyperref` is a package that gives you hyperlinks in your pdf,
% which can be very helpful for those reading it on a computer
% or tablet.  You get the default color options for links with
%
%          \usepackage{hyperref}
%
% but I think the default colors are VERY ugly and distracting,
% so I use the following options:
\usepackage[
  colorlinks=true,
  urlcolor=black,
  linkcolor=black,
  citecolor=black
]{hyperref}


%%%%%%%%%%%%%%%%%%%%%%%%%%%%%%%%%%%%%%%%%%%%%%%%%%%%%%%%%%%%%%%%%%%%
%
% The `geometry.sty` package is for customizing page layout
%
\usepackage[  
  top    = 3cm,  %% <<== adjust top margin by changing this number
  left   = 4cm,  %% <<== adjust left margin by changing this number
  bottom = 3cm,  %% <<==  ...etc...
  right  = 4cm
]{geometry}


%% The next line incorporates some macros employed by Bob's students.
\usepackage{cmu-macros}  %% see the `cmu-macros.sty` file for details.


%%%%%%%%%%%%%%%%%%%%%%%%%%%%%%%%%%%%%%%%%%%%%%%%%%%%%%%%%%%%%%%%%%%%
%
% The `proof-dashed.sty` package is for typesetting inference rules
%
\usepackage{proof-dashed}
%
% Look at the comments in the file `proof-dashed.sty` for more details
% about how to use it, but looking at this one example may be sufficient.
% (Notice you start with the base of the tree and work up.)
%
% Example: The LaTeX commands
%
%		\infer[R2]{ E }
%                     { A & \infer[R1]{ D }
%                                   { B & C } }
%
%                produce following derivation tree
%
%                       B C
%                      ----- R1
%                  A     D
%                 ---------- R2
%                     E
%
% Note: spacing doesn't matter, so you could write the above derivation
% like this:  \infer[R2]{E}{A & \infer[R1]{D}{B & C}}


%%%%%%%%%%%%%%%%%%%%%%%%%%%%%%%%%%%%%%%%%%%%%%%%%%%%%%%%%%%%%%%%%%%%
% `tikz` is a package for making diagrams
% `tikz-cd` has additional support for commutative diagrams
%  We don't need it for this document, but if you will add diagrams,
%  be sure you have the `tikz-cd.sty` file in latex search path,
%  and then uncomment the next line.
%% \usepackage{tikz-cd}  


%%%%%%%%%%%%%%%%%%%%%%%%%%%%%%%%%%%%%%%%%%%%%%%%%%%%%%%%%%%%%%%%%%%%
% `fancy` is a package for making ``fancy'' headers and footers.
%
% If you want fancy headers and footers, then uncomment the next three lines.
%
%%   \pagestyle{fancy}
%%   \lhead{Democrat}   \chead{Moderate}   \rhead{Republican}
%%   \lfoot{}  \cfoot{\thepage}   \rfoot{}
%    \renewcommand{\headrulewidth}{0.4pt}



%%%%%%%%%%%%%%%%%%%%%%%%%%%%%%%%%%%%%%%%%%%%%%%%%%%%%%%%%%%%%%%%%%%%%%%%%%%%%%%
%=============================================================================%
%%%%%%%%%%%%%%%%%%%%%%%%%%%%%%%%%%%%%%%%%%%%%%%%%%%%%%%%%%%%%%%%%%%%%%%%%%%%%%%

\title{Notes on Intuitionistic Proposition Logic}
\author{William DeMeo\\
  \url{williamdemeo@gmail.com}}
\date{28 Feb 2014}


\begin{document}  %% <<== Here's where the document actually begins!

\maketitle

\section{Introduction}
These notes summarize the rules of inference for \ac{IPL} as I understand them.
This is based on the lectures given by Professor Robert Harper in September
2013 at CMU~\cite{Harper2012}. 
Notes for Harper's lectures were transcribed by his students and
this summary is based on the recorded lectures and the notes. 

As advanced by Per Martin-L\"{o}f, a modern presentation of \ac{IPL}
distinguishes the notions of \vocab{judgment} and \vocab{proposition}. A
judgment is something that may be known, whereas a proposition is something that
sensibly may be the subject of a judgment. For instance, the statement ``Every
natural number larger than $1$ is either prime or can be uniquely factored into
a product of primes\@.'' is a proposition because it sensibly may be subject to
judgment. That the statement is in fact true is a judgment.
Only with a proof, however, is it evident that the judgment indeed holds.

Thus, in \ac{IPL}, the two most basic judgments are $A \prop$ and $A \true$:
\begin{alignat*}{2}
  A \prop &&\quad& \text{$A$ is a well-formed proposition} \\
  A \true &&& \text{\begin{tabular}[t]{@{}l@{}}
                Proposition $A$ is intuitionistically true, i.e., has a proof.
              \end{tabular}}
\end{alignat*}
The inference rules for the $\prop$ judgment are called \vocab{formation rules}.
The inference rules for the $\true$ judgment are divided into classes:
\vocab{introduction rules} and \vocab{elimination rules}. 

The meaning of a proposition $A$ is given by the introduction rules for the
judgment $A \true$. The elimination rules are dual and  describe what may be
deduced from a proof of $A \true$.  The principle of \vocab{internal coherence},
also known as \emph{Gentzen's principle of inversion}, is that the introduction
and elimination rules for a proposition $A$ fit together properly.  The
elimination rules should be strong enough to deduce all information that was
used to introduce $A$ (\vocab{local completeness}), but not so strong as to
deduce information that might not have been used to introduce $A$ (\vocab{local
  soundness}). 


\section{Conjunction}
We begin with \emph{conjunction}, which is interpreted semantically as ``logical and.''
The formation, introduction, and elimination rules for conjunction are as follows.

\begin{itemize}
\item[(formation)] 
If $A$ and $B$ are well-formed propositions, then so is
their \emph{conjunction}, which we write as $A \conj B$.
\begin{equation*}
  \infer[{\conj}\mathsf{F}]{A \conj B \prop}{
    A \prop & B \prop}
\end{equation*}

\item[(introduction)]
To give meaning to conjunction, we must say how
to introduce the judgment $A \conj B \true$.
A verification of $A \conj B$ requires a proof of $A$ and
a proof of $B$.
\begin{equation*}
  \infer[{\conj}\mathsf{I}]{A \conj B \true}{
    A \true & B \true}
\end{equation*}

\item[(elimination)]
Because every proof of $A \conj B$ comes from a pair of proofs, one of $A$ and
one of $B$, we are justified in deducing $A \true$ and $B \true$ from a proof of
$A \conj B$: 
\begin{mathpar}
  \infer[{\conj}\mathsf{E}_1]{A \true}{
    A \conj B \true}
  \and
  \infer[{\conj}\mathsf{E}_2]{B \true}{
    A \conj B \true}
\end{mathpar}
\end{itemize}

%\newpage

\section{Truth}

\begin{itemize}
\item[(formation)]
The formation rule serves as immediate evidence for the judgment that $\truth$ is a
well-formed proposition.
\begin{equation*}
  \infer[{\truth}\mathsf{F}]{\truth \prop}{
    }
\end{equation*}

\item[(introduction)]
Since $\truth$ is a trivially true proposition, its introduction rule makes the
judgment $\truth \true$ immediately evident.

\begin{equation*}
  \infer[{\truth}\mathsf{I}]{\truth \true}{
    }
\end{equation*}

\item[(elimination)]
Since $\truth$ is trivially true, an elimination rule should not increase
our knowledge---we put in no information when we introduced $\truth \true$, so,
by the principle of conservation of proof, we should get no information out. Thus,
there is no elimination rule for $\truth$.
\end{itemize}

%% \newpage

\section{Entailment}

\emph{Entailment} is a judgment and is written as 
\begin{equation*}
  A_1 \true, \dotsc, A_n \true \entails A \true
\end{equation*}
This expresses the judgment that $A \true$ follows from 
$A_1 \true, \dotsc, A_n \true$. 
One can view $A_1 \true, \dotsc, A_n \true$ as being assumptions from which
the conclusion $A \true$ may be deduced. 
We assume that the entailment judgment satisfies several \emph{structural
  properties}: reflexivity, transitivity, weakening, contraction, and
permutation. 
\begin{itemize}
\item[Reflexivity:] An assumption is enough to conclude the same judgment.
\begin{equation*}
  \infer[\text{\textsf{R}}]{A \true \entails A \true}{
    }
\end{equation*}

\item[Transitivity:]
If you prove $A \true$, then you are justified in using it in a proof.
\begin{equation*}
  \infer[\text{\textsf{T}}]{C \true}{
    A \true &
    A \true \entails C \true}
\end{equation*}

Reflexivity and transitivity are undeniable since without them it would be
unclear what is meant by an \emph{assumption}.  An assumption should be strong enough
to prove conclusions (reflexivity), and only as strong as the proofs they stand for
(transitivity). 
The remaining structural properties---weakening, contraction, and
permutation---could be denied.  Logics that deny any of these properties are
called \emph{substructural logics}. 

\item[Weakening:]
We can add assumptions to a proof without invalidating that proof.
\begin{equation*}
  \infer[\text{\textsf{W}}]{B \true \entails A \true}{
    A \true}
\end{equation*}
\item[Contraction:]
The number of copies of an assumption does not matter.
\begin{equation*}
  \infer[\text{\textsf{C}}]{A \true \entails C \true}{
    A \true, A \true \entails C \true}
\end{equation*}
\item[Permutation:]
aka ``exchange;'' the order of assumptions does not matter.
\begin{equation*}
  \infer[\text{\textsf{P}}]{\pi(\ctx) \entails C \true}{
    \ctx \entails C \true}
\end{equation*}
\end{itemize}


\section{Implication}

\begin{itemize}
\item[(formation)] 
\begin{equation*}
  \infer[{\imp}\mathsf{F}]{A \imp B \prop}{
    A \prop & B \prop}
\end{equation*}
\item[(introduction)]
\begin{equation*}
  \infer[{\imp}\mathsf{I}]{A \imp B \true}{
    A \true \entails B \true}
\end{equation*}
In this way, implication internalizes the entailment judgment as a proposition,
while we nonetheless maintain the distinction between propositions and
judgments.
%% (As an aside for those familiar with category theory, the relationship between
%% entailment and implication is analogous to the relationship between a mapping
%% and a collection of mappings internalized as an object in some category.) 
\item[(elimination)]
\begin{equation*}
  \infer[{\imp}\mathsf{E}]{B \true}{
    A \imp B \true & A \true} \,.
\end{equation*}
This rule is sometimes referred to as \latin{modus ponens}.
\end{itemize}

%% \newpage

\section{Disjunction}
\begin{itemize}
\item[(formation)]
\begin{equation*}
  \infer[{\disj}\mathsf{F}]{A \disj B \prop}{
    A \prop & B \prop}
\end{equation*}
\item[(introduction)]
\begin{mathpar}
  \infer[{\disj}\mathsf{I_1}]{A \disj B \true}{
    A \true}
  \and
  \infer[{\disj}\mathsf{I_2}]{A \disj B \true}{
    B \true}
\end{mathpar}
\item[(elimination)]
\begin{equation*}
  \infer[{\disj}\mathsf{E}]{C \true}{
    A \disj B \true &
    A \true \entails C \true & B \true \entails C \true}
\end{equation*}
\end{itemize}


\section{Falsehood}

\begin{itemize}
\item[(formation)]
The unit of disjunction is falsehood, the proposition that is trivially never
true, which we write as $\falsehood$.  Its formation rule is immediate evidence
that $\falsehood$ is a well-formed proposition. 
\begin{equation*}
  \infer[{\falsehood}\mathsf{F}]{\falsehood \prop}{
    }
\end{equation*}
\item[(introduction)]
Because $\falsehood$ should never be true, it has no introduction rule.
\item[(elimination)]
\begin{equation*}
  \infer[{\falsehood}\mathsf{E}]{C \true}{
    \falsehood \true}
\end{equation*}
The elimination rule captures \latin{ex falso quodlibet}:
from a proof of $\falsehood \true$, we may deduce that \emph{any} proposition $C$
is true because there is ultimately no way to introduce $\falsehood \true$.
Once again, the rules cohere. The elimination rule is very strong, but remains
justified due to the absence of any introduction rule for falsehood.
\end{itemize}


\begin{thebibliography}{1}


\bibitem{Harper2012}
Robert Harper.
\newblock Carnegie Mellon University course: 15-819 Homotopy Type Theory.
\newblock
  \url{http://www.cs.cmu.edu/~rwh/courses/hott/},
  Fall 2012.

\bibitem{Pfenning2009b}
Frank Pfenning.
\newblock Lecture notes on harmony.
\newblock
  \url{http://www.cs.cmu.edu/~fp/courses/15317-f09/lectures/03-harmony.pdf},
  September 2009.

\bibitem{Pfenning2009a}
Frank Pfenning.
\newblock Lecture notes on natural deduction.
\newblock
  \url{http://www.cs.cmu.edu/~fp/courses/15317-f09/lectures/02-natded.pdf},
  August 2009.

\end{thebibliography}

\end{document}

%%%%%%%%%%%%%%%%%%%%%%%%%%%%%%%%%%%%%%%%%%%%%%%%%%%%%%%%%%%%%%%%%%%%%%%%%%%%%%%
%=============================================================================%
%%%%%%%%%%%%%%%%%%%%%%%%%%%%%%%%%%%%%%%%%%%%%%%%%%%%%%%%%%%%%%%%%%%%%%%%%%%%%%%
