\begin{frame}[fragile,label=P5Lemma]{The $P^5$ Lemma}
\begin{lemma}[\Palfy-\Pudlak, 1980]
 Let  $\bA = \<A, F\>$ be  a  unary  algebra  where  $F$  is  a  monoid.\\[5pt]
 Suppose $e \in F$ satisfies $e\circ e = e$.\\[5pt]
 Define  $\bB = \<B, G\>$
 \[
 B=e(A) \quad \text{ and } \quad  
 G = \{e\,f\resB \mid f\in F\}.
 \]
 Let $\resB :  Con (\bA)\rightarrow Con (\bB)$  be the restriction mapping:
\[
\theta \resB = \theta \cap B^2
\]
Then  $\resB$  is  a  surjective  homomorphism  
(even  for  arbitrary meets  and  joins). 
\end{lemma}

\vskip1cm

{\small P{\'e}ter P{\'a}l \Palfy\ and Pavel Pudl{\'a}k: {\it Congruence lattices of finite algebras} AU (1980).}
%% {\small P{\'e}ter P{\'a}l \Palfy\ and Pavel Pudl{\'a}k: {\it Congruence lattices of finite algebras\\ and
%%   intervals in subgroup lattices of finite groups.}\\[4pt]
%%  Algebra Universalis \textbf{11}(1), 22--27 (1980).}

\note{The restriction mapping $\resB$, defined on $\Con\bA$ by 
$\alpha\resB = \alpha \cap B^2$, is a lattice epimorphism of $\Con\bA$ onto $\Con\bB$.}
\end{frame}



%%%%%%%%%%%%%%%%%%%%%%%%%%%
\begin{frame}[fragile,label=P5Lemma]{The Star Map and Hat Map}
  \begin{columns}
    \begin{column}{0.3\textwidth}
      \hskip1cm \includegraphics[height=3cm]{inputs/amconfus.png}
    \end{column}
    \begin{column}{0.3\textwidth}
      \includegraphics[height=2cm]{inputs/starchart}
      %% \begin{center} \only<1>{star map}\only<2>{\sout{star map} star chart}\end{center}
      \begin{center}star map?\end{center}
    \end{column}
    \begin{column}{0.3\textwidth}
      \includegraphics[height=2cm]{inputs/maphat.jpg}
      %% \begin{center}\only<1>{hat map}\only<2>{\sout{hat map} map hat}\end{center}
      \begin{center}hat map?\end{center}
    \end{column}
  \end{columns}
\end{frame}


%%%%%%%%%%%%%%%%%%%%%%%%%%%
\begin{frame}[fragile,label=P5Lemma]{The Star Map and Hat Map}

  %% \begin{exampleblock}{}
    
  \begin{columns}
    \begin{column}{0.25\textwidth}
      \includegraphics[height=3cm]{inputs/amnotconfus.jpg}
    \end{column}
    \begin{column}{.8\textwidth}
      \begin{itemize}
      \item[star map] 
        $^*:\Con\bB \rightarrow \Con\bA$ is
        congruence generation: %restricted to the set $\Con\bB$:
        \[\beta^* = \Cg^\bA(\beta)\qquad (\forall \, \beta \in \Con\bB)\]
        \vskip3mm
      \item[hat map]
        $\hatmap: \Con\bB \rightarrow \Con\bA$ is
      \[\widehat{\beta} =
        \{(x,y) \in A^2 \mid (ef(x), ef(y))\in \beta,
        \;\forall\, f\in  \Pol_1(\bA) \}.\]
             \note{
               It is not hard to see that $\hatmap$ maps $\Con\bB$ into $\Con\bA$.
               For example, if $(x,y) \in \widehat{\beta}$ and $g\in \Pol_1(\bA)$,
               then for all $f\in \Pol_1(\bA)$ we have $(efg(x),efg(y)) \in \beta$,
               so $(g(x),g(y))\in \widehat{\beta}$.}
      \end{itemize}
    \end{column}
  \end{columns}
  %% \end{exampleblock}
  \vskip1cm
  \visible<2>{{\small The hat map appears in McKenzie's ``Finite Forbidden Lattices''
      paper (Puebla, 1982) where he gives an alternative proof of the $P^5$ Lemma.}}

    %% \end{exampleblock}
  %% \vskip5mm{\small       Ralph McKenzie: {\it Finite forbidden lattices.} Puebla (1982).}
\end{frame}

%%%%%%%%%%%%%%%%%%%%%%%%%%%
\begin{frame}[fragile,label=P5Lemma]{Residuation lemma}
A lemma relating the three maps $\,^*\,$, $\, \resB\,$, and $\,\hatmap$.
\vskip2mm

\begin{lemma} %[D, 2011]
\label{lem:residuation-lemma}
  \begin{enumerate}[(i)]
  \item $^*: \Con\bB \rightarrow \Con\bA$ is a \defn{residuated mapping} with
    \defn{residual} $\resB$.
  \item $\resB : \Con\bA \rightarrow \Con\bB$ is a \defn{residuated mapping} with
    \defn{residual} $\hatmap$.
  \item For all $\alpha \in \Con\bA, \, \beta \in \Con\bB$,
    \[\beta = \alpha\resB \quad \Leftrightarrow  \quad 
    \beta^* \leq \alpha \leq \widehat{\beta}.\]
    In particular, 
    $\beta^*\resB = \beta = \widehat{\beta}\resB$.
  \end{enumerate}
\end{lemma}

\note{
{\small If $X$ and $Y$
  are partially ordered sets, and if 
  $f: X \rightarrow Y$ and 
  $g: Y \rightarrow X$ are order preserving maps, then TFAE:
  \begin{enumerate}[(a)]
  \item $f: X \rightarrow Y$ is a {\it residuated mapping} with {\it residual}
    $g: Y \rightarrow X$;
  \item for all $x\in X,\, y\in Y$,  $f(x) \leq y$ iff $x \leq g(y)$;
  \item $g\circ f \geq \id_X$ and $f\circ g \leq \id_Y$.
  \end{enumerate}
  So, for each $y\in Y$,  $\exists !$ maximum $x\in X$ such that $f(x) \leq y$, and the
  max is given by $g(y)$.
  Thus, {\it (i)} is equivalent to 
  \begin{equation}
    \label{eq:OAi}
    \beta^* \leq \alpha \quad \Leftrightarrow \quad \beta \leq \alpha\resB
    \quad (\forall \, \alpha \in \Con\bA,\; \forall \, \beta \in \Con\bB).
  \end{equation}
  This is easily verified, as follows:  If 
  $\beta^* \leq \alpha$ then
  $\beta = (\beta^*)\resB \leq \alpha\resB$ 
  If $\beta \leq \alpha\resB$ then 
  $\beta^* \leq (\alpha\resB)^* \leq \Cg^\bA(\alpha) = \alpha$.

Statement {\it (ii)} is equivalent to 
  \begin{equation}
    \label{eq:OAii}
    \alpha\resB\leq \beta 
    \quad \Leftrightarrow \quad 
    \alpha \leq \widehat{\beta}
    \quad (\forall \, \alpha \in \Con\bA,\; \forall \, \beta \in \Con\bB).
  \end{equation}
  This is also easy to check.  For, suppose
  $\alpha\resB\leq \beta$ and $(x,y)\in \alpha$. Then $(ef(x), ef(y)) \in \alpha$
  for all $f \in \Pol_1(\bA)$ and $(ef(x), ef(y)) \in B^2$, therefore, 
  $(ef(x), ef(y)) \in \alpha\resB \leq \beta$, so $(x,y) \in \widehat{\beta}$.
  Suppose $\alpha \leq \widehat{\beta}$ and $(x,y) \in \alpha\resB$. 
  Then $(x,y) \in \alpha \leq  \widehat{\beta}$, so 
  $(ef(x), ef(y)) \in \beta$ for all $f\in \Pol_1(\bA)$, including $f=\id_A$, so 
  $(e(x), e(y)) \in \beta$. But $(x, y) \in B^2$, so $(x, y) = (e(x), e(y)) \in
  \beta$.

  Combining~(\ref{eq:OAi}) and~(\ref{eq:OAii}), we obtain statement {\it (iii)} of the lemma.}
}
\end{frame}

%%%%%%%%%%%%%%%%%%%%%%%%%%%
\begin{frame}[fragile,label=P5Lemma]{Residuation/adjunction lemma}
New version...
\vskip2mm
\begin{lemma} %[D, 2011]
\label{lem:residuation-lemma}
%% \vskip-2mm
{\Large \[^* \quad \dashv \quad \resB \quad \dashv \quad \hatmap\] }
%% {\Large $^* \quad \dashv \quad \resB \quad \dashv \quad \hatmap$ }
\end{lemma}
\vskip2mm
\visible<2>{...that is...
\vskip2mm
  \begin{enumerate}[(i)]
  \item $^*: \Con\bB \rightarrow \Con\bA$ is \defn{left adjoint} to $\resB$;\\[5pt]
  \item $\resB : \Con\bA \rightarrow \Con\bB$ is \defn{left adjoint} to $\hatmap$;\\[5pt]
  \item For all $\alpha \in \Con\bA, \, \beta \in \Con\bB$,
    \[\beta = \alpha\resB \quad \Leftrightarrow  \quad 
    \beta^* \leq \alpha \leq \widehat{\beta}.\]
    In particular, 
    $\beta^*\resB = \beta = \widehat{\beta}\resB$.
  \end{enumerate}}
%% \end{lemma}
\end{frame}

%%%%%%%%%%%%%%%%%%%%%%%%%%%
\begin{frame}[fragile,label=P5Lemma]{New Proof of the $P^5$ Lemma}
  \begin{lemma}[\Palfy-\Pudlak, 1980]
The restriction mapping 
\[
\Con\bA \ni \alpha \mapsto \alpha\resB = \alpha \cap B^2 \in \Con \bB
\]
is a complete lattice epimorphism. % of $\Con\bA$ onto $\Con\bB$.
  \end{lemma}
\note{
  Our approach to proving Lemma 1, which is similar to the
  proof of McKenzie, does not reveal any information about
  the permutability of the congruences of $\bA$, unlike the more direct proof
  given by \PAP. }
\begin{proof}
  Recall, for $f: X \to Y$ a monotone function on preorders $X$ and $Y$,\\
  if $f$ has a right (left) adjoint, then $f$ preserves all joins (meets)
  that exist in $X$.\\[5pt]
  By the lemma $\resB$ has both a left and right adjoint.
\end{proof}
\end{frame}



































% \vskip5mm
% \visible<1->{
%   \begin{columns}
%     \begin{column}{0.2\textwidth}
%       \begin{flushright}
%         \includegraphics[height=1.2cm]{qrcodeDeMeoExpansions}
%       \end{flushright}
%     \end{column}
%     \begin{column}{0.9\textwidth}
% {\small   {\it Expansions of finite algebras and their congruence lattices} (to appear)\\
%       \urlprefix  \url{http://arXiv.org/abs/1205.1106}}
%     \end{column}
%   \end{columns}
% }










%% \visible<2->{
%%   \begin{columns}
%%     \begin{column}{0.2\textwidth}
%%       \begin{flushright}
%%         \includegraphics[height=1.5cm]{inputs/qrcodeMcKenzie1983}
%%       \end{flushright}
%%     \end{column}
%%     \begin{column}{0.9\textwidth}
%% {\small       Ralph McKenzie: {\it Finite forbidden lattices.}\\[4pt]
%%      In: Universal algebra and lattice theory ({P}uebla, 1982),\\
%%       \emph{Lecture Notes in Math.}, vol. 1004, pp. 176--205. Springer, Berlin (1983).\\
%%   \url{http://dx.doi.org/10.1007/BFb0063438}}
%%     \end{column}
%%   \end{columns}
%% }
%%   \note{In~\cite{McKenzie:1983}, McKenzie, taking Lemma 1 as a starting point,
%%     develops the foundations of what would become tame congruence theory.}












%% \begin{columns}\begin{column}{0.2\textwidth}
%%       \begin{flushright}\includegraphics[height=1.5cm]{inputs/qrcodeP5}\end{flushright}
%%     \end{column}\begin{column}{0.9\textwidth}
%%     {\small P{\'e}ter P{\'a}l \Palfy\ and Pavel Pudl{\'a}k: {\it Congruence lattices of finite algebras\\ and
%%         intervals in subgroup lattices of finite groups.}\\[4pt]
%%       Algebra Universalis \textbf{11}(1), 22--27 (1980).\\
%%       \url{http://dx.doi.org/10.1007/BF02483080}}
%%   \end{column}
%% \end{columns}
