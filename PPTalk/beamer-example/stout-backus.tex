% $Header: /cvsroot/latex-beamer/latex-beamer/solutions/conference-talks/conference-ornate-20min.en.tex,v 1.6 2004/10/07 20:53:08 tantau Exp $

\documentclass{beamer}
%%\documentclass[handout]{beamer}

% This file is a solution template for:
% - Talk at a conference/colloquium.
% - Talk length is about 20min.
% - Style is ornate.
% Copyright 2004 by Till Tantau <tantau@users.sourceforge.net>.

\mode<presentation>
{
  \usetheme{Warsaw}
  % or ...
  %\setbeamercovered{transparent}
  % or whatever (possibly just delete it)
}


\usepackage[english]{babel}
\usepackage[latin1]{inputenc}
\usepackage{times}
\usepackage[T1]{fontenc}
\usepackage[mathscr]{euscript}
\newcommand{\mbb}{\mathbb}
\newcommand{\mbf}{\mathbf}                    
\newcommand{\msc}{\mathscr}

\newcommand{\Hawaii}{Hawai\kern.05em`\kern.05em\relax i}
\newcommand{\Manoa}{M\=anoa}
\definecolor{MyDarkBlue}{rgb}{0.2,0.2,0.7}
\newcommand{\emphcyan}[1]{\textcolor{MyDarkBlue}{\textbf{#1}}}
\newcommand{\la}{\langle}
\newcommand{\ra}{\rangle}
\newcommand{\op}{\operatorname}
\newcommand{\join}{\vee}
\newcommand{\meet}{\wedge}
\renewcommand{\Join}{\bigvee}
\newcommand{\Meet}{\bigwedge}
\newcommand{\vare}{\varepsilon}

\title[Lattices of Theories]{Lattices of Theories}  

\author[Adaricheva and Nation]{Kira Adaricheva and J.B. Nation}
\institute[Stern College and U. \Hawaii] 
{Stern College for Women\\
 University of \Hawaii\ at \Manoa}

\date[New York 2009] 
{New York, October 2009}

\subject{Lattices of theories}

\begin{document}

\thicklines

%%000

\begin{frame}
  \titlepage
\end{frame}

%%010101

\begin{frame}
  \frametitle{Lattices of theories}

\uncover<1->{%
\begin{block}{What is a lattice?}
   {}
}

\uncover<2->{%
A \emph{lattice} is an ordered set in which every pair of elements has
a g.l.b. and a l.u.b.
}
\end{block}

\uncover<3->{%
\begin{block}{Examples}
   {}
}

\begin{itemize}
\uncover<4->{%
   \item Subsets of a set
}

\uncover<5->{%
   \item Closed subsets of a topology.
}

\uncover<6->{%
   \item Subgroups of a group.
}

\end{itemize}
\end{block}
\end{frame}

%%020202

\begin{frame}
  \frametitle{Picture}

\uncover<1->{%

\begin{center}
\hspace{2cm}
\setlength{\unitlength}{2.0cm}
\begin{picture}(2,2)

\put(0,0){\circle*{0.15}}
\put(-1,1){\circle*{0.15}}
\put(1,1){\circle*{0.15}}
\put(0,2){\circle*{0.15}}
%%%%%
\put(0,0){\line(1,1){1}}
\put(0,0){\line(-1,1){1}}
\put(-1,1){\line(1,1){1}}
\put(1,1){\line(-1,1){1}}
%%%%%
\put(0.2,-0.3){\textbf{\llap{$a \meet b$}}}
\put(0.2,2.2){\textbf{\llap{$a \join b$}}}
\put(-1.3,1){\textbf{$a$}}
\put(1.2,1){\textbf{$b$}}

\end{picture}
\end{center}
}
\end{frame}

%%030303

\begin{frame}
  \frametitle{Lattices of theories}

\uncover<1->{%
\begin{block}{What is a theory?}
   {}
}

\uncover<2->{%
A \emph{theory} is a set of (first-order) sentences closed under deduction.
}
\end{block}

\uncover<3->{%
\begin{block}{Examples}
   {}
}

\begin{itemize}
\uncover<4->{%
   \item Equational theory of semigroups
}

\uncover<5->{%
   \item Equational theory of groups
}

\uncover<6->{%
   \item Equational theory of abelian groups
}

\uncover<7->{%
   \item Equational theory of one-element groups
}

\end{itemize}
\end{block}
\end{frame}

%%040404

\begin{frame}
  \frametitle{Equational theories of groups}

\uncover<1->{%
\begin{block}{}
   {}
}

\begin{itemize}
\uncover<2->{%
\item Lattice of theories is dual to the lattice of models (varieties).
}

\uncover<3->{%
   \item Groups of exponent $k$: \  $x^k \approx 1$
}

\uncover<4->{%
   \item Abelian groups: \  $[x,y] \approx 1$
}

\uncover<5->{%
   \item Nil-2 groups: \  $[[x,y],z] \approx 1$
}

\uncover<6->{%
   \item Nilpotent and solvable group varieties
}

\uncover<7->{%
   \item Theorem (H-N-N):  Every group variety is determined by its exponent
and a set of commutator identities.
}

\uncover<8->{%
   \item There are uncountably many equational theories of groups.
}

\end{itemize}
\end{block}
\end{frame}

%%qqqqqqqqqqqqqqqqqqqqqqqqqqqqqqqqqqqqqqqqqqqqqqqqq
%%050505

\begin{frame}
  \frametitle{Lattice of equational theories of groups}

\uncover<1->{%

\begin{center}
\hspace{2cm}
\setlength{\unitlength}{1.0cm}
\begin{picture}(6,6)

\put(0,0){\circle*{0.15}}
\put(-.25,.25){\circle*{0.05}}
\put(-.5,.5){\circle*{0.05}}
\put(-.75,.75){\circle*{0.05}}
\put(3.75,4){\circle*{0.05}}
\put(4,4){\circle*{0.05}}
\put(4.25,4){\circle*{0.05}}
\put(-1,1){\circle*{0.15}}
\put(-2,2){\circle*{0.15}}
\put(-3,3){\circle*{0.15}}
\put(-2,3){\circle*{0.15}}
\put(-1,3){\circle*{0.15}}
\put(0,3){\circle*{0.15}}
\put(2,3){\circle*{0.15}}
\put(-2,4){\circle*{0.15}}
\put(0,4){\circle*{0.15}}
\put(2,4){\circle*{0.15}}
\put(-1,5){\circle*{0.15}}
\put(0,5){\circle*{0.15}}
\put(0,6){\circle*{0.15}}
%%%%%
\put(-1,1){\line(-1,1){1}}
\put(-2,2){\line(0,1){1}}
\put(-2,3){\line(0,1){1}}
\put(-3,3){\line(1,1){1}}
\put(-2,4){\line(1,1){1}}
\put(-1,5){\line(1,1){1}}
\put(0,4){\line(0,1){1}}
\put(0,5){\line(0,1){1}}
\put(2,3){\line(0,1){1}}
\put(-1,3){\line(0,2){2}}
\put(-1,3){\line(1,1){1}}
\put(0,3){\line(2,1){2}}
\put(0,3){\line(-1,2){1}}
%\put(2,4){\line(-2,2){2}}
\put(2,4){\line(-1,1){1}}
\put(1,5){\line(-1,1){1}}
\put(2,3){\line(-2,1){2}}
%\put(-2,3){\line(2,2){2}}
\put(-2,3){\line(1,1){1}}
\put(-1,4){\line(1,1){1}}
%%%%%
\put(-3.2,3.1){\textbf{\llap{$x^4 \approx 1$}}}
\put(-1.2,5.1){\textbf{\llap{$x^2 \approx 1$}}}
\put(2.2,4.1){\textbf{\rlap{$x^5 \approx 1$}}}
\put(-0.2,6.1){\textbf{\llap{$x \approx 1$}}}
\put(-2.1,1){Nil-2}
\put(-3,2){Abel}
\put(0.2,0){All groups}

\end{picture}
\end{center}
}
\end{frame}

%%060606

\begin{frame}
  \frametitle{Lattice of group theories}

\uncover<1->{%
\begin{block}{}
   {}
The lattice of equational theories of groups is isomorphic to the lattice
of fully invariant subgroups of the free group $\operatorname{FG}(\omega)$ ...
}
\end{block}

\uncover<2->{%
\begin{block}{}
   {}
... so it is modular.
}
\end{block}

\end{frame}

%%qqqqqqqqqqqqqqqqqqqqqqqqqqqqqqqqqqqqqqqqqqqqqqqqq

%%090909

\begin{frame}
  \frametitle{}

\begin{block}{Types of structures}
\uncover<1->{%
   {}
}

\begin{itemize}

\uncover<2->{%
   \item Algebras $\mbf A = \la A, \mathscr F  \ra$
}

\uncover<3->{%
   \item Pure relational structures $\mbf A = \la A,  \mathscr R \ra$
}

\uncover<4->{%
   \item Relational Structures $\mbf A = \la A, \mathscr F, \mathscr R \ra$
}

\end{itemize}

\end{block}

\uncover<5->{%
\begin{block}{Types of classes}
   {}
}

\begin{itemize}

\uncover<6->{%
   \item Equational classes: $s \approx t$
}

\uncover<7->{%
   \item Atomic classes (varieties):  $s \approx t$ and $R(s_1, \dots, s_n)$
}

\uncover<8->{%
   \item Horn classes (quasivarieties):  $\& \alpha_i \implies \beta$
}

\end{itemize}

\end{block}
\end{frame}

%%171717

\begin{frame}
  \frametitle{Varieties (Atomic classes)}

\uncover<1->{%
\begin{block}{}
\begin{itemize}
   {}
}

\uncover<2->{%
   \item Characterized by atomic formulae 
   $s \approx t$ and $R(s_1, \dots, s_n)$
}

\uncover<3->{%
   \item Closed under $\op{HSP}$
}

\uncover<4->{%
   \item Lattice $\op{ATh}(\mbf V)$ of atomic theories extending the
   theory of $\mbf V$ is algebraic.
}

\uncover<5->{%
   \item Example:  group laws $x^4 \approx 1$, $[[x,y],z] \approx 1$
}

\uncover<6->{%
   \item Example:  distributive lattices $x \meet (y \join z) \approx 
   (x \meet y) \join (x \meet z)$
}

\end{itemize}
\end{block}
\end{frame}

%%111

\begin{frame}
  \frametitle{Motivation: Varieties of algebras}

\uncover<1->{%
\begin{block}{Birkhoff}
	If $\mbf V$ is a variety of algebras, then
     \[  \mbf L_v(\mbf V) \cong^d \op{EqTh}(\mbf V) \cong 
     \op{Ficon}(\mbf F_{\mbf V}(\omega)) .\]
\end{block}
}

\uncover<2->{%
\begin{block}{McKenzie et. al.}
    If $\mbf V$ is a variety of algebras, then $\op{EqTh}(\mbf V) \cong 
    \op{Con}(A,\msc F)$ where $\msc F$ includes a binary operation with
    left $0$ and left $1$.
\end{block}
}

\uncover<3->{%
\begin{block}{Lampe et. al.}
    $\op{EqTh}(\mbf V)$ satisfies the Zipper Condition: if $\Join a_i = 1$
    and $a_i \meet c = z$ for all $i$, then $c=z$.
\end{block}
}

\end{frame}

%%222

\begin{frame}
  \frametitle{}

\uncover<1->{%
\begin{block}{Nurakunov}
    $\mbf L \cong \op{EqTh}(\mbf V)$ for some variety $\mbf V$ of algebras 
    iff $\mbf L \cong \op{Con}(A,\msc F)$ for a monoid with a right $0$
    and two unary operations satisfying certain properties.
\end{block}
}

\uncover<2->{%
\begin{block}{Pigozzi}
    If $\mbf L$ is an algebraic lattice with completely join irreducible
    $1$, then $\mbf L \cong \op{EqTh}(\mbf V)$ for some variety $\mbf V$.
\end{block}
}

\end{frame}

%%333

\begin{frame}
  \frametitle{Quasivarieties}

\uncover<1->{%
\begin{block}{}
\begin{itemize}
   {}
}

\uncover<2->{%
   \item Characterized by Horn sentences
   $\ \& \alpha_i \implies \beta$
}

\uncover<3->{%
   \item Closed under $\op{SPU}$
}

\uncover<4->{%
   \item Lattice $\op{QuTh}(\mbf K)$ of quasi-equational theories extending 
   the theory of $\mbf K$ is algebraic.
}

\uncover<5->{%
   \item Example:  group laws $x^4 \approx 1$ and
   \[ x^2 \approx 1 \implies [x,y] \approx 1 \]
}

\uncover<6->{%
   \item Example:  meet semidistributive lattices 
 \[ 
x \meet y \approx  x \meet z \implies x \meet y \approx  x \meet (y \join z) 
 \]
}

\end{itemize}
\end{block}
\end{frame}

%%444

\begin{frame}
  \frametitle{Quasi-equational theories}

\begin{block}{Setup}
\uncover<1->{%
}

\begin{itemize}

\uncover<2->{%
   \item $\mbf T = \op{Con_c}\, \mbf F_{\mbf V}(\omega)$ with $\mbf V$
   a quasivariety of relational structures
}

\uncover<3->{%
   \item $\widehat{\msc E}=\{ \widehat{\vare}: \vare \in \op{End}\,\mbf F \}$, 
   the induced action of the endomorphisms on $\mbf T$.
}

\uncover<4->{%
   \item $\widehat{\msc E}$ is a monoid of \emph{operators}, i.e.,
   $(\join,0)$-homomorphisms.
}

\end{itemize}

\end{block}

\uncover<5->{%
\begin{block}{Theorem}
   $\op{QuTh}(\mbf V) \cong \op{Con}\,(\mbf T, \join, 0, \widehat{\msc E})$
\end{block}
}

\uncover<6->{%
\begin{block}{}
   \emphcyan{Idea:} $\ \& \alpha_i \implies \beta$ \quad corresponds to
   \quad $\op{con} \beta \leq \Join \op{con} \alpha_i$
\end{block}
}

\uncover<7->{%
\begin{block}{Corollary}
   If $\mbf K$ is a quasivariety, then $\op{QuTh}(\mbf K) \cong \op{Con}\,
   ( \mbf S,+,0,\msc F )$ for some semilattice with operators.
\end{block}
}

\end{frame}

%%555

\begin{frame}
  \frametitle{Restrictions}

\uncover<1->{%
\begin{block}{Gorbunov}
   $\op{QuTh}(\mbf K)$ is meet semidistributive and coatomic.
\end{block}
}

\uncover<2->{%
\begin{block}{Dziobiak et. al.}
   $\op{QuTh}(\mbf K)$ supports an equa-interior operator.
\end{block}
}

\uncover<3->{%
\begin{block}{Natural equa-interior operators}
\begin{itemize}
   {}
}

\uncover<4->{%
   \item On $\op{QuTh}(\mbf K)$, $\eta(\Sigma)$ is the restriction of
   $\Sigma$ to atomic formulae.
}

\uncover<5->{%
   \item On $\op{Con}\, ( \mbf S,+,0,\msc F )$,
   $\eta(\theta)$ is the smallest congruence collapsing $0/\theta$.
}

\end{itemize}
\end{block}
\end{frame}

%%515151

\begin{frame}
  \frametitle{Equa-interior operators}

\begin{block}{Definition}
\uncover<1->{%
\begin{itemize}
}

\uncover<2->{%
\item $\eta(x) \leq x$
\item $x \geq y$ implies $\eta(x) \geq \eta(y)$
\item $\eta^2(x)=\eta(x)$
}

\uncover<3->{%
\item $\eta(1)=1$
\item $\eta(x)=u$ for all $x \in X$ implies $\eta(\Join X)=u$
\item $\eta(x) \join (y \meet z) = (\eta(x) \join y) \meet (\eta(x) \join z)$
\item Three more technical conditions you don't want to know.
\end{itemize}
}

\end{block}

\uncover<4->{%
\begin{block}{}
The $\eta$-block of an element $x$ has a least element $\eta(x)$ 
and a greatest element $\tau(x)$.
\end{block}
}

\end{frame}

%%525252

\begin{frame}
  \frametitle{New restrictions}

\uncover<1->{%
{}
}

\uncover<2->{%
\begin{block}{A-N}
   $\op{QuTh}(\mbf K)$ satisfies the condition
	   \[ \tau(x) \leq \tau(c) \ \&\ \eta(z) \leq c \implies
	   \eta(\eta(z) \join \tau(x \meet z)) \leq c   \]
\end{block}
}

\uncover<3->{%
\begin{block}{Infinite version}
   $\op{QuTh}(\mbf K)$ satisfies the condition
   \[ \Meet_{x \in X}\tau(x) \leq \tau(c) \ \&\ \eta(z) \leq c \implies
   \eta(\eta(z) \join \Meet_{x \in X}\tau(x \meet z)) \leq c   \]
\end{block}
}

\end{frame}

%%xxxxxxxxxxxxxxxxxxxxxxxxxxxxxxxxxxxxxxxxxxxxxxxx
%%666

\begin{frame}
  \frametitle{A lattice not representable as $\op{QuTh}(\mbf K)$}

\uncover<1->{%

\begin{center}
\hspace{2cm}
\setlength{\unitlength}{1.0cm}
\begin{picture}(2,4)

\put(0,0){\circle*{0.15}}
\put(-1,1){\circle*{0.15}}
\put(1,1){\circle*{0.15}}
\put(-1,2){\circle*{0.15}}
\put(1,2){\circle*{0.15}}
\put(0,2){\circle*{0.15}}
\put(0,3){\circle*{0.15}}
\put(1,3){\circle*{0.15}}
\put(1,4){\circle*{0.15}}
%%%%%
\put(0,0){\line(1,1){1}}
\put(0,0){\line(-1,1){1}}
\put(-1,1){\line(0,1){1}}
\put(-1,1){\line(1,1){1}}
\put(1,1){\line(0,1){1}}
\put(1,1){\line(-1,1){1}}
\put(-1,2){\line(1,1){1}}
\put(1,2){\line(-1,1){1}}
\put(0,2){\line(0,1){1}}
\put(0,2){\line(1,1){1}}
\put(0,3){\line(1,1){1}}
\put(1,3){\line(0,1){1}}
%%%%%
\put(-0.2,2.9){\textbf{\llap{$c$}}}
\put(1.2,2.9){\textbf{$a = \tau(a)$}}
\put(-1.2,1.9){\textbf{\llap{$x$}}}
\put(1.2,1.9){\textbf{$z$}}
\put(0.2,-0.1){\textbf{$0 = \eta(a)$}}

\end{picture}
\end{center}

\begin{block}{}
   \[ \tau(x) \leq \tau(c) \ \&\ \eta(z) \leq c \implies
   \eta(\eta(z) \join \tau(x \meet z)) \leq c   \]
\end{block}
}

\end{frame}

%%777

\begin{frame}
  \frametitle{Lattice representable as $\op{QuTh}(\mbf K)$}

\uncover<1->{%
\begin{block}{A-N}
   If $\mbf S$ has $0$ and $1$, and $\msc G$ is a group of operators 
   on $\mbf S$, then $\op{Con}\,(\mbf S,+,0,\msc G) \cong \op{QuTh}(\mbf K)$ 
   for some $\mbf K$.
\end{block}
}

\uncover<2->{%

\begin{center}
\hspace{4cm}
\setlength{\unitlength}{0.95cm}
\begin{picture}(6,6)

\put(0,0){\circle*{0.15}}
\put(-1,1){\circle*{0.15}}
\put(1,1){\circle*{0.15}}
\put(-2,2){\circle*{0.15}}
\put(2,2){\circle*{0.15}}
\put(0,2){\circle*{0.15}}
\put(-3,3){\circle*{0.15}}
\put(-1,3){\circle*{0.15}}
\put(1,3){\circle*{0.15}}
\put(3,3){\circle*{0.15}}
\put(0,4){\circle*{0.15}}
\put(-1,5){\circle*{0.15}}
\put(1,5){\circle*{0.15}}
\put(0,6){\circle*{0.15}}
%%%%%
\put(0,0){\line(1,1){1}}
\put(0,0){\line(-1,1){1}}
\put(-1,1){\line(-1,1){1}}
\put(-1,1){\line(1,1){1}}
\put(1,1){\line(1,1){1}}
\put(1,1){\line(-1,1){1}}
\put(-2,2){\line(1,1){1}}
\put(-2,2){\line(-1,1){1}}
\put(0,2){\line(1,1){1}}
\put(0,2){\line(-1,1){1}}
\put(2,2){\line(1,1){1}}
\put(2,2){\line(-1,1){1}}
\put(-3,3){\line(3,1){3}}
\put(3,3){\line(-3,1){3}}
\put(-1,3){\line(1,1){1}}
\put(1,3){\line(-1,1){1}}
\put(0,4){\line(1,1){1}}
\put(0,4){\line(-1,1){1}}
\put(-1,5){\line(1,1){1}}
\put(1,5){\line(-1,1){1}}
%%%%%

\end{picture}
\end{center}
}

\end{frame}


%%888

\begin{frame}
  \frametitle{Con(SLO) vs. Quasi-equational theories}

\uncover<1->{%
}

\uncover<2->{%
\begin{block}{Negative}
   The lattice $\omega + 1$ is the congruence lattice of an SLO, but not a 
   lattice of quasi-equational theories.
\end{block}
}

\uncover<3->{%
\begin{block}{Positive}
   Every $\operatorname{Con}(\mbf T,+,0,\msc F)$ is isomorphic to a lattice
   of implicational theories%
}
\uncover<4->{%
   in a language that may not contain equality.
\end{block}
}


\end{frame}

\end{document}


%%yyyyyyyyyyyyyyyyyyyyyyyyyyyyyyyyyyyyyyyyyyyyyyy

An equa-interior operator on a lattice $\L$ should satisfy the following 
properties.
\begin{enumerate}
\item[(I1)] $\eta(x) \leq x$
\item[(I2)] $x \geq y$ implies $\eta(x) \geq \eta(y)$
\item[(I3)] $\eta^2(x)=\eta(x)$
\item[(I4)] $\eta(1)=1$
\item[(I5)] $\eta(x)=u$ for all $x \in X$ implies $\eta(\Join X)=u$
\item[(I6)] $\eta(x) \join (y \meet z) = (\eta(x) \join y) \meet (\eta(x) 
\join z)$
\item[(I7)] If $\L$ is algebraic, then $\eta(\L)$ is an algebraic lattice, and
$x$ is compact in $\eta(\L)$ iff $x \in \eta(\L)$ and $x$ is compact in $\L$.
\item[(I8)] If $X$ is up-directed, then $\eta(\Join X) = \Join \eta(X)$.
\item[(I9)] There is a compact element $w \in \L$ such that $\eta(w)=w$ and
the interval $[w,1]$ is isomorphic to the congruence lattice of a 
semilattice.  (Thus the interval $[w,1]$ is coatomistic.)
%%We could also require that $\eta$ behave like the natural equa-interior
%%operator on that semilattice $\S$.
\end{enumerate}


