\documentclass[wjd,wjdthms,boldvectors]{wjdarticle}
\usepackage{enumerate,amsmath,amssymb,fancyhdr,mathrsfs,amscd}
% \usepackage[mathscr]{euscript}
% \usepackage{marvosym}
% \usepackage{stmaryrd}
\usepackage{makeidx}
\usepackage{graphicx}
\usepackage[all]{xy}
\raggedbottom
\usepackage[myheadings]{fullpage}
\smartqed

\pagestyle{fancy}


%\title{Comprehensive Examination in\\
%  Abstract Algebra\\
%\textcolor{red}{\it corrections and additions}
%}
%\date{2008 November 29}
%\author{William J. DeMeo}

% \newcommand{\cd}{\Crossedbox}
\newcommand{\tensor}{\ensuremath{\otimes}}
% \newcommand{\ex}{\marginpar[{\LARGE \Bicycle}]{{\LARGE \Bicycle}}}
\newcommand{\con}[1]{\ensuremath{\langle #1 \rangle}}
\newcommand{\ii}[1]{{\it #1}}
\newcommand{\power}[1]{\ensuremath{\mathscr{P}(#1)}}
\newcommand{\scrA}{\ensuremath{\mathscr{A}}}
\newcommand{\scrI}{\ensuremath{\mathscr{I}}}
\newcommand{\sA}{\ensuremath{\mathcal{A}}}
\newcommand{\sB}{\ensuremath{\mathcal{B}}}
\newcommand{\sC}{\ensuremath{\mathcal{C}}}
\newcommand{\id}{\mbox{id}}
\newcommand{\im}{\mbox{im}}
\newcommand{\Tr}{\mbox{Tr}}
%\newcommand{\deg}{\mbox{deg}}
\newcommand{\Hom}{\mbox{Hom}}
\newcommand{\End}{\mbox{End}}
\newcommand{\HomR}{\ensuremath{\mbox{Hom}_R}}

   % -------  SPACING BETWEEN PROBLEMS  ---------------------
   % For more (less) space between problems uncomment the 1st (2nd) line below:
     \newcommand{\pspace}{\vspace{5mm}} % <-- extra space
   % \newcommand{\pspace}{}             % <-- no extra space
   % --------------------------------------------------------
\title{{\large Exercises in}\\
{\sc Group Theory}}
%\author{\copyright\ William J.~DeMeo}
\author{William J.~DeMeo}

\begin{document}
\maketitle
%\pagestyle{fancy}
%\lhead{Math 631 -- Smith}  \chead{Homework 5}  \rhead{William DeMeo}
\begin{abstract}
\begin{center}
This document contains solutions to some of the problems appearing on comprehensive
exams in group theory given by the Mathematics Department at the
University of Hawaii over the past two decades. I have done my best to ensure that
the solutions are clear and correct, and that the level of rigor is at least as high
as that expected of students taking the ph.d.~exams.
In solving many of these problems, I benefited from the wisdom and guidance
of Professor William. A number of other professors provided  additional expert advice
and assistance, and I would like thank Ron Brown, Tom Craven, and J.B.~Nation in
particular.   Of course, some typographical and mathematical errors surely remain,
for which I am solely responsible. Nonetheless, I hope this document  will be of some
use to you as you explore the wonderful world of groups.  \\  
Please email comments, suggestions, and corrections to \verb!williamdemeo@gmail.com!.  
\end{center}

\end{abstract}

\tableofcontents

\pagestyle{fancy}

\newpage

%%%%%%%%%%%%%%%%%%%%%%%%%%%%%%%%%%%%%%%%%%%%
%                                          %
%  Group Theory -- 2001 November           %
%                                          %
%%%%%%%%%%%%%%%%%%%%%%%%%%%%%%%%%%%%%%%%%%%%
\section{2001 April 11}
\begin{enumerate}[{\bf 1.}]

%%%% Question 1 -- 2001 April 11 %%%
\item 
\begin{enumerate}[{\bf a.}]
\item Let $\alpha$ be an element of the symmetric group $S_n$ and let $(i_1 \, i_2\dots i_k)$ 
be a cycle in $S_n$.  Prove that 
$\alpha^{-1}(i_1 \, i_2\dots i_k) \alpha = (i_1\alpha \, i_2\alpha \dots i_k)$.
[Note that it is assumed that permutations act on the right: $i\alpha$.]
\item
Show that any five-cycle $\sigma \in S_5$ and any two-cycle $\tau \in S_5$ together
generate $S_5$.
\end{enumerate}

%%%% Question 2 -- 2001 April 11 %%%
\item\protect\hspace{-2mm}\footnotemark
\footnotetext{Same as prob.~5 of April 2006 exam.}
Let $G$ be a group and let $Z$ be its center.
\begin{enumerate}[{\bf a.}]
\item Show that if $G/Z$ is cyclic then $G$ is abelian.
\item Show that any group of order $p^2$, where $p$ is a prime, is abelian.
\item Give an example of a non-abelian group $G$ where $G/Z$ is abelian.
\item Let $\varphi$ be a homomorphism from $G$ to $K$, where $K$ is an abelian group. Let $N$
  be the kernel of $\varphi$ and suppose $N$ is contained in $Z$.  Suppose that there is
  an abelian subgroup $H$ of $G$ such that $\varphi(H) = K$.  Show $G$ is abelian.
\end{enumerate}
\begin{solution}~\\
\underline{Lemma} If $N$

\end{solution}

%%%%%%%%%%%%%%%%%%%%%%%%%%%%%%%%%%%%%%%%%%%%
%                                          %
%  Group Theory -- 2008 November 21        %
%                                          %
%%%%%%%%%%%%%%%%%%%%%%%%%%%%%%%%%%%%%%%%%%%%
%\subsection{2008 November 21}
%\begin{enumerate}[{\bf 1.}]


\end{enumerate}



\bibliographystyle{plain}
\bibliography{wjd}
\end{document}
