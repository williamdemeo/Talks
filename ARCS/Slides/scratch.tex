In particular,
it shows that there is no lattice-theoretical condition stronger than algebraicity
satisfied by all lattices.  We
call such an embedding a ``concrete representation.'' 

Then L is representable if and only if it is the
congruence lattice of the unary algebra (X, F), where F denotes the set of unary maps
of X which respect all equivalences in L.   

unary 
For a given set F of unary maps on X, consider the set \rho(F) of equivalence
relations on X which are respected by every f \in \rho(F).  Then the pair (\lambda,
\rho) define a Galois correspondence between Eq(X) and X^X, and thus $\rho\lambda$ is
a closure operator.  We call a finite lattice L \leq Eq(X) \emph{closed} provided
$\rho\lambda(L) = L$.  Thus If we consider There is a Galois correspondence that is
useful for working For  $\rho(F)$



In 1980, Pudlak and Tuma proved that every finite lattice can be
embedded in the lattice Eq(X) of equivalences of a finite set X. Consequently, we can
assume our finite lattice is such a ``concrete representation.''
