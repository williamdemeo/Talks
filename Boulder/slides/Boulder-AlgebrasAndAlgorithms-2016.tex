%\documentclass[xcolor=dvipsnames,9pt]{beamer} 
\documentclass[xcolor=dvipsnames,9pt,hide notes,mathserif]{beamer}
%\documentclass[xcolor=dvipsnames,9pt,handout,hide notes,mathserif]{beamer}

\usepackage{pgfpages}
\usepackage{listings}
%\usepackage{enumitem}
\newtheorem{Fact}[theorem]{Fact}

%% For creating a handout:
\mode<handout>{\pgfpagesuselayout{4 on 1}[border shrink=5mm]
  \pgfpageslogicalpageoptions{1}{border code=\pgfusepath{stroke}}
  \pgfpageslogicalpageoptions{2}{border code=\pgfusepath{stroke}}
  \pgfpageslogicalpageoptions{3}{border code=\pgfusepath{stroke}}
  \pgfpageslogicalpageoptions{4}{border code=\pgfusepath{stroke}}
}

%% For creating notes for the speaker:
%\setbeameroption{notes on second screen}
%\setbeameroption{show notes}

\setbeamerfont{structure}{family=\rmfamily,shape=\scshape} 
\usepackage{graphicx}
\usepackage{tikz}
\usepackage{scalefnt}
\usepackage{amsmath}%
\usepackage{amsfonts}%
\usepackage{amssymb}%
%(wjd) added stmaryrd and enumerate packages
\usepackage{stmaryrd,enumerate}
\usepackage{graphicx}
\usepackage{comment}
\usetikzlibrary{matrix,arrows}
\usepackage{ulem}
\usepackage{mathrsfs,textcomp}
\setbeamertemplate{navigation symbols}{}
\usepackage{verbatim}
\usepackage{xspace}
\usepackage[mathcal]{euscript}
\usepackage[smaller]{acronym}
\acrodef{lics}[LICS]{Logic in Computer Science}
\acrodef{CSP}[CSP]{constraint satisfaction problem}
\newcommand{\csp}{\acs{CSP}\xspace}

%% \usepackage{euler} 
\usepackage[T1]{fontenc}
\usepackage[english]{babel}
\usepackage[latin1]{inputenc}
\usepackage{times}
% Or whatever. Note that the encoding and the font should match. If T1
% does not look nice, try deleting the line with the fontenc.

% This changes the color of alerted text to blue:
\definecolor{MyDarkBlue}{rgb}{0.2,0.2,0.7}
\definecolor{olivegreen}{cmyk}{0.64,0,0.95,0.40} % PANTONE 582
\setbeamercolor{alerted text}{fg=blue}
\newcommand{\emphcyan}[1]{\textcolor{MyDarkBlue}{\textbf{#1}}}
%\renewcommand{\alert}[1]{\textcolor{olivegreen}{\emph{#1}}}
%% \renewcommand{\alert}[1]{\textcolor{olivegreen}{{\it #1}}}
\renewcommand{\alert}[1]{\textcolor{olivegreen}{#1}}
%\renewcommand{\alert}[1]{\textbf{{\emph{#1}}}}
% (default is red, but my slides are green and I don't like red and green together)

%\usecolortheme[named=OliveGreen]{structure} 
\usecolortheme[named=olivegreen]{structure} 
\setbeamertemplate{items}[square] 
\setbeamertemplate{blocks}[rounded][shadow=false]
\renewcommand{\emph}[1]{\alert{{\it #1}}}

\usepackage{inputs/wjd}

\mode<presentation>{\usetheme{boxes}}  %boxes,Pittsburgh JuanLesPins, PaloAlto, Singapore, Szeged, Warsaw, Boadilla
%\usetheme{Madrid}}
%\usetheme{boxes}  %boxes,Pittsburgh JuanLesPins, PaloAlto, Singapore, Szeged, Warsaw, Boadilla


\title{The Rectangularity Theorem \\
  {\large of Libor Barto and Marcin Kozik}\\
  {\small with applications to small CIBs}}
\author[William DeMeo]{William DeMeo\\
  {\small \url{williamdemeo@gmail.com}}\\[4pt]
  %  {\small Iowa State University}\\[4pt]
  {\footnotesize joint work with}\\[4pt] 
  Cliff Bergman and Josh Thompson
}
\institute{Iowa State University}

\date[Boulder 2016]{ % (optional, should be abbreviation of conference name)
  Workshop on Algebras and Algorithms\\{\small University of Colorado, Boulder, May 19--22}\\[6pt]
}

\subject{Universal Algebra; Lattice Theory; CSP.}% (optional) inserted into PDF info catalog.

\begin{document}
\thicklines

%% \includeonlyframes{titlepage,intro,intro2,intro3,intro4,intro5,intro6,problem1,cib,general1,cube,cp,general2,remaining,examples1,examples2,examples3,examples4,others}
%%\includeonlyframes{intro6}


\frame[label=titlepage]{
  \titlepage
  \hfill {\footnotesize slides available at\\ 
    \hfill \url{https://github.com/williamdemeo/Talks}}
}


%%%%%%%%%%%%%%%%%%%%%%%%%%%%%%%%%%%%%%%%%%%%%%%%%%%%%%%%%%%
%% 1: CSP Intro 1
\frame[label=intro]{
  \frametitle{Definition of CSP}
  \framesubtitle{(naive version)}
  
  {\bf Input}
  \begin{itemize}
  \item \emph{variables:} $\sV = \{v_1, v_2, \dots\}$
  \item \emph{domain:}  $\sD$
  \item \emph{constraints:} $C_1, C_2, \dots$
  \end{itemize}
  \vskip1cm
  %  \underline{Output}
      {\bf Output}
      \begin{itemize}
      \item ``yes'' if there is a \emph{solution}  
        \[
        f : \sV \rightarrow \sD \quad 
        \text{    {\small (an assignment of values to variables that satisfies all $C_i$)}}
        \] 

      \item ``no'' otherwise
      \end{itemize}

}


\begin{frame}
  \frametitle{Definition of CSP}
  \framesubtitle{(jaded version)}

  $\bA = \<A, \sF\>$ is a finite idempotent algebra,
  $\Sub(\bA)$ is all subuniverses of $\bA$. \\[4pt]

  {\it In this talk} $\CSP(\bA)$ denotes the following decision problem:

  \bigskip
  
  An \defn{instance of degree} $n$ of $\CSP(\bA)$ is the tuple $\<\sV, \sA, \sS, \sR\>$ 
  \begin{itemize}
  \item \emph{variables} $\sV = \{0, 1, \dots, n-1\}$;\\[4pt]
  \item \emph{domains} $\sA = \{\bA_0, \bA_1, \dots, \bA_{n-1}\} \subset \Sub(\bA)$  (one for each variable)\\[4pt]
  \item \emph{scope functions} $\sS = (\bs_0, \bs_1, \dots, \bs_{p-1})$ 
    with \emph{constraint arities} $\ar(\sS) = (m_0, m_1, \dots, m_{p-1})$;\\[4pt]
  \item \emph{constraint relations} $\sR = (\bR_0, \bR_1, \dots, \bR_{p-1})$, where
    \[\bR_i \leq \bA_{\bs_i(0)} \times \bA_{\bs_i(1)}\times \cdots \times \bA_{\bs_i(m_i-1)}.\]
  \end{itemize}
  
  \bigskip
  
  A \defn{solution} to $\<\sV, \sA, \sS, \sR\>$ is an assignment
  $f: \sV \to A$ of values to variables that satisfies all constraints. That is,
  \[f\in \myprod_{\sV}A_j
  \quad  \text{and} \quad 
  \Proj_{\bs_i} f \in \bR_i, \;\text{ for each $0\leq i < p$.}\]

  \onslide<2>{ {\bf Notation:} $\nn = \{0,1,\dots, n-1\}$, so the $i$-th scope
    has type $\bs_i : \mm_i \to \nn$ and
    \[ \Proj_{\bs_i} f  = f \circ \bs_i \]}
\end{frame}

\begin{frame}
  \frametitle{Example 1}
  \framesubtitle{...thanks, Ross!}

  Let $\bA=\<\{0,1\}, \{f\}\>$, where
  \[f(x,y,z) = x+y+z \pmod 2.\]

  Consider the ternary relations
  \[  R_0 = \{(0,0,0), (1,1,0), (1,0,1), (0,1,1) \}\]
  \[ R_1 =  \{(1,0,0), (0,1,0), (0,0,1), (1,1,1) \}\]
  %% \begin{align*}
  %%   R_0 &= \{(x,y,z)\in \{0,1\}^3: f(x,y,z)=0 \}\\
  %%   &= \{(0,0,0), (1,1,0), (1,0,1), (0,1,1) \}
  %% \end{align*}
  %% \begin{align*}
  %%   R_1 &= \{(x,y,z)\in \{0,1\}^3: f(x,y,z)=1\}\\
  %%   & = \{(1,0,0), (0,1,0), (0,0,1), (1,1,1) \}
  %% \end{align*}

  \onslide<2->{Each $\bR_i=\<R_i, \{f\}\>$ is a subalgebra of $\bA^3$}
  \onslide<3->{...in fact, they're subdirect.}

  \smallskip

  \onslide<3->{\hfill {\bf notation:} $\bR_i \sdp \bA^3$\phantom{XXXXXXX}}

  \bigskip

  \onslide<4->{So we have a degree 3 instance of $\CSP(\bA)$, where
    \begin{itemize}
    \item \alert{variables:} $\sV = \{0,1,2\}$
    \item \alert{domains:} $A_i =\{0,1\}, \quad i = 0,1,2$
    \item \alert{scope functions:} the identity on $\{0,1,2\}$
    \item \alert{constraint relations:} $\bR_0$ and $\bR_1$
    \end{itemize}
  }
\end{frame}

\tikzstyle{lat} = [circle,draw,inner sep=0.8pt]
\begin{frame}
  \frametitle{Example 1}
  \framesubtitle{$\cap$ and potatoes}
  \begin{center}
    \begin{tikzpicture}[scale=1]
          \draw[red] (0,.5) ellipse (4mm and 12mm);
    \draw[red] (4,.5) ellipse (4mm and 12mm);
    \draw[blue] (2,.5) ellipse (3mm and 20mm);
    \draw[yellow] (6,.5) ellipse (3mm and 30mm);
    \draw[yellow] (8,.5) ellipse (3mm and 30mm);
    \node[lat] (0000) at (0,0) {};
    \node[lat] (0010) at (0,1) {};
    \node[lat] (2000) at (2,0) {};
    \node[lat] (2005) at (2,.5) {};
    \node[lat] (2010) at (2,1) {};
    \node[lat] (2015) at (2,1.5) {};
    \node[lat] (4000) at (4,0) {};
    \node[lat] (4010) at (4,1) {};
    \draw (0000) node [left]{$0$};
    \draw (0010) node [left]{$1$};
    \draw (2000) node [left]{$0$};
    \draw (2005) node [left]{$1$};
    \draw (2010) node [left]{$2$};
    \draw (2015) node [left]{$3$};
    \draw (4000) node [left]{$0$};
    \draw (4010) node [left]{$1$};

      \draw[red,thick] (00) -- (10) -- (20);
      \draw[blue,thick] (00) -- (11) -- (21);
      \draw[green,thick] (01) -- (10) -- (21);
      \draw[yellow,thick] (01) -- (11) -- (20);
      \draw (6.5,0.5) node {(lines are elements of $R_0$)};
    \end{tikzpicture}
    %% \nopagebreak[4]
    \vskip2mm
    \begin{tikzpicture}[scale=1]
          \draw[red] (0,.5) ellipse (4mm and 12mm);
    \draw[red] (4,.5) ellipse (4mm and 12mm);
    \draw[blue] (2,.5) ellipse (3mm and 20mm);
    \draw[yellow] (6,.5) ellipse (3mm and 30mm);
    \draw[yellow] (8,.5) ellipse (3mm and 30mm);
    \node[lat] (0000) at (0,0) {};
    \node[lat] (0010) at (0,1) {};
    \node[lat] (2000) at (2,0) {};
    \node[lat] (2005) at (2,.5) {};
    \node[lat] (2010) at (2,1) {};
    \node[lat] (2015) at (2,1.5) {};
    \node[lat] (4000) at (4,0) {};
    \node[lat] (4010) at (4,1) {};
    \draw (0000) node [left]{$0$};
    \draw (0010) node [left]{$1$};
    \draw (2000) node [left]{$0$};
    \draw (2005) node [left]{$1$};
    \draw (2010) node [left]{$2$};
    \draw (2015) node [left]{$3$};
    \draw (4000) node [left]{$0$};
    \draw (4010) node [left]{$1$};

      \draw[yellow,thick] (01) -- (11) -- (21);
      \draw[red,thick] (00) -- (10) -- (21);
      \draw[green,thick] (01) -- (10) -- (20);
      \draw[blue,thick] (00) -- (11) -- (20);
      \draw (6.5,0.5) node {(lines are elements of $R_1$)};
    \end{tikzpicture}
  \end{center}
  \medskip
  \onslide<2->{Notice for all $i, j \in \{0,1,2\}$,
    \[\Proj_{ij}R_0 = \Proj_{ij}R_1\]}
  \onslide<3->{\hfill    ...yet $R_0 \cap R_1 = \emptyset$. \phantom{XXXXXXX} }

\end{frame}

\begin{frame}
  \frametitle{Example 2}
  \framesubtitle{...thanks, Cliff!}
  
  Let $\bA = \<\{0,1\}, \{m\}\>$, where $m: A^3 \to A$ is a majority operation, 
  \[m(x,x,y)\approx m(x,y,x)\approx m(y,x,x) \approx x.\]
  
  Let $\bR_0, \bR_1 \sdp \bA^3$ with universes
  \begin{align*}
    R_0 &= \{(0,0,0), (0,0,1), (0,1,0), (1,0,0)\},\\
    R_1 &= \{(0,1,1), (1,0,1), (1,1,0), (1,1,1)\}.
  \end{align*}

  This describes the instance of $\CSP(\bA)$ with
  \begin{itemize}
  \item \alert{variables:} $\sV = \{0,1,2\}$
  \item \alert{domains:} $A_i =\{0,1\}, \quad i = 0,1,2$
  \item \alert{scope functions:} the identity on $\{0,1,2\}$
  \item \alert{constraint relations:} $\bR_0$ and $\bR_1$
  \end{itemize}
\end{frame}
\begin{frame}
  \frametitle{Some conveniences}

  Restrict attention to instances where all constraint relation are
  subdirect,
  
  \[\bR_i \sdp \bA_{\bs_i(0)} \times \bA_{\bs_i(1)}\times \cdots \times \bA_{\bs_i(m_i-1)}\]

  \bigskip
  \onslide<2->{Could visualize $(\bs_i, \bR_i)$ as specifying a subalgebra
    of the full product $\myprod_{\sV}\bA_j$ 
    \[\lb \bs_i, \bR_i\rb = \{\ba \in \myprod_{j\in \sV}A_j \mid \Proj_{\bs_i}\ba \in \bR_i\}\]
\hfill {\small (thanks again, Ross!)} \phantom{XX}

\medskip

Convenient because now solutions are the elements in
    $\bigcap_{i\in V}\lb \bs_i, \bR_i\rb$.}

  \bigskip

  \onslide<3->{
    BUT input size is not a function of these ``full'' subdirect products!
  }

  \medskip

  \onslide<3->{(Input size could be defined as the length of a string of
    all tuples in scopes and constraint relations of the instance.)}
\end{frame}

\begin{frame}
  \frametitle{End of Act I}
  \framesubtitle{first intermission}
  pause...

  \bigskip

  \onslide<1->{\hfill ...draw more potatoes... \phantom{XXXXXXXXXXXXXXXXXXXXXXXXX}}

  \bigskip

  \onslide<1->{ \hfill ...give audience chance to escape.}
\end{frame}

\begin{frame} \frametitle{Absorption Theory {\small (for mortals)}}
%  \framesubtitle{Definitions}
  Let $\bA = \<A, F^{\bA}\>$ be a finite algebra in a Taylor variety.

  \bigskip

  Let $t\in \Clo(\bA)$ be a $k$-ary term operation.

  \bigskip

  A subalgebra  %% $\bB = \<B, F^{\bB}\> \leq \bA$ is
  $\bB \leq \bA$ is 
  {\it \alert{absorbing} in $\bA$ with respect to $t$} if 
  \[a\in A, \; b_i \in B \quad \Longrightarrow  \quad
  t^{\bA}(b_0, \dots, b_{j-1}, a, b_{j+1}, \dots, b_{k-1})\in B  \quad (\text{all } j\in \kk)
  \]
  \onslide<2->{%
    Equivalently, $t^{\bA}[B^{j-1}\times A \times B^{k-j}] \subseteq B$,  for all
    $0\leq j < k$, that is,
    \[(\bb, \ba, \bb')\in B^{j-1}\times A \times B^{k-j} \quad \Longrightarrow  \quad
    t^{\bA}(\bb, \ba, \bb')\in B.
    \]}

  \onslide<3->{%

    {\bf Notation:}

    \medskip
    $\bB \absorbing \bA$ means $\bB$ is absorbing in $\bA$ with respect to some term.

    \medskip

    To be explicit about the term, $\bB \absorbing_t \bA$.

    \bigskip

    %% Call $t$ an ``absorbing term'' for $\bB$ in this case.
    %% \bigskip
 
    $\bB \minabsorbing \bA$ means $\bB\absorbing \bA$ and
    $B$ is minimal (with respect to inclusion)
    among absorbing subuniverses of $\bA$. 

    \bigskip
    
    An algebra is \defn{absorption-free} (AF) if it has no proper absorbing
    subalgebras.
  }
\end{frame}

\begin{frame}
  \frametitle{Where are we going with this?}
  
  ``The Absorption Theorem'' of Barto and Kozik (LMCS 2012)

  \bigskip

  Concerns the special class of ``linked'' subdirect products.  

  \bigskip

  {\it Identifies some special cases in which a subdirect product is the full product!}

  \onslide<2->{
  \begin{theorem}[Absorption Theorem]
    If $V$ is an idempotent locally finite variety, then TFAE
    \begin{itemize}
    \item $V$ is a Taylor variety;
    \item if $\bA_0, \bA_1 \in V$ are finite idempotent absorption-free algebras 
      and $\bR \sdp \bA_0 \times \bA_1$ is linked, then $\bR = \bA_0 \times \bA_1$.
    \end{itemize}
  \end{theorem}
  }
  \onslide<3->{
    At Vanderbilt Shanks Workshop (2015),
    Barto presented more joint work with Kozik 
    generalizing the Absorption Theorem to more than two factors.

    \bigskip

    The ``Rectangularity Theorem'' 
    says roughly$^\ast$, a subdirect product of simple nonabelian algebras 
    contains the full product of minimal absorbing subalgebras.
  }

  \onslide<4->{
   \hfill {\small $^\ast$assuming suitable conditions under which the theorem is true.}
    }

  
\end{frame}
  

\begin{frame}
  \frametitle{Linked subdirect products}
  A subdirect product $\bR \sdp \bA_0 \times \bA_1$ is \alert{linked} if
  for all $a, a' \in \Proj_0 R$,
  
  \[\exists\; c_0, c_2, \dots, c_{2n} \in A_0, \quad
  \exists\; c_1, c_3, \dots, c_{2n+1} \in A_1\]
  such that
  \[
  a = c_0, \quad
  (c_{2i},c_{2i+1})\in R,\quad
  (c_{2i+2},c_{2i+1})\in R,\quad c_{2n} = a'
  \]

  %% \begin{align*}
  %% c_0 &= a, \\
  %% (c_{2i},c_{2i+1})&\in R,\\
  %% (c_{2i+2},c_{2i+1})&\in R,\\
  %% c_{2n} &= a'
  %% \end{align*}

  \bigskip
  \onslide<2->{
  %% \hfill  ...potatoes anyone? \phantom{XXXXXXXXXXXXXX}

  \includegraphics[height=3cm]{inputs/amconfus.png}

  \hfill  {\small [todo: insert potato diagram]} \phantom{XXXXXXXXXXXXXX}
  }
\end{frame}


\begin{frame}
  \frametitle{Linked subdirect products}
  \framesubtitle{for algebraists}
  {\bf Notation:}

  \medskip
  For $\bR \sdp \bA_0 \times \bA_1$, let $\etaR_i$ denote the kernel of the
  $i$-th projection of $\bR$. That is,
  \[
  \etaR_i = \ker(\bR \onto \bA_i) = \{(\br, \br')\in R^2 \mid \Proj_i \br =
  \Proj_i \br'\}
  \]
  Let $R^{-1} = \{(y,x) \in A_1 \times A_0 \mid (x,y) \in R\}$.

  \bigskip
  \onslide<2->{
  The following are equivalent:
  \begin{itemize}
  \item $\bR \sdp \bA_0 \times \bA_1$ is linked;
  \item $\etaR_0\join \etaR_1 = 1_R$;
  \item if $a, a' \in \Proj_0 R$, then $(a,a')$ is in the transitive closure of $R\circ R^{-1}$.
  \end{itemize}
  }

  \bigskip


\end{frame}
%%%% ALTERNATIVE (SHORTER) VERSION:
%% It is easy to see that if
%% $\bR \sdp \bA_0 \times \bA_1$ and $\etaR_i := \ker(\bR \onto \bA_i)$, then 
%% $\bR$ is linked if and only if
%% $\etaR_0\join \etaR_1 = 1_R$.
\begin{frame} \frametitle{Properties of Absorption I}
Absorption has nice properties...
  \begin{itemize}
  \item (transitivity) $\bC \absorbing \bB \absorbing \bA \; \Longrightarrow \; \bC \absorbing \bA$
  \item (closure under nonempty $\cap$ and finite products)
    
    \medskip
    If $\bB \absorbing_f \bA$ and $\bC \absorbing_g \bA$ and $B \cap C\neq \emptyset$, then 
    $\bB \cap \bC\absorbing \bA$.
    %$ is absorbing in $

    \medskip

    If $\bB_0 \absorbing_f \bA_0$ and $\bB_1 \absorbing_g \bA_1$,
    then $\bB_0\times \bB_1 \absorbing_t \bA_0\times \bA_1$. 
    %% $\bB_0\times \bB_1$ is absorbing in $\bA_0\times \bA_1$ with respect to $f\star g$.

    \smallskip
    {\small ...with respect to $t = f\star g$ in both cases.}

    \medskip
    
  \end{itemize}

  \begin{overprint}
    \onslide<2|handout:1>
    
        If $f: A^\ell\to A$ and $g: A^m\to A$, then
        $f \star g$  is the $\ell m$-ary operation 
        \[f(g(a_{1 1}, \dots, a_{1 m}), g(a_{2 1}, \dots, a_{2 m}), \dots,  g(a_{\ell 1}, \dots, a_{\ell m}))\]
    \onslide<3-|handout:2>
    More generally, 
    if $\bB_i\absorbing_{t_i}\bA_i$ for $0\leq i < n$, then
    $\myprod \bB_i \absorbing_s \myprod \bA_i$.
    
    %% If $\bB_i\absorbing_{t_i}\bA_i$ (resp, $\bB_i\minabsorbing_{t_i}\bA_i$) for each $0\leq i < n$,
    %% then

    %% \medskip
    %% $\myprod \bB_i \absorbing_s \myprod \bA_i$ (resp, $\myprod \bB_i \minabsorbing_s \myprod \bA_i$) 
    \smallskip
    {\small ...with respect to $s= t_0\star t_1 \star \cdots \star t_{n-1}$.}

    \bigskip
    An obvious but important consequence:
    \begin{center}
    {\it A finite product of finite idempotent algebras is AF if each factor is AF.}
    \end{center}
  \end{overprint}
  \onslide<4|handout:2>{
    \bigskip
        {\bf Restriction Lemma.}
        
      If $\bB \absorbing_t \bA$ and $\bC \leq \bA$ and $D = B\cap C \neq \emptyset$,
      then $\bD\absorbing \bC$ with respect to the restriction of $t$ to $C$.
  }
\end{frame}

\begin{frame} \frametitle{Properties of Absorption II}
  \framesubtitle{LSD Lemmas}
    
\begin{lemma}[LSD 1]
\label{lem:gen-abs1}
If $\bB_i\absorbing \bA_i$ and $\bR \leq \myprod_{i}\bA_i$ and $R':= R \cap \myprod_i B_i \neq \emptyset$,
then %$\bR \cap \myprod_i \bB_i\absorbing \bR$.
$\bR'\absorbing \bR$.
\end{lemma}
{\it Proof.}  $\myprod \bB_i \absorbing_t \myprod \bA_i$, 
{\small  so follows Restriction Lemma if we put} $C = R$.

  \begin{lemma}[LSD 2]
    \label{lem:sdp-general}
    Suppose $\bB_i \minabsorbing \bA_i$ and  $\bR \sdp \myprod \bA_i$.
    If $R' := R \cap \myprod B_i \neq \emptyset$, then  
    $\bR' \sdp \myprod \bB_i$.
  \end{lemma}

  \begin{lemma}[LSD 2]
    \label{lem:linked-absorber}
    If $\bR \sdp \bA_0 \times \bA_1$ is linked and 
    $\bS \absorbing \bR$, then $\bS$ is linked.
  \end{lemma}

  
\end{frame}
\begin{frame} \frametitle{Linking is easy}
  \framesubtitle{...sometimes}
  %% When can we expect a subdirect product to be \emph{linked}, as is required 
  %% of $\bR \sdp \bA_0\times \bA_1$ in the statement of the Absorption Theorem?
  In some simple cases we get linking from LSD Lemmas along with 
  the following elementary

  \bigskip
  
  {\bf Fact.}  Suppose $\bR \sdp \bA_0 \times \bA_1$ and let $\etaR_i = \ker(\bR \onto \bA_i)$.
    \begin{enumerate}
    \item 
      If $\bA_0$ is simple, then either $\etaR_0 \join \etaR_1 = 1_R$ or $\etaR_0 \geq \etaR_1$.
    \item If $\bA_0$ and $\bA_1$ are both simple, then either $\etaR_0 \join \etaR_1 = 1_R$
      or $\etaR_0 = 0_R = \etaR_1$.
    \end{enumerate}
  
  \bigskip
  ...so, if both factors are simple, then $\etaR_0 \neq \etaR_1$ gives the linking...

  \bigskip
{\bf Cor 1.}    Let $\bA_0$ and $\bA_1$ be simple. If  $\bR \sdp \bA_0\times \bA_1$
and $\etaR_0 \neq \etaR_1$, then $\bR$ is linked.

  \bigskip

...and if one factor is simple nonabelian and the other abelian, linking is free! 

  \bigskip
{\bf Cor 2.}  If $\bA_0$ is simple nonabelian and $\bA_1$ abelian, then
    every subdirect product of $\bA_0 \times \bA_1$ is linked.

    %% \onslide<2->{
    %%   Moreover, if 
    %%   $\bR \sdp \bA_0 \times \bA_1$ and if $\bB_0 \minabsorbing \bA_0$,
    %%  then  $\bR$ intersects $\bA_0\times \bB_1$ nontrivially and this intersection
    %% forms a linked subdirect product of $\bA_0 \times \bB_1$.}

\end{frame}

%% \begin{frame}  \frametitle{Properties of Absorption III}\end{frame}

\begin{frame} \frametitle{Absorption Theorem: Application}
  Suppose we add to the respective contexts of the last three results the hypothesis
  that the algebras live in an idempotent variety with a Taylor term...

  \bigskip  
  (We will refer to such varieties as ``Taylor varieties'' and we call the
  algebras they contain ``Taylor algebras.'')

  \bigskip  
  ...then the Absorption Theorem (in combination with facts above) yields

  \bigskip  
  {\bf Lemma:}
  Let $\bA_0$ and $\bA_1$ be finite Taylor algebras
  with $\bB_i\minabsorbing \bA_i$ ($i =0,1$)
  and suppose $\bR \sdp \bA_0 \times \bA_1$ and $\etaR_0 \neq \etaR_1$.
    \begin{enumerate}[(i)]
    \item If $\bA_0$ and $\bA_1$ are simple and $R\cap (B_0 \times B_1) \neq \emptyset$, then
      $\bB_0 \times \bB_1\leq \bR$. 
    \item  If $\bA_0$ is simple nonabelian and $\bA_1$ is abelian,
      then $\bB_0 \times \bA_1 \leq \bR$.
    \end{enumerate}

    \bigskip
    \onslide<2->{How is this relevant to CSP?}

    \bigskip
    \onslide<3->{Abelian potatoes can all go in the same sack...}

    \bigskip
    \onslide<4->{...simple nonabelian potatoes cannot.}
    
\end{frame}


\begin{frame} \frametitle{The Rectangularity Theorem}
  \framesubtitle{A Generalization of the Absorption Theorem}

  Barto and Kozik generalized the Absorption Theorem
  to multiple simple nonabelian factors. This is...
  \bigskip
  \onslide<2->{
    
  {\bf The Rectangularity Theorem.}

  \medskip
  Let $\bA_0, \bA_1, \dots, \bA_{n-1}$ be finite algebras in a 
  Taylor variety, $\bB_i \minabsorbing \bA_i$, and
  \begin{itemize}
  \item at most one $\bA_i$ abelian, and all nonabelian factors simple, 
  \item $\bR \sdp \bA_0 \times \bA_1 \times \cdots \times \bA_{n-1}$,
  \item $\etaR_i \neq \etaR_j$ for all $i\neq j$. %where $\etaR_i = \ker(\bR %\onto \bA_i)$,
  \item $\bR'= \bR \cap (\bB_0 \times \bB_1 \times \cdots \times \bB_{n-1})$ is nonempty.
  \end{itemize}
  
  \[\text{Then } \; \bR' = \bB_0 \times \bB_1 \times \cdots \times \bB_{n-1}.\] % \leq \bR$.
  }
\end{frame}


\begin{frame} \frametitle{Rectangularity Theorem}
  \framesubtitle{NOTATION}

  Let $\nn =\{0, 1, 2, \dots, n-1\}$.

  \bigskip

  Let $\sigma' = \nn -\sigma$, when $\sigma$ is a subset of $\nn$.

  \bigskip

  For $\bR \sdp \myprod_{\nn}\bA_i$ %\bA_0 \times \bA_1 \times \cdots \times \bA_{n-1}$, let 
  let
  \[
  \etaR_\sigma = \ker(R \onto \Pi_\sigma A_i) = \{(\br, \br') \in R^2 \mid
  \Proj_\sigma \br = \Proj_\sigma \br' \},
  \]


  If $\sigma\subseteq \nn$, then by
  $\bR \sdp \myprod_\sigma \bA_i \times \myprod_{\sigma'}\bA_i$ we 
  mean
  \[
  \bR\leq \myprod_{\nn} \bA_i, \quad 
\Proj_\sigma\bR = \myprod_\sigma \bA_i, \quad \text{ and }\quad
\Proj_{\sigma'}\bR = \myprod_{\sigma'} \bA_i.
\]
  and we say that $\bR$ is a \defn{subdirect product of} 
  $\myprod_\sigma \bA_i$ and $\myprod_{\sigma'}\bA_i$ in this case.

  \bigskip 

  The subdirect product $\bR \sdp \myprod_\sigma \bA_i \times
  \myprod_{\sigma'}\bA_i$ 
  is said to be \defn{linked} if $\etaR_\sigma \join \etaR_{\sigma'} = 1_R$.

  \bigskip

  We may use $\bR_\sigma$ for $\Proj_\sigma\bR$, the projection 
  of $\bR$ onto coordinates in $\sigma$.

  
\end{frame}

\begin{frame} \frametitle{Rectangularity Theorem}
  \framesubtitle{Lemmas needed for the proof}
  From now on,
    \emph{all algebras are finite and belong to the same Taylor variety}.

    \onslide<2->{    {\bf Lemma 1.}
      
    Let $\bB_i\minabsorbing \bA_i$ for each $i\in \nn $, 
    and let $\nn  = \sigma \cup {\sigma'}$ be a disjoint union.
    
    Assume $\bR$ is a \emph{linked} subdirect product of 
    $\myprod_{\sigma} \bA_i$ and $\myprod_{\sigma'}\bA_i$.
    
    Suppose $R' = R \cap \myprod_i B_i \neq \emptyset$.    Then $\bR' = \myprod_i \bB_i$.
    }

    \bigskip

    \onslide<3->{
    {\bf Lemma 2.} [Kearnes-Kiss, Thm~3.27]
    
    Suppose $\alpha$ and $\beta$ are congruences of a Taylor algebra. Then
    %% $\sansC(\alpha, \alpha; \alpha \meet \beta)$ if and only if
    %% $\sansC(\alpha \join \beta, \alpha \join \beta; \beta)$.
    \[\sansC(\alpha, \alpha; \alpha \meet \beta) \quad \Longleftrightarrow \quad
    \sansC(\alpha \join \beta, \alpha \join \beta; \beta).\]
    }
    \onslide<4->{
    %% \medskip
    %% The Kearnes and Kiss theorem can be used to prove
    %% \medskip

    {\bf Lemma 3.} [Linking Lemma]
    
    Let $n\geq 2$ and $\bB_i \minabsorbing \bA_i$ for all $i\in \nn$. Suppose
    \begin{itemize}
    \item at most one $\bA_i$ abelian, all nonabelian factors simple
    \item $\bR \sdp \bA_0 \times \bA_1 \times \cdots \times \bA_{n-1}$,
    \item $\etaR_i \neq \etaR_j$ for all $i\neq j$.
    \end{itemize}
    Then there exists $k$ such that $\bR\sdp \bA_k \times \bR_{k'}$ is linked.
}
  %% \begin{proof}
%% We have $\bR' \sdp \myprod_\sigma \bB_i \times \myprod_{\sigma'} \bB_i$, and
%% $\bR' \absorbing \bR$, so LSD 3 implies $\bR'$ is linked. 
%% Both $\myprod_{\sigma} \bB_i$ and $\myprod_{\sigma'} \bB_i$ are AF, so the Absorption
%% Theorem implies $\bR' = \myprod_{\sigma}\bB_i \times \myprod_{\sigma'}\bB_i$.
%% \end{proof}

\end{frame}

\begin{frame} \frametitle{Rectangularity Theorem}
  \framesubtitle{Proof Sketch}


    
  {\bf The Theorem.} Assume $\bA_i$ are finite Taylor algebras with $\bB_i \minabsorbing \bA_i$, and
  \begin{itemize}
  \item at most one $\bA_i$ abelian, all nonabelian factors simple, 
  \item $\bR \sdp \bA_0 \times \bA_1 \times \cdots \times \bA_{n-1}$, with
    $\etaR_i \neq \etaR_j$  for $i\neq j$,
  \item $\bR'= \bR \cap (\bB_0 \times \bB_1 \times \cdots \times \bB_{n-1})$ nonempty.
  \end{itemize}
    \[\text{Then } \; \bR' = \bB_0 \times \bB_1 \times \cdots \times \bB_{n-1}.\] % \leq \bR$.

    {\bf Proof sketch.}

    Induct on the number of factors in the product
  $\bA_0 \times \bA_1 \times \cdots \times \bA_{n-1}$.

  \medskip
  For $n=2$ the result holds by an earlier Lemma (slide 16).
  
  \medskip
  Fix $n>2$ and assume for all $2 \leq k < n$ the 
  result holds for $k$ factors. We prove it for subdirect products
  of $n$ factors.

  \medskip
  Fix $\emptyset \subsetneq \sigma\subsetneq \nn$.

  \medskip
  Then $\bR_\sigma = \Proj_\sigma \bR$ and  $\bR_{\sigma'}=\Proj_{\sigma'} \bR$
  satisfy assumptions of RT.

  \medskip
  Induction hypothesis implies
  $\myprod_{\sigma}\bB_i \leq \bR_{\sigma}$ and
  $\myprod_{\sigma'}\bB_i \leq \bR_{\sigma'}$.

  \medskip
  A few more easy steps gives, for all $\emptyset \subsetneq \sigma\subsetneq \nn$,
  \[
  \bR \sdp \bR_{\sigma} \times \bR_{\sigma'}, \quad
  \myprod_\sigma \bB_i \minabsorbing \bR_{\sigma}, \quad
  \myprod_{\sigma'} \bB_i \minabsorbing \bR_{\sigma'}.
  \]
  \medskip
  By Linking Lemma and Absorption Theorem, the proof is complete.

\end{frame}

\begin{frame} \frametitle{Rectangularity Theorem}
  \framesubtitle{Extensions and Application to CSP}
  What if there is more than one abelian factor? 

  \bigskip
  {\bf Cor. 1} Let $\bA_i$ be finite Taylor algebras with
  $\bB_i \minabsorbing \bA_i$ ($i\in \nn$).
  
  Let $\bB_i \minabsorbing \bA_i$ ($i\in \nn$) and $\alpha \subseteq \nn$.  Suppose
  \begin{itemize}
  \item $\bA_i$ is abelian for each $i \in \alpha$,
  \item $\bA_i$ is nonabelian and simple for each $i \in \alpha'$,
  \item $\bR \sdp \bA_0 \times \bA_1 \times \cdots \times \bA_{n-1}$,
  \item $\etaR_i \neq \etaR_j$ for all $i\neq j$, %where $\etaR_i = \ker(\bR \onto \bA_i)$,
  \item $R':= R \cap (B_0 \times B_1 \times \cdots \times B_{n-1}) \neq \emptyset$.
  \end{itemize}
  Then $\bR'= \bR_\alpha  \times \myprod_{\alpha'}\bB_i$.

  \bigskip
  {\bf Proof.}
  
    Suppose $\alpha' = \{i_0, i_1, \dots, i_{m-1}\}$.
    Clearly, 
    $\bR \sdp \bR_\alpha \times \bA_{i_0} \times \bA_{i_1} \times \cdots \times \bA_{i_{m-1}}$. 
    If $\alpha\neq \emptyset$, then the product 
    has a single abelian factor
    $\bR_\alpha \leq \myprod_{\alpha} \bA_i$.
    If $\alpha= \emptyset$, then the product has no abelian factors.
    In either case, the result follows from the RT Theorem.

\end{frame}

\begin{frame} \frametitle{Rectangularity Theorem}
  \framesubtitle{Extensions and Application to CSP}

  Two more observations facilitate application to \csp problems. 

  \bigskip
  {\bf Cor. 2} Let $\bA_i$ be finite Taylor algebras with
  $\bB_i \minabsorbing \bA_i$ ($i\in \nn$).
  
  Let $\bB_i \minabsorbing \bA_i$ for each $i\in \nn$ and suppose
  $\bR$ and $\bS$ are subdirect products of $\myprod_{\nn} \bA_i$.
  Let $\alpha \subseteq \nn$ and assume
  \begin{itemize}
  \item $\bA_i$ is abelian for each $i \in \alpha$,
  \item $\bA_i$ is nonabelian and simple for each $i \notin \alpha$,
  \item $\etaR_i \neq \etaR_j$ for all $i\neq j$,
  \item $R$ and $S$ both intersect $\myprod_{\nn} B_i$ nontrivially, 
  \item there exists $\bx \in R_\alpha \cap S_\alpha$.
  \end{itemize}
  Then $R \cap S \neq \emptyset$.

  {\bf Proof.}
  
    By Cor 1, $\bR'= \bR_\alpha   \times \myprod_{\alpha'}\bB_i$ and
    $\bS'= \bS_\alpha   \times \myprod_{\alpha'}\bB_i$. 
    Therefore, since $\bx \in R_\alpha \cap S_\alpha$, we have
    $\{\bx\} \times \myprod_{\alpha'}\bB_i \subseteq R \cap S$.

\end{frame}

\begin{frame} \frametitle{Rectangularity Theorem}
  \framesubtitle{Extensions and Application to CSP}

  Generalizing to more than two relations is easy...

  \bigskip
  {\bf Cor. 3} Let $\bA_i$ be finite Taylor algebras with
  $\bB_i \minabsorbing \bA_i$ ($i\in \nn$).
  
  Suppose $\{\bR_\ell : 0\leq \ell < m\}$ are subdirect products of 
  $\myprod_{\nn} \bA_i$.

  Let $\alpha \subseteq \nn$, and assume
      \begin{itemize}
      \item $\bA_i$ is abelian for $i \in \alpha$ and nonabelian simple for $i \notin \alpha$,
      \item $\forall \ell \in \mm$, $\forall i\neq j$, 
        $\etaR^\ell_i \neq \etaR^\ell_j$  {\small (where $\etaR^\ell_i := \ker(\bR_\ell \onto \bA_i)$)},
      \item each $R_\ell$ intersects $\myprod B_i$ nontrivially, 
      \item there exists $\bx \in \bigcap \Proj_\alpha R_\ell$.
      \end{itemize}
      Then $\bigcap R_\ell \neq \emptyset$.

\end{frame}

\begin{frame} \frametitle{Concluding Remarks}
  \framesubtitle{Obstacles to Application}

  \begin{enumerate}
  \item<1-> {\bf Nonabelian factors must be simple.}
    This is the most obvious limitation of the theorem and
    we don't yet have a way to overcome it that works in general.
    However, we have some ideas and tools for special cases.

  \item<2-> {\bf Abelian factors must have easy partial solutions.}
    The last two corollaries assume that
    when the given constraint relations are projected onto the abelian factors,
    we can solve the ``partial instance''---that is, an element that satisfies all
    constraint relations after projecting these relations onto the abelian
    factors of the full product. This is not a problem.  Abelian algebras are
    tractable! (cf.~Theorem 7.12 of Hobby \& McKenzie)
    

  \item<3-> {\bf Intersecting mass products.}
    RT and corollaries assume that the universe $R$ of the subdirect
    product in question intersects nontrivially
    with a product $\myprod B_i$ of minimal absorbing subuniverses
    (or ``mass product'').

    \bigskip
    In a CSP instance, there are typically many constraint relations.
    To apply Rectangularity, we have to be sure they all intersect
    a single mass product.
  \end{enumerate}

\end{frame}

\begin{frame} \frametitle{Concluding Remarks}
 \framesubtitle{Some final observations}

  The Rectangularity Theorem states that under certain hypotheses (including nontrivial
  intersections with a single \mas product) the given instance has a solution.

  \bigskip

  Consider the converse.  That is, suppose
  $\sR = (\bR_0, \bR_1,\dots, \bR_{p-1})$ is a list of subdirect products,
  the full intersection of which is nonempty $\bx \in \bigcap_{\pp} R_i$.
  
  \bigskip

  Does it follow that a single \mas product intersects nontrivially with
  $\bigcap_{\pp} R_i$?

  \bigskip
  
  If the answer to this question is yes, then for each \csp instance either
  there's a \mas product intersecting nontrivially with 
  all constraint relations, or the instance has no solution.

  \bigskip
  What's the complexity of deciding whether all relations intersect a common
  \mas product?  Surely easier than deciding whether they intersect at all.

  \bigskip

  \begin{center}
  \onslide<2->{Thank you!}
  \end{center}
  
\end{frame}

\end{document}

  %% \bB_{\nn } \minabsorbing \bR_{\nn } \quad \text{ and } \quad
  %% \bB_n \minabsorbing \bA_n.
  %% \bB_{\nn } \minabsorbing \bR_{\nn } \quad \text{ and } \quad
  %% \bB_n \minabsorbing \bA_n.
  \end{equation}

  Finally, by the Linking Lemma (Lem.~\ref{lem:Link-2}) there
  is a $k$ such that 
  $\bR \sdp \bR_{k} \times \bR_{k'}$ is linked.  Therefore,
  by Lemma~\ref{lem:general-linked} the
  the proof is complete.
\end{proof}

\begin{frame} \frametitle{Proof of Rectangularity Theorem I}


\end{frame}



Here is an easily proved fact that provides some equivalent ways to define ``linked.''
%% \todo{write out the proofs that these are equivalent and maybe insert them here
%%   (unless they are too simple and tedious).}

\end{frame}


  %% \bigskip
  %% You will grow tired of hearing me say,
  %% \begin{center}
  %% ``product of minimal absorbing subuniverses''
  %% \end{center}
  %% Can we agree that a {\bf M}inimal {\bf A}bsorbing {\bf S}ubuniverse is a ``\mas''?
  %% \bigskip
  %% ...and the product of \masses is a ``\mas product''?
  %% \bigskip

\end{frame}


\end{document}

